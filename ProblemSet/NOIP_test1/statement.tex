\documentclass[UTF8]{ctexart}
\usepackage{multirow}
\usepackage[top=3cm,bottom=2.5cm,left=3.5cm,right=3.5cm]{geometry} % 页边距
\begin{document}
\newpage
\section{LaTang's Number}

时间限制:1000 ms

内存限制:256 MB

\subsection{题目描述}

小辣和小汤是好朋友,好朋友当然要快快乐乐一起玩耍啦

一天小辣带来了他最喜欢的一堆数字,小汤也要拿出他最喜欢的数字和小辣玩

现在他们的游戏是这样的,小汤可以从小辣喜欢的数字中选出任意个(不能不选,不选小辣会不高兴)

然后小汤可以拿出一些数字和从小辣手里选出的数字组合

设小辣手里的数字为 $a_1, a_2, a_3 \cdots$

设小汤从小辣手里挑选出的数字为 $a_{i_1},a_{i_2},a_{i_3}\cdots$

设小汤自己拿出的数字为 $b_1, b_2, b_3 \cdots$ (小汤的百宝箱里什么数字都有)

如果 $a_{i_1}\times b_1+a_{i_2}\times b_2+a_{i_3}\times b_3 \cdots = 1$

那么小辣就会非常高兴,否则小辣就不开心了

小汤非常想让小辣高兴,你能告诉小汤他能否让小辣高兴吗

\subsection{输入格式}

第一行一个整数 $T$ 表示数据组数,每组数据有两行

每组数据第一行一个整数 $n$ 表示小辣有几个数字

每组数据第二行 $n$ 个整数 $a_1, a_2\cdots a_n$ 分别表示小辣手里的每个数字

\subsection{输出格式}

每组数据一行,如果小汤能让小辣高兴,输出 "Yes" 否则输出 "No"

\subsection{样例}

\begin{tabular}{|p{6.5cm}|p{6.5cm}|}
\hline
\multicolumn{1}{|c|}{输入数据}                            & \multicolumn{1}{c|}{输出数据} \\ \hline
\begin{tabular}[c]{@{}l@{}}1\\4\\ 16 5 7 23\end{tabular} & Yes                       \\ \hline
\end{tabular}

选 $16, 5$ 有 $16*1+5*(-3)=1$

或者选 $5, 7$ 有 $5*3+7*(-2)=1$


\subsection{数据范围}

$1 \leq T \leq 5$

$b_i$ 为任意整数

$a_i, b_i$ 都可能出现重复

\begin{tabular}{|c|c|c|}
\hline
  数据组数 & $n$                            & $a_i$                                       \\ \hline
  1    & \multirow{3}{*}{$n \leq 10$}   & $1 \leq a_i \leq 10$                        \\ \cline{1-1} \cline{3-3} 
  2    &                                & $1 \leq a_i \leq 10^3$                      \\ \cline{1-1} \cline{3-3} 
  3    &                                & $1 \leq a_i \leq 10^9$                      \\ \hline
  4    & \multirow{3}{*}{$n \leq 10^3$} & $1 \leq a_i \leq 10^3$                      \\ \cline{1-1} \cline{3-3} 
  5    &                                & $1 \leq a_i \leq 10^6$                      \\ \cline{1-1} \cline{3-3} 
  6    &                                & $-10^9 \leq a_i \leq 10^9$                  \\ \hline
  7    & \multirow{4}{*}{$n \leq 10^5$} & \multirow{2}{*}{$1 \leq a_i \leq 10^3$}     \\ \cline{1-1}
  8    &                                &                                             \\ \cline{1-1} \cline{3-3} 
  9    &                                & \multirow{2}{*}{$-10^9 \leq a_i \leq 10^9$} \\ \cline{1-1}
  10   &                                &                                             \\ \hline
\end{tabular}

\newpage
\section{LaTang's Game}

时间限制:1000 ms

内存限制:256 MB

\subsection{题目描述}

小辣和小汤是好朋友,他们又在开开心心一起玩耍啦

小辣和小汤对某些颜色特别敏感,比如黄色(不是

这天他们一起去一个色彩主题的游乐园玩,所谓的色彩主题游乐园,就是游乐园不同场地有不同的主题色,花花绿绿的,比较吸引人

但是这个游乐园是闯关的,你只有通关了某个场地才能去下一个场地

小辣和小汤可喜欢玩了,但是想通关就很累,每次通关都要消耗一点体力

不过聪明的小辣小汤发现了一个诀窍,相同颜色的场地有一定的规律

也就是说小汤和小辣只用一点体力就可以连过好几关颜色一样的场地

小辣和小汤希望你告诉他们最少用多少体力能通关,如果不能到终点则输出 -1

注意,起点也算一关,需要消耗体力

\subsection{输入格式}

第一行五个整数 $n, m, k, s, t$ 分别表示一共有几个场地,有几条通关路径,最多可以连通几关颜色相同的,小辣小汤的起点和终点

第二行 $n$ 个数, $c_1, c_2, \cdots c_n$ 表示每个场地的颜色

接下来 $m$ 行每行一条通关路径包含两个整数 $u, v$ 表示你通关了第 $u$ 个场地可以去第 $v$ 个场地

\subsection{输出格式}

一行一个整数,表示通关需要的最少体力

\subsection{样例}

\begin{tabular}{|p{6.5cm}|p{6.5cm}|}
\hline
\multicolumn{1}{|c|}{输入数据}                                                                  & \multicolumn{1}{c|}{输出数据} \\ \hline
\begin{tabular}[c]{@{}l@{}}5 5 2 1 5\\ 1 2 1 1 1\\ 1 2\\ 2 5\\ 1 3\\ 3 4\\ 4 5\end{tabular} & 2                         \\ \hline
\end{tabular}

有两条路

1->2->5 颜色分别是 1 2 1 消耗体力 3

1->3->4->5 颜色分别是 1 1 1 1 消耗体力 2 (1,3消耗1点 4,5消耗1点)

\subsection{数据范围}

$1 \leq u,v,s,t,c_i \leq n$

保证每通关一个场地至少可以通往一个另外的场地

保证数据不会出现自环

\begin{tabular}{|c|c|c|c|}
\hline
  数据组数 & $n$                            & $m$                                    & $k$                                   \\ \hline
  1    & \multirow{5}{*}{$n \leq 10^3$} & $m = n$                                & $k=1$                                   \\ \cline{1-1} \cline{3-4} 
  2    &                                & $m = n$                                & $1 \leq k \leq 10^3$                  \\ \cline{1-1} \cline{3-4} 
  3    &                                & \multirow{2}{*}{$m \leq 10^4$}         & $k = n$                                 \\ \cline{1-1} \cline{4-4} 
  4    &                                &                                        & \multirow{2}{*}{$1 \leq k \leq 10^3$} \\ \cline{1-1} \cline{3-3}
  5    &                                & $m \leq 4\times 10^5$                  &                                       \\ \hline
  6    & \multirow{5}{*}{$n \leq 10^5$} & $m = n$                                & $1 \leq k \leq 10^5$                  \\ \cline{1-1} \cline{3-4} 
  7    &                                & \multirow{4}{*}{$m \leq 4\times 10^5$} & $k=1$                                 \\ \cline{1-1} \cline{4-4} 
  8    &                                &                                        & $k=n$                                 \\ \cline{1-1} \cline{4-4} 
  9    &                                &                                        & \multirow{2}{*}{$1 \leq k \leq 10^5$} \\ \cline{1-1}
  10   &                                &                                        &                                       \\ \hline
\end{tabular}


\newpage
\section{LaTang's Fight}

时间限制:1000 ms

内存限制:256 MB

\subsection{题目描述}

虽然小辣和小汤一直是好朋友,但没想到吧,今天他们吵架了

他们又在玩一个恶心的游戏

小辣说:正数是世界上最好的数字

小汤说:负数才是宇宙中最美的数字

他们面前有一个数组,要求把数组分成长度 $[L, R]$ 范围内的段段

对于小辣来说,如果一段数组的和为正数,他的高兴值就会增加一点,如果是负数就减少一点

对于小汤来说,恰恰相反,如果一段数组的和为负数,他的高兴值就会增加一点,如果是正数就减少一点

如果一段数组的和为 0 ,并不会影响他们的高兴值

你一定不希望这对好朋友闹掰,请你分别告诉他俩这个数组对他最多能提供多少高兴值,好让他们消消气

\subsection{输入格式}

第一行三个整数 $n,L,R$ 表示数组长度,分割的每段数组长度要求范围

第二行 $n$ 个整数 $a_1, a_2, \cdots a_n$ 表示数组的每个数字

\subsection{输出格式}

第一行一个整数,表示小辣能获得的最大高兴值

第二行一个整数,表示小汤能获得的最大高兴值

\subsection{样例}

\begin{tabular}{|p{6.5cm}|p{6.5cm}|}
\hline
\multicolumn{1}{|c|}{输入数据}                                  & \multicolumn{1}{c|}{输出数据}                     \\ \hline
\begin{tabular}[c]{@{}l@{}}5 1 5\\ 1 -2 0 -2 1\end{tabular} & \begin{tabular}[c]{@{}l@{}}1\\ 2\end{tabular} \\ \hline
\end{tabular}

小辣 [ 1 | -2, 0, -2 | 1 ] 高兴值为 1

小汤 [ 1, -2, 0 | -2, 1 ] 高兴值为 2


\subsection{数据范围}

保证至少存在一种分割方案

\begin{tabular}{|c|c|c|c|}
\hline
数据组数 & $n$                            & $a_i$                                      & $L,R$                                     \\ \hline
1    & \multirow{2}{*}{$n \leq 10^3$} & $-1\leq a_i \leq 1$                        & $L=R=1$                                   \\ \cline{1-1} \cline{3-4} 
2    &                                & $10^-9 \leq a_i \leq 10^9$                 & $1 \leq L \leq R \leq n$                  \\ \hline
3    & \multirow{8}{*}{$n \leq 10^5$} & \multirow{4}{*}{$-1\leq a_i \leq 1$}       & $R-L \leq 20$                             \\ \cline{1-1} \cline{4-4} 
4    &                                &                                            & $L=1,R=n$                                 \\ \cline{1-1} \cline{4-4} 
5    &                                &                                            & \multirow{2}{*}{$1 \leq L \leq R \leq n$} \\ \cline{1-1}
6    &                                &                                            &                                           \\ \cline{1-1} \cline{3-4} 
7    &                                & \multirow{4}{*}{$-10^9 \leq a_i\leq 10^9$} & \multirow{2}{*}{$L=1,R=n$}                \\ \cline{1-1}
8    &                                &                                            &                                           \\ \cline{1-1} \cline{4-4} 
9    &                                &                                            & \multirow{2}{*}{$1 \leq L \leq R \leq n$} \\ \cline{1-1}
10   &                                &                                            &                                           \\ \hline
\end{tabular}

\end{document}