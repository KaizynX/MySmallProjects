\documentclass[12pt]{article}
\usepackage[UTF8]{ctex}
\usepackage{listings}
\usepackage{xcolor}
\usepackage{geometry}
\usepackage{setspace}
\usepackage{fancyhdr}
\usepackage{amsmath}
\geometry{left=2.2cm,right=1.8cm,top=2.0cm,bottom=1.5cm} % 页边距
\setlength{\headheight}{15pt}
\pagestyle{fancy}
\fancyfoot[C]{\thepage} % 页脚
\renewcommand{\baselinestretch}{0.8} % 行间距
% 定义可能使用到的颜色
\definecolor{CPPLight}  {HTML} {686868}
\definecolor{CPPSteel}  {HTML} {888888}
\definecolor{CPPDark}   {HTML} {262626}
\definecolor{CPPBlue}   {HTML} {4172A3}
\definecolor{CPPGreen}  {HTML} {487818}
\definecolor{CPPBrown}  {HTML} {A07040}
\definecolor{CPPRed}    {HTML} {AD4D3A}
\definecolor{CPPViolet} {HTML} {7040A0}
\definecolor{CPPGray}  {HTML} {B8B8B8}
\lstset{
    columns=fixed,       
    numbers=left,                                        % 在左侧显示行号
    breaklines=true,                                     %代码过长则换行
    % frame=shadowbox,                                      %用方框框住代码块
    frame=single,                                      %用方框框住代码块
    % backgroundcolor=\color[RGB]{245,245,244},            % 设定背景颜色
    keywordstyle=\color[RGB]{40,40,255},                 % 设定关键字颜色
    numberstyle=\footnotesize\color{darkgray},           % 设定行号格式
    commentstyle=\it\color[RGB]{0,96,96},                % 设置代码注释的格式
    stringstyle=\rmfamily\slshape\color[RGB]{128,0,0},   % 设置字符串格式
    showstringspaces=false,                              % 不显示字符串中的空格
    language=c++,                                        % 设置语言
    morekeywords={alignas,continute,friend,register,true,alignof,decltype,goto,
    reinterpret_cast,try,asm,defult,if,return,typedef,auto,delete,inline,short,
    typeid,bool,do,int,signed,typename,break,double,long,sizeof,union,case,
    dynamic_cast,mutable,static,unsigned,catch,else,namespace,static_assert,using,
    char,enum,new,static_cast,virtual,char16_t,char32_t,explict,noexcept,struct,
    void,export,nullptr,switch,volatile,class,extern,operator,template,wchar_t,
    const,false,private,this,while,constexpr,float,protected,thread_local,
    const_cast,for,public,throw,std},
    emph={map,set,multimap,multiset,unordered_map,unordered_set,
    unordered_multiset,unordered_multimap,vector,string,list,deque,
    array,stack,forwared_list,iostream,memory,shared_ptr,unique_ptr,
    random,bitset,ostream,istream,cout,cin,endl,move,default_random_engine,
    uniform_int_distribution,iterator,algorithm,functional,bing,numeric,},
    emphstyle=\color{CPPViolet}, 
}

\begin{document}     
\tableofcontents
\newpage

\part{杂项}
\section{ 快读快写}

\section{ 正则表达式}
{\setmainfont{Consolas}
\begin{lstlisting}
char str[];
scanf("%3s", str); // 读取长度为n的字符串
scanf("%[abc]", str); // 读取a,b,c,读到之外的立即停止
scanf("%[a-z0-9]", str); // 同上,读取小写字母和数字
scanf("%*[a-z]%s", str); // 过滤掉小写字母读取
scanf("%[^a-z]", str); // 读取小写字符外字符,^表示非 
\end{lstlisting}
\section{ 随机数}
{\setmainfont{Consolas}
\begin{lstlisting}
#include <random>
// 范围 unsigned int
mt19937 rnd(time(NULL));
mt19937 rnd(chrono::high_resolution_clock::now().time_since_epoch().count());

std::random_device rd;  //获取随机数种子
std::mt19937 gen(rd()); //Standard mersenne_twister_engine seeded with rd()
std::uniform_int_distribution<> dis(0, 9);
std::cout << dis(gen) << endl;

inline ull xorshift128(){
  static U SX=335634763,SY=873658265,SZ=192849106,SW=746126501;
  U t=SX^(SX<<11);
  SX=SY;
  SY=SZ;
  SZ=SW;
  return SW=SW^(SW>>19)^t^(t>>8);
}
inline ull myrand(){return (xorshift128()<<32)^xorshift128();}
\end{lstlisting}

\section{ 计算log2}
{\setmainfont{Consolas}
\begin{lstlisting}
#define log(x) (31-__builtin_clz(x))
// lg2[i] = lg2(i) +1
for(int i = 1; i <= n; ++i) lg2[i] = lg2[i>>1]+1;
// lg2[i] = (int)log2(i)
for(int i = 2; i <= n; ++i) lg2[i] = lg2[i>>1]+1;
\end{lstlisting}
\section{ 快速开根号|牛顿迭代法}
{\setmainfont{Consolas}
\begin{lstlisting}
double sqrt(const double &a) {
  double x = a, y = .0;
  while (abs(x-y) > err) {
    y = x;
    x = .5*(x+a/x);
  }
  return x;
}
\end{lstlisting}

\section{ i/k == j 的 k 的个数}
{\setmainfont{Consolas}
\begin{lstlisting}
for (int i = 1; i <= n; ++i) {
  for (int j = 1, l, r; j <= n; ++j) {
    l = max(1, i/(j+1));
    while (l-1 >= 1 && i/(l-1) == j) --l;
    while (i/l > j) ++l;
    r = i/j;
    while (r+1 <= i && i/(r+1) == j) ++r;
    while (i/r < j) --r;
    if (r-l+1 != i/j-i/(j+1)) {
        cout << i << " " << j << endl;
    }
  }
}
\end{lstlisting}

\section{ 三分法}
示例为凹函数
{\setmainfont{Consolas}
\begin{lstlisting}
while (l < r) {
  int mid = (l+r)>>1;
  if (f(mid) < f(mid+1)) r = mid;
  else l = mid+1;
}
\end{lstlisting}
{\setmainfont{Consolas}
\begin{lstlisting}
while(r-l>5){
	int ml=(l+l+r)/3;
	int mr=(l+r+r)/3;
	if(f(ml)<f(mr))r=mr;
	else l=ml;
}
for (int i = l; i <= r; ++i) res = min(res, f(i));
\end{lstlisting}
{\setmainfont{Consolas}
\begin{lstlisting}
while (r-l > eps) {
  double ml = l+(r-l)/3, mr = r-(r-l)/3;
  if (f(ml) < f(mr)) r = mr;
  else l = ml;
}
\end{lstlisting}

\section{ 三维偏序|CDQ分治}
{\setmainfont{Consolas}
\begin{lstlisting}
void cdq(int l, int r) {
  if (l >= r) return;
  int mid = (l+r)>>1, i = l, j = mid+1;
  cdq(l, mid); cdq(mid+1, r);
  for (int k = l; k <= r; ++k) {
    if (j > r || (i <= mid && a[i].y <= a[j].y)) {
      tree.add(a[i].z, a[i].w);
      b[k] = a[i++];
    } else {
      a[j].f += tree.query(a[j].z);
      b[k] = a[j++];
    }
  }
  for (int k = l; k <= mid; ++k) tree.add(a[k].z, -a[k].w);
  for (int k = l; k <= r; ++k) a[k] = b[k];
}
\end{lstlisting}

\subsection{四维偏序}

CDQ套CDQ

第一维在第一层CDQ合并时标记左右区间

合并后第二维有序,进入第二层CDQ

此时按照正常CDQ合并第三维,用数据结构统计第四维

只有标记左区间的加入数据结构,标记右边区间的更新答案

{\setmainfont{Consolas}
\begin{lstlisting}
void cdq(int l, int r) {
  cdq(l, mid); cdq(mid+1, r);
}

void CDQ(int l, int r) {
  CDQ(l, mid); CDQ(mid+1, r);
  // ...
  cdq(l, r);
}
\end{lstlisting}

\section{ 某区间操作问题}

每次操作可以使一个区间+1(或-1),使得序列为0的操作次数下界 $\frac{\sum\limits_{i=1}^{n+1} | a_i-a_{i-1} |}{2}$

\section{ 分数规划}
分数规划用来求一个分式的极值。

形象一点就是,给出 $a_i$ 和 $b_i$,求一组 $w_i\in{0,1}$,最小化或最大化

$$ \displaystyle\frac{\sum\limits_{i=1}^na_i\times w_i}{\sum\limits_{i=1}^nb_i\times w_i} $$

每种物品有两个权值 $a$ 和 $b$,选出若干个物品使得 $\displaystyle\frac{\sum a}{\sum b}$ 最小/最大。

例如代价为 平均值 或 乘积开个数次根(取ln)

分数规划问题的通用方法是二分。

假设我们要求最大值。二分一个答案 $mid$,然后推式子(为了方便少写了上下界):

$$\displaystyle
\begin{aligned}
&\frac{\sum a_i\times w_i}{\sum b_i\times w_i}>mid\\ 
\Longrightarrow&\sum a_i\times w_i-mid\times \sum b_i\cdot w_i>0\\ 
\Longrightarrow&\sum w_i\times(a_i-mid\times b_i)>0 \end{aligned}$$

那么只要求出不等号左边的式子的最大值就行了。如果最大值比 $0$ 要大,说明 $mid$ 是可行的,否则不可行。


\part{计算几何}
\section{ 向量 坐标 直线 圆 (结构体)}
{\setmainfont{Consolas}
\begin{lstlisting}
struct Point {
  typedef double T;
  T x, y;
  int id;
  Point(){}
  Point(const T &_x, const T &_y, const int &_i = 0) : x(_x), y(_y), id(_i) {}
  friend Point operator + (const Point &p1, const Point &p2) {
    return Point(p1.x+p2.x, p1.y+p2.y, p1.id);
  }
  friend Point operator - (const Point &p1, const Point &p2) {
    return Point(p1.x-p2.x, p1.y-p2.y, p1.id);
  }
  friend Point operator - (const Point &p) {
    return Point(-p.x, -p.y, p.id);
  }
  // a*b b在a的顺负逆正
  friend T operator * (const Point &p1, const Point &p2) {
    return p1.x*p2.y-p1.y*p2.x;
  }
  template <typename TT>
  friend Point operator / (const Point &p, const TT &k) {
    return Point(p.x/k, p.y/k, p.id);
  }
  template <typename TT>
  friend Point operator * (const Point &p, const TT &k) {
    return Point(p.x*k, p.y*k, p.id);
  }
  Point operator += (const Point &p) { return *this = *this+p; }
  Point operator -= (const Point &p) { return *this = *this+p; }
  template <typename TT>
  Point operator *= (const TT &k) { return *this = *this*k; }
  template <typename TT>
  Point operator /= (const TT &k) { return *this = *this/k; }
  friend bool operator < (const Point &p1, const Point &p2) {
    return make_pair(p1.x, p1.y) < make_pair(p2.x, p2.y);
  }
  friend bool operator > (const Point &p1, const Point &p2) {
    return make_pair(p1.x, p1.y) > make_pair(p2.x, p2.y);
  }
  friend bool operator == (const Point &p1, const Point &p2) {
    return p1.x == p2.x && p1.y == p2.y;
  }
  friend bool operator != (const Point &p1, const Point &p2) {
    return p1.x != p2.x || p1.y != p2.y;
  }
  friend istream& operator >> (istream &is, Point &p) {
    return is >> p.x >> p.y;
  }
  friend ostream& operator << (ostream &os, Point &p) {
    return os << p.x << " " << p.y << " " << p.id << endl;
  }
  double length() { return sqrt(1.0*x*x+1.0*y*y); }
  friend double dis(const Point &p1, const Point &p2) { return (p2-p1).length(); }
  double dis(const Point &p) { return (p-*this).length(); }
  friend T dot(const Point &p1, const Point &p2) { return p1.x*p2.x+p1.y*p2.y; }
  T dot(const Point &p) { return x*p.x+y*p.y; }
  friend Point rotate_90_c(const Point &p) { return Point(p.y, -p.x, p.id); }
  Point rotate_90_c() { return Point(y, -x, id); }
  friend double atan(const Point &p) { return atan2(p.y, p.x); }
};

template <typename T = double>
struct Vec { // 三维向量
  T x, y, z;
  Vec(const T &_x = 0, const T &_y = 0, const T &_z = 0) : x(_x), y(_y), z(_z) {}
  double len() { return sqrt(1.0*x*x+1.0*y*y+1.0*z*z); }
  friend Vec operator +(const Vec &v1, const Vec &v2) { return Vec(v1.x+v2.x, v1.y+v2.y, v1.z+v2.z); }
  friend Vec operator -(const Vec &v1, const Vec &v2) { return Vec(v1.x-v2.x, v1.y-v2.y, v1.z-v2.z); }
  friend Vec operator *(const T &k, const Vec &v) { return Vec(k*v.x, k*v.y, k*v.z); }
  friend Vec operator *(const Vec &v, const T &k) { return k*v; }
  friend Vec operator *(const Vec &v1, const Vec &v2) {
    return Vec(
        v1.y*v2.z-v1.z*v2.y,
        v1.z*v2.x-v1.x*v2.z,
        v1.x*v2.y-v1.y*v2.x
    );
  }
  friend T dot(const Vec &v1, const Vec &v2) { return v1.x*v2.x+v1.y*v2.y+v1.z*v2.z; }
  T dot(const Vec &v) { return dot(*this, v); }
  Vec& operator +=(const Vec &v) { return *this = *this+v; }
  Vec& operator -=(const Vec &v) { return *this = *this-v; }
  Vec& operator *=(const T &k) { return *this = *this*k; }
  Vec& operator *=(const Vec &v) { return *this = *this*v; }
  friend istream& operator >>(istream &is, Vec &v) { return is >> v.x >> v.y >> v.z; }
};

inline bool polar_angle1(const Point &p1, const Point &p2) {
  double d1 = atan(p1), d2 = atan(p2); 
  return d1 == d2 ? p1 < p2 : d1 < d2;
}

inline bool polar_angle2(const Point &p1, const Point &p2) {
  auto tmp = p1*p2;
  return tmp == 0 ? p1 < p2 : tmp > 0;
}

inline long long S(const Point &p1, const Point &p2, const Point &p3) {
  return abs(p1.x*p2.y+p1.y*p3.x+p2.x*p3.y-p1.x*p3.y-p1.y*p2.x-p2.y*p3.x);
}

struct Line {
  Point p1, p2;
  Line(){}
  Line(const Point &_p1, const Point &_p2) : p1(_p1), p2(_p2) {}
  friend bool cross(const Line &l1, const Line &l2) {
    #define SJ1(x) max(l1.p1.x, l1.p2.x) < min(l2.p1.x, l2.p2.x) || \
             max(l2.p1.x, l2.p2.x) < min(l1.p1.x, l1.p2.x)
    if (SJ1(x) || SJ1(y)) return false;
    #undef SJ1
    #define SJ2(a, b, c, d) ((a-b)*(a-c))*((a-b)*(a-d)) <= 0
    return SJ2(l1.p1, l1.p2, l2.p1, l2.p2) &&
         SJ2(l2.p1, l2.p2, l1.p1, l1.p2);
    #undef SJ2
  }
  friend bool on_line(const Line &l, const Point &p) {
    return abs((l.p1-l.p2)*(l.p1-p)) < eps;
  }
  friend Point cross_point(const Line &l1, const Line &l2) {
    Point v1 = l1.p2-l1.p1, v2 = l2.p2-l2.p1;
    if (abs(v1*v2) < eps) return Point(0, 0); // no cross_point
    double t = (l2.p1-l1.p1)*v2/(v1*v2);
    return l1.p1+v1*t;
  }
};

struct Circular {
  Point o;
  double r;
  Circular(){}
  Circular(const Point &_o, const double &_r) : o(_o), r(_r) {}
  template <typename T>
  Circular(const T &_x, const T &_y, const double &_r) : o(Point(_x, _y)), r(_r) {}
  friend bool in_cir(const Circular &c, const Point &p) { return dis(c.o, p) <= c.r; }
  bool in_cir(const Point &p) { return dis(o, p) <= r; }
};

inline Circular get_cir(const Point &p1, const Point &p2, const Point &p3) {
  Circular res;
  res.o = cross_point(Line((p1+p2)/2, (p1+p2)/2+(p2-p1).rotate_90_c()),
            Line((p1+p3)/2, (p1+p3)/2+(p3-p1).rotate_90_c()));
  res.r = dis(res.o, p1);
  return res;
}
\end{lstlisting}


\section{ 二维凸包}
{\setmainfont{Consolas}
\begin{lstlisting}
int n;
int stk[N], used[N], tp;
Point p[N];

inline void Andrew() {
  memset(used, 0, sizeof used);
  sort(p+1, p+n+1);
  tp = 0;
  stk[++tp] = 1;
  for (int i = 2; i <= n; ++i) {
    while (tp >= 2 && (p[stk[tp]]-p[stk[tp-1]])*(p[i]-p[stk[tp]]) <= 0)
      used[stk[tp--]] = 0;
    used[i] = 1;
    stk[++tp] = i;
  }
  int tmp = tp;
  for (int i = n-1; i; --i) {
    if (used[i]) continue;
    while (tp >= tmp && (p[stk[tp]]-p[stk[tp-1]])*(p[i]-p[stk[tp]]) <= 0)
      used[stk[tp--]] = 0;
    used[i] = 1;
    stk[++tp] = i;
  }
}
\end{lstlisting}


\section{ 平面最近点对}
{\setmainfont{Consolas}
\begin{lstlisting}
Point a[N];
int n, ansa, ansb;
double mindist;

inline bool cmp_y(const Point &p1, const Point &p2) { return p1.y < p2.y; }

void upd_ans(const Point &p1, const Point &p2) {
  double dist = dis(p1, p2);
  if (dist < mindist) mindist = dist, ansa = p1.id, ansb = p2.id;
}

void rec(int l, int r) {
  if (r-l <= 3) {
    for (int i = l; i < r; ++i)
      for (int j = i+1; j <= r; ++j)
        upd_ans(a[i], a[j]);
    sort(a+l, a+r+1, cmp_y);
    return;
  }

  static Point t[N];
  int m = (l+r)>>1, midx = a[m].x;
  rec(l, m); rec(m+1, r);
  merge(a+l, a+m+1, a+m+1, a+r+1, t, cmp_y);
  copy(t, t+r-l+1, a+l);

  int tsz = 0;
  for (int i = l; i <= r; ++i)
    if (abs(a[i].x-midx) <= mindist) {
      for (int j = tsz; j && a[i].y-t[j].y < mindist; --j)
        upd_ans(a[i], t[j]);
      t[++tsz] = a[i];
    }
}

inline void mindist_pair() {
  sort(a+1, a+n+1);
  mindist = INF;
  rec(1, n);
}
\end{lstlisting}


\section{ 最小圆覆盖|随即增量法}
{\setmainfont{Consolas}
\begin{lstlisting}
inline Circular RIA() {
  Circular cir;
  random_shuffle(a+1, a+n+1);
  for (int i = 1; i <= n; ++i) {
    if (cir.in_cir(a[i])) continue;
    cir = Circular(a[i], 0);
    for (int j = 1; j < i; ++j) {
      if (cir.in_cir(a[j])) continue;
      cir = Circular((a[i]+a[j])/2, dis(a[i], a[j])/2);
      for (int k = 1; k < j; ++k) {
        if (cir.in_cir(a[k])) continue;
        cir = get_cir(a[i], a[j], a[k]);
      }
    }
  }
  return cir;
}
\end{lstlisting}


\part{数据结构}
\section{ 堆}
{\setmainfont{Consolas}
\begin{lstlisting}
struct Heap {
  static const int Maxn = 1e6+7;
  int sz, a[Maxn];
  Heap() { sz = 0; memset(a, 0, sizeof a); }
  inline bool cmp(int x, int y) { return x < y; } // 小根堆
  inline int size() { return sz; }
  inline bool empty() { return sz == 0; }
  inline int top() { return a[1]; }
  inline void push(int x) { a[++sz] = x; swift_up(sz); }
  inline void pop() { swap(a[1], a[sz--]); swift_down(1); }
  inline void swift_up(int p) {
    while(p > 1 && cmp(a[p], a[p>>1])) // a[p] < a[p<<1]
      swap(a[p], a[p>>1]), p >>= 1;
  }
  inline void swift_down(int p) {
    int l, r, s;
    while(true) {
      l = p<<1; r = p<<1|1;
      if(l > sz) break;
      if(r > sz || cmp(a[l], a[r])) s = l; // a[l] < a[r]
      else s = r;
      if(cmp(a[s], a[p])) // a[s] < a[p]
        swap(a[p], a[s]), p = s;
      else break;
    }
  }
};
\end{lstlisting}

\subsection{可删堆}
copyright by axiomofchoice
{\setmainfont{Consolas}
\begin{lstlisting}
template <typename T>
struct Heap{
  priority_queue<T> a,b;  // heap=a-b
  void push(T x){a.push(x);}
  void erase(T x){b.push(x);}
  T top(){
    while(!b.empty() && a.top()==b.top())
      a.pop(),b.pop();
    return a.size() ? a.top() : 0; // ???
  }
  void pop(){
    while(!b.empty() && a.top()==b.top())
      a.pop(),b.pop();
    a.pop();
  }
  T top2(){ // 次大值
    T t=top(); pop();
    T ans=top(); push(t);
    return ans;
  }
  size_t size(){return a.size()-b.size();}
};
\end{lstlisting}

\section{ 二叉查找树}
\section{ 平衡树}
\subsection{Splay}
{\setmainfont{Consolas}
\begin{lstlisting}
struct Splay {
  #define root e[0].ch[1]
  typedef int T;
  struct node {
    T v = 0;
    int ch[2] = {0, 0};
    int fa = 0, sum = 0, cnt = 0;
  } e[N];
  int n;
  void update(int x) { e[x].sum = e[e[x].ch[0]].sum+e[e[x].ch[1]].sum+e[x].cnt; }
  int identify(int x) { return x == e[e[x].fa].ch[1]; } // check left or right child
  void connect(int x, int f, int son) { e[x].fa = f; e[f].ch[son] = x; }
  void rotate(int x) {
    int y = e[x].fa,
        r = e[y].fa,
        rson = identify(y),
        yson = identify(x),
        b = e[x].ch[yson^1];
    connect(b, y, yson);
    connect(y, x, yson^1);
    connect(x, r, rson);
    update(y); update(x);
  }
  void splay(int at, int to) {
    to = e[to].fa;
    int up;
    while((up = e[at].fa) != to) {
      if(e[up].fa != to)
        rotate(identify(up) == identify(at) ? up : at);
      rotate(at);
    }
  }
  int add_point(T v, int fa) {
    ++n; e[n].v = v; e[n].fa = fa; e[n].sum = e[n].cnt = 1;
    return n;
  }
  int find(T v) {
    int now = root, last = 0;
    while (now && e[now].v != v)
      last = now, now = e[now].ch[v > e[now].v];
    splay((now ? now : last), root);
    return now;
  }
  void insert(T v) {
    if (!root) { root = add_point(v, root); return; }
    int now = root, last = 0;
    while (now && e[now].v != v)
      last = now, now = e[now].ch[v > e[now].v];
    if (now) ++e[now].cnt;
    else now = e[last].ch[v > e[last].v] = add_point(v, last);
    splay(now, root);
  }
  void erase(T v) {
    int del = find(v);
    if (!del) return;
    if (e[del].cnt > 1) {
      --e[del].cnt;
      --e[del].sum;
    } else if (!e[del].ch[0]) {
      root = e[del].ch[1];
      e[root].fa = 0;
    } else {
      int oldroot = root;
      splay(nex(e[del].ch[0], 1), root);
      connect(e[oldroot].ch[1], root, 1);
      update(root);
    }
  }
  int rank(T v) { return e[e[find(v)].ch[0]].sum+1; }
  T atrank(int x) {
    if (x > e[root].sum) return -INF;
    int now = root;
    while (true) {
      if (x <= e[e[now].ch[0]].sum) now = e[now].ch[0];
      else if ((x -= e[e[now].ch[0]].sum) <= e[now].cnt) break;
      else x -= e[now].cnt, now = e[now].ch[1];
    }
    splay(now, root);
    return e[now].v;
  }
  // small 0, big 1
  int nex(int x, int opt) { while (e[x].ch[opt]) x = e[x].ch[opt]; return x; }
  T lower(T v, int opt) {
    insert(v);
    T res = e[nex(e[root].ch[opt], opt^1)].v;
    erase(v);
    return res;
  }
  #undef root
};
\end{lstlisting}
区间反转
{\setmainfont{Consolas}
\begin{lstlisting}
struct Splay {
  typedef int T;
  struct node {
    T v = 0;
    int ch[2] = { 0, 0 };
    int fa = 0, sum = 0, cnt = 0, tag = 0;
  } e[N];
  int sz, &root = e[0].ch[1];
  void update(int x) { e[x].sum = e[e[x].ch[0]].sum+e[e[x].ch[1]].sum+e[x].cnt; }
  int identify(int x) { return x == e[e[x].fa].ch[1]; }
  void connect(int x,int f,int son) { e[x].fa = f; e[f].ch[son] = x; }
  void rotate(int x) {
    int y = e[x].fa,
      r = e[y].fa,
      rson = identify(y),
      yson = identify(x),
      b = e[x].ch[yson^1];
    connect(b, y, yson);
    connect(y, x, yson^1);
    connect(x, r, rson);
    update(y); update(x);
  }
  void splay(int at,int to = 0) {
    to = e[to].fa;
    int up;
    while((up = e[at].fa) != to) {
      if(e[up].fa != to)
        rotate(identify(up) == identify(at) ? up : at);
      rotate(at);
    }
  }
  int add_point(T v, int fa) {
    ++sz; e[sz].v = v; e[sz].fa = fa; e[sz].sum = e[sz].cnt = 1;
    return sz;
  }
  int find(int x) {
    if (x > e[root].sum) return -INF;
    int now = root;
    while (true) {
      push_down(now);
      if (x <= e[e[now].ch[0]].sum) now = e[now].ch[0];
      else if ((x -= e[e[now].ch[0]].sum) <= e[now].cnt) break;
      else x -= e[now].cnt, now = e[now].ch[1];
    }
    return now;
  }
  int build(int l, int r, int fa) {
    if (l > r) return 0;
    int mid = (l+r)>>1,
      now = add_point(mid, fa);
    e[now].ch[0] = build(l, mid-1, now);
    e[now].ch[1] = build(mid+1, r, now);
    update(now);
    return now;
  }
  void push_down(int x) {
    if (x && e[x].tag) {
      e[e[x].ch[0]].tag ^= 1;
      e[e[x].ch[1]].tag ^= 1;
      swap(e[x].ch[0], e[x].ch[1]);
      e[x].tag = 0;
    }
  }
  void reverse(int l, int r) {
    int pl = find(l-1+1), pr = find(r+1+1);
    splay(pl); splay(pr, pl);
    e[e[e[root].ch[1]].ch[0]].tag ^= 1;
  }
  void print_LMR(int x) {
    if (!x) return;
    push_down(x);
    print_LMR(e[x].ch[0]);
    if (e[x].v != 0 && e[x].v != n+1)
      write(a[e[x].v]), putchar(' ');
    print_LMR(e[x].ch[1]);
  }
} tree;
\end{lstlisting}

\section{ 李超线段树}

李超线段树是一种用于维护平面直角坐标系内线段关系的数据结构。它常被用来处理这样一种形式的问题:给定一个平面直角坐标系,支持动态插入一条线段,询问从某一个位置 (x,+∞) 向下看能看到的最高的一条线段(也就是给一条竖线,问这条竖线与所有线段的最高的交点。

\section{ 线段树合并}

有 $O(m)$ 棵树(个操作) 结点数量 $O(m\log n)$, 暴力合并,均摊复杂度 $O(\log n)$

{\setmainfont{Consolas}
\begin{lstlisting}
int merge(int x, int y) { // 太难了,现场全重写吧
  if (!x || !y) return x+y;
  int &z = x; // int z = new_node(); // 新建结点?
  lc[z] = merge(lc[x], lc[y]);
  rc[z] = merge(rc[x], rc[y]);
  if (!lc[z] && !rc[z]) tr[z].first = tr[x].first+tr[y].first;
  else push_up(z); // tr[z] = giao(x, y);
  // del(y); // del(x); // 保留结点?
  return z;
}
\end{lstlisting}

\section{ 线段树分裂}

一次分裂复杂度 $O(\log n)$

{\setmainfont{Consolas}
\begin{lstlisting}
template <typename T>
struct Tree {
  int tot, lc[NLOG], rc[NLOG];
  T tr[NLOG];
  void init() { tot = 0; }
  T giao(const T &x, const T &y) { return x+y; }
  void push_up(int i) { tr[i] = giao(tr[lc[i]], tr[rc[i]]); }
  int new_node(T v = 0) {
    ++tot; // assert(++tot < NLOG);
    lc[tot] = rc[tot] = 0;
    tr[tot] = v;
    return tot;
  }
  void add(int x, T v, int l, int r, int &i) {
    if (!i) i = new_node();
    if (l == r) return tr[i] += v, void();
    int mid = (l+r)>>1;
    if (x <= mid) add(x, v, l, mid, lc[i]);
    else add(x, v, mid+1, r, rc[i]);
    push_up(i);
  }
  void merge(int l, int r, int &x, int &y) { // merge y to x
    if (!x || !y) return x += y, void();
    if (l == r) return tr[x] += tr[y], void();
    int mid = (l+r)>>1;
    merge(l, mid, lc[x], lc[y]);
    merge(mid+1, r, rc[x], rc[y]);
    push_up(x); // del(y);
  }
  /*
  int merge(int l, int r, int x, int y) { // new node
    if (!x || !y) return x += y;
    int cur = new_node(), mid = (l+r)>>1;
    if (l == r) return tr[cur] = tr[x]+tr[y], cur;
    lc[cur] = merge(l, mid, lc[x], lc[y]);
    rc[cur] = merge(mid+1, r, rc[x], rc[y]);
    push_up(cur);
    return cur;
  }
  */
  void split(int L, int R, int l, int r, int &x, int &y) { //split x [L, R] to y
    if (!x) return;
    if (L <= l && r <= R) return y = x, x = 0, void();
    if (!y) y = new_node();
    int mid = (l+r)>>1;
    if (L <= mid) split(L, R, l, mid, lc[x], lc[y]);
    if (R >  mid) split(L, R, mid+1, r, rc[x], rc[y]);
    push_up(x); push_up(y);
  }
  T query(int L, int R, int l, int r, int i) {
    if (!i) return 0;
    if (L <= l && r <= R) return tr[i];
    int mid = (l+r)>>1;
    if (R <= mid) return query(L, R, l, mid, lc[i]);
    if (L >  mid) return query(L, R, mid+1, r, rc[i]);
    return giao(query(L, R, l, mid, lc[i]), query(L, R, mid+1, r, rc[i]));
  }
  int query_kth(int k, int l, int r, int i) {
    if (l == r) return l;
    int mid = (l+r)>>1;
    if (k <= tr[lc[i]]) return query_kth(k, l, mid, lc[i]);
    return query_kth(k-tr[lc[i]], mid+1, r, rc[i]);
  }
};
\end{lstlisting}

\section{ 猫树}

所谓 "猫树" 就是一种不支持修改,仅仅支持快速区间询问的一种静态线段树。

构造一棵这样的静态线段树需要 $O(n\log n)$ 次合并操作,但是此时的查询复杂度被加速至 $O(1)$ 次合并操作。

\section{ 吉老师线段树|吉司机线段树}

\subsection{区间最值操作}

1. 区间取min

2. 询问区间min

3. 询问区间和

$O(m \log n)$

{\setmainfont{Consolas}
\begin{lstlisting}
struct Tree {
  struct TreeNode {
    int l, r, mx, se, cnt, tag;
    ll sum;
  } tr[N<<2];
  void push_up(int i) {
    tr[i].sum = tr[i<<1].sum+tr[i<<1|1].sum;
    if (tr[i<<1].mx > tr[i<<1|1].mx) {
      tr[i].mx = tr[i<<1].mx;
      tr[i].cnt = tr[i<<1].cnt;
      tr[i].se = max(tr[i<<1].se, tr[i<<1|1].mx);
    } else if (tr[i<<1].mx < tr[i<<1|1].mx) {
      tr[i].mx = tr[i<<1|1].mx;
      tr[i].cnt = tr[i<<1|1].cnt;
      tr[i].se = max(tr[i<<1|1].se, tr[i<<1].mx);
    } else {
      tr[i].mx = tr[i<<1].mx;
      tr[i].cnt = tr[i<<1].cnt+tr[i<<1|1].cnt;
      tr[i].se = max(tr[i<<1].se, tr[i<<1|1].se);
    }
  }
  void push_tag(int i, int v) {
    if (v >= tr[i].mx) return;
    tr[i].sum -= 1ll*(tr[i].mx-v)*tr[i].cnt;
    tr[i].mx = tr[i].tag = v;
  }
  void push_down(int i) {
    if (tr[i].tag == -1) return;
    push_tag(i<<1, tr[i].tag);
    push_tag(i<<1|1, tr[i].tag);
    tr[i].tag = -1;
  }
  void build(int l, int r, int i = 1) {
    tr[i].l = l; tr[i].r = r; tr[i].tag = -1;
    if (l == r) {
      tr[i].sum = tr[i].mx = a[l];
      tr[i].cnt = 1;
      tr[i].se = -1;
      return;
    }
    int mid = (l+r)>>1;
    build(l, mid, i<<1);
    build(mid+1, r, i<<1|1);
    push_up(i);
  }
  void update_min(int l, int r, int v, int i = 1) {
    if (v >= tr[i].mx) return;
    if (l <= tr[i].l && r >= tr[i].r && v > tr[i].se) return push_tag(i, v);
    push_down(i);
    int mid = (tr[i].l+tr[i].r)>>1;
    if (l <= mid) update_min(l, r, v, i<<1);
    if (r >  mid) update_min(l, r, v, i<<1|1);
    push_up(i);
  }
  int query_max(int l, int r, int i = 1) {
    if (l <= tr[i].l && r >= tr[i].r) return tr[i].mx;
    push_down(i);
    int mid = (tr[i].l+tr[i].r)>>1;
    if (r <= mid) return query_max(l, r, i<<1);
    if (l >  mid) return query_max(l, r, i<<1|1);
    return max(query_max(l, r, i<<1), query_max(l, r, i<<1|1));
  }
  ll query_sum(int l, int r, int i = 1) {
    if (l <= tr[i].l && r >= tr[i].r) return tr[i].sum;
    push_down(i);
    int mid = (tr[i].l+tr[i].r)>>1;
    if (r <= mid) return query_sum(l, r, i<<1);
    if (l >  mid) return query_sum(l, r, i<<1|1);
    return query_sum(l, r, i<<1)+query_sum(l, r, i<<1|1);
  }
};
\end{lstlisting}

1. 给一个区间[L,R] 加上一个数x 

2. 把一个区间[L,R] 里小于x 的数变成x 

3. 把一个区间[L,R] 里大于x 的数变成x 

4. 求区间[L,R] 的和

5. 求区间[L,R] 的最大值

6. 求区间[L,R] 的最小值

$O(m\log^2 n)$

{\setmainfont{Consolas}
\begin{lstlisting}
struct SegmentTree {
#define rt tr[i]
#define ls tr[i<<1]
#define rs tr[i<<1|1]
  typedef int T;
  struct TreeNode {
    int l, r;
    T mn1, mn2, mx1, mx2, cmn, cmx, tag1, tag2, tag3;
    long long sum;
  };
  vector<T> a;
  vector<TreeNode> tr;
  void push_up(int i) {
    rt.sum = ls.sum+rs.sum;
    if (ls.mn1 == rs.mn1) {
      rt.mn1 = ls.mn1;
      rt.cmn = ls.cmn+rs.cmn;
      rt.mn2 = min(ls.mn2, rs.mn2);
    } else if (ls.mn1 < rs.mn1) {
      rt.mn1 = ls.mn1;
      rt.cmn = ls.cmn;
      rt.mn2 = min(ls.mn2, rs.mn1);
    } else if (ls.mn1 > rs.mn1) {
      rt.mn1 = rs.mn1;
      rt.cmn = rs.cmn;
      rt.mn2 = min(ls.mn1, rs.mn2);
    }
    if (ls.mx1 == rs.mx1) {
      rt.mx1 = ls.mx1;
      rt.cmx = ls.cmx+rs.cmx;
      rt.mx2 = max(ls.mx2, rs.mx2);
    } else if (ls.mx1 > rs.mx1) {
      rt.mx1 = ls.mx1;
      rt.cmx = ls.cmx;
      rt.mx2 = max(ls.mx2, rs.mx1);
    } else if (ls.mx1 < rs.mx1) {
      rt.mx1 = rs.mx1;
      rt.cmx = rs.cmx;
      rt.mx2 = max(ls.mx1, rs.mx2);
    }
  }
  // 1 2 3 -> min, max, other
  void push_tag(int i, T add1, T add2, T add3) {
    if (rt.mn1 == rt.mx1) {
      add1 == add3 ? add1 = add2 : add2 = add1; // 不应被其他值的标记作用
      rt.sum += 1ll*rt.cmn*add1;
    } else {
      rt.sum += 1ll*rt.cmn*add1+1ll*rt.cmx*add2
          +(rt.r-rt.l+1ll-rt.cmn-rt.cmx)*add3;
    }
    if (rt.mn2 == rt.mx1) rt.mn2 += add2;
    else if (rt.mn2 != INF) rt.mn2 += add3;
    if (rt.mx2 == rt.mn1) rt.mx2 += add1;
    else if (rt.mx2 != -INF) rt.mx2 += add3;
    rt.mn1 += add1; rt.mx1 += add2;
    rt.tag1 += add1; rt.tag2 += add2; rt.tag3 += add3;
  }
  void push_down(int i) {
    T mn = min(ls.mn1, rs.mn1);
    T mx = max(ls.mx1, rs.mx1);
    push_tag(i<<1  , ls.mn1 == mn ? rt.tag1 : rt.tag3, ls.mx1 == mx ? rt.tag2 : rt.tag3, rt.tag3);
    push_tag(i<<1|1, rs.mn1 == mn ? rt.tag1 : rt.tag3, rs.mx1 == mx ? rt.tag2 : rt.tag3, rt.tag3);
    rt.tag1 = rt.tag2 = rt.tag3 = 0;
  }
  template <typename TT> void build(int n, TT arr[]) {
    a.resize(1);
    a.insert(a.end(), arr+1, arr+n+1);
    tr.resize(n*4+1);
    build(1, n, 1);
  }
  void build(int n, T val = 0) {
    a.resize(n+1, val);
    tr.resize(n*4+1);
    build(1, n, 1);
  }
  void build(int l, int r, int i) {
    rt.l = l; rt.r = r;
    rt.tag1 = rt.tag2 = rt.tag3 = 0;
    if (l == r) {
      rt.sum = rt.mn1 = rt.mx1 = a[l];
      rt.mn2 = INF; rt.mx2 = -INF;
      rt.cmn = rt.cmx = 1;
      return;
    }
    int mid = (l+r)>>1;
    build(l, mid, i<<1); build(mid+1, r, i<<1|1);
    push_up(i);
  }
  void update_add(int l, int r, T v, int i = 1) {
    if (rt.r < l || rt.l > r) return;
    if (rt.l >= l && rt.r <= r)
      return push_tag(i, v, v, v);
    push_down(i);
    update_add(l, r, v, i<<1); update_add(l, r, v, i<<1|1);
    push_up(i);
  }
  void update_max(int l, int r, T v, int i = 1) {
    if (rt.r < l || rt.l > r || rt.mn1 >= v) return;
    if (rt.l >= l && rt.r <= r && rt.mn2 > v)
      return push_tag(i, v-rt.mn1, 0, 0);
    push_down(i);
    update_max(l, r, v, i<<1); update_max(l, r, v, i<<1|1);
    push_up(i);
  }
  void update_min(int l, int r, T v, int i = 1) {
    if (rt.r < l || rt.l > r || rt.mx1 <= v) return;
    if (rt.l >= l && rt.r <= r && rt.mx2 < v)
      return push_tag(i, 0, v-rt.mx1, 0);
    push_down(i);
    update_min(l, r, v, i<<1); update_min(l, r, v, i<<1|1);
    push_up(i);
  }
  long long query_sum(int l, int r, int i = 1) {
    if (rt.r < l || rt.l > r) return 0;
    if (rt.l >= l && rt.r <= r) return rt.sum;
    push_down(i);
    return query_sum(l, r, i<<1)+query_sum(l, r, i<<1|1);
  }
  T query_max(int l, int r, int i = 1) {
    if (rt.r < l || rt.l > r) return -INF;
    if (rt.l >= l && rt.r <= r) return rt.mx1;
    push_down(i);
    return max(query_max(l, r, i<<1), query_max(l, r, i<<1|1));
  }
  T query_min(int l, int r, int i = 1) {
    if (rt.r < l || rt.l > r) return INF;
    if (rt.l >= l && rt.r <= r) return rt.mn1;
    push_down(i);
    return min(query_min(l, r, i<<1), query_min(l, r, i<<1|1));
  }
#undef rt
#undef ls
#undef rs
};
\end{lstlisting}

\subsection{区间历史最值}

\subsection{区间加区间修改}

1. 区间加

2. 区间修改

3. 区间最大值

4. 区间历史最大值

$O(m\log n)$

{\setmainfont{Consolas}
\begin{lstlisting}
struct SegmentTree {
#define rt tr[i]
#define ls tr[i<<1]
#define rs tr[i<<1|1]
  typedef int T;
  struct TreeNode {
    int l, r;
    bool tag;
    T add, cov, mx, hadd, hcov, hmx;
  } tr[N<<2];
  void push_up(int i) {
    rt.mx = max(ls.mx, rs.mx);
    rt.hmx = max(ls.hmx, rs.hmx);
  }
  void plus(int i, T k, T hk) {
    rt.hmx = max(rt.hmx, rt.mx+hk);
    rt.mx += k;
    rt.tag ? rt.hcov = max(rt.hcov, rt.cov+hk), rt.cov += k
        : rt.hadd = max(rt.hadd, rt.add+hk), rt.add += k;
  }
  void cover(int i, T k, T hk) {
    rt.hmx = max(rt.hmx, hk);
    rt.mx = k;
    rt.hcov = max(rt.hcov, hk);
    rt.cov = k;
    rt.tag = 1;
  }
  void push_down(int i) {
    if (rt.add) {
      plus(i<<1  , rt.add, rt.hadd);
      plus(i<<1|1, rt.add, rt.hadd);
      rt.add = rt.hadd = 0;
    }
    if (rt.tag) {
      cover(i<<1  , rt.cov, rt.hcov);
      cover(i<<1|1, rt.cov, rt.hcov);
      rt.tag = 0; rt.hcov = -INF;
    }
  }
  void build(int l, int r, int i = 1) {
    rt.l = l; rt.r = r; rt.tag = false;
    rt.add = rt.hadd = 0;
    rt.hcov = -INF;
    if (l == r) return rt.hmx = rt.mx = 0, void();
    int mid = (l+r)>>1;
    build(l, mid, i<<1); build(mid+1, r, i<<1|1);
    push_up(i);
  }
  T query_max(int l, int r, int i = 1) {
    if (rt.r < l || rt.l > r) return -INF;
    if (l <= rt.l && rt.r <= r) return rt.mx;
    push_down(i);
    return max(query_max(l, r, i<<1), query_max(l, r, i<<1|1));
  }
  T query_hmax(int l, int r, int i = 1) {
    if (rt.r < l || rt.l > r) return -INF;
    if (l <= rt.l && rt.r <= r) return rt.hmx;
    push_down(i);
    return max(query_hmax(l, r, i<<1), query_hmax(l, r, i<<1|1));
  }
  void update_add(int l, int r, T v, int i = 1) {
    if (rt.r < l || rt.l > r) return;
    if (l <= rt.l && rt.r <= r) return plus(i, v, v);
    push_down(i);
    update_add(l, r, v, i<<1); update_add(l, r, v, i<<1|1);
    push_up(i);
  }
  void update_cov(int l, int r, T v, int i = 1) {
    if (rt.r < l || rt.l > r) return;
    if (l <= rt.l && rt.r <= r) return cover(i, v, v);
    push_down(i);
    update_cov(l, r, v, i<<1); update_cov(l, r, v, i<<1|1);
    push_up(i);
  }
#undef rt
#undef ls
#undef rs
};
\end{lstlisting}

\subsection{区间最值操作与历史最值询问同向}

单点查询

{\setmainfont{Consolas}
\begin{lstlisting}
struct SegmentTree {
#define rt tr[i]
#define ls tr[i<<1]
#define rs tr[i<<1|1]
  typedef long long T;
  struct Tag {
    T add, mx;
    Tag(T _add = 0, T _mx = -INF) : add(_add), mx(_mx) {}
    Tag operator +(const Tag &t) const { // 合并tag
      return Tag(max(-INF, add+t.add), max(mx+t.add, t.mx));
    }
    Tag operator *(const Tag &t) const { // 取max
      return Tag(max(add, t.add), max(mx, t.mx));
    }
    Tag& operator +=(const Tag &t) { return *this = *this+t; }
    Tag& operator *=(const Tag &t) { return *this = *this*t; }
  };
  struct TreeNode {
    int l, r;
    Tag his, cur;
  } tr[N<<2];
  void push_tag(int i, Tag hk, Tag k) {
    rt.his *= rt.cur+hk;
    rt.cur += k;
  }
  void push_down(int i) {
    push_tag(i<<1  , rt.his, rt.cur);
    push_tag(i<<1|1, rt.his, rt.cur);
    rt.his = rt.cur = Tag();
  }
  void build(int l, int r, int i = 1) {
    rt.l = l; rt.r = r; rt.his = rt.cur = Tag();
    if (l == r) return rt.his = rt.cur = Tag(a[l]), void();
    int mid = (l+r)>>1;
    build(l, mid, i<<1); build(mid+1, r, i<<1|1);
  }
  // add(val, -INF) cov(-INF, val) max(0, val)
  void update(int l, int r, T a, T x = -INF, int i = 1) {
    if (rt.r < l || rt.l > r) return;
    if (l <= rt.l && rt.r <= r) return push_tag(i, Tag(a, x), Tag(a, x));
    push_down(i);
    update(l, r, a, x, i<<1); update(l, r, a, x, i<<1|1);
  }
  T query_max(int x, int i = 1) {
    if (rt.l == rt.r) return max(rt.cur.add, rt.cur.mx);
    push_down(i);
    int mid = (rt.l+rt.r)>>1;
    if (x <= mid) return query_max(x, i<<1);
    else return query_max(x, i<<1|1);
  }
  T query_hmax(int x, int i = 1) {
    if (rt.l == rt.r) return max(rt.his.add, rt.his.mx);
    push_down(i);
    int mid = (rt.l+rt.r)>>1;
    if (x <= mid) return query_hmax(x, i<<1);
    else return query_hmax(x, i<<1|1);
  }
#undef rt
#undef ls
#undef rs
};
\end{lstlisting}

\subsection{区间最值操作与历史最值询问反向}

1. 区间加

2. 区间max

3. 询问min

4. 历史min

{\setmainfont{Consolas}
\begin{lstlisting}
struct SegmentTree {
#define rt tr[i]
#define ls tr[i<<1]
#define rs tr[i<<1|1]
  typedef int T;
  struct TreeNode {
    int l, r;
    T mn, hmn, se, tag1, htag1, tag2, htag2;
  } tr[N<<2];
  void push_up(int i) {
    rt.hmn = min(ls.hmn, rs.hmn);
    if (ls.mn == rs.mn) {
      rt.mn = ls.mn;
      rt.se = min(ls.se, rs.se);
    } else if (ls.mn < rs.mn) {
      rt.mn = ls.mn;
      rt.se = min(ls.se, rs.mn);
    } else if (ls.mn > rs.mn) {
      rt.mn = rs.mn;
      rt.se = min(ls.mn, rs.se);
    }
  }
  void push_tag(int i, T add1, T hadd1, T add2, T hadd2) {
    rt.hmn = min(rt.hmn, rt.mn+hadd1);
    rt.htag1 = min(rt.htag1, rt.tag1+hadd1);
    rt.mn += add1; rt.tag1 += add1;
    rt.htag2 = min(rt.htag2, rt.tag2+hadd2);
    if (rt.se != INF) rt.se += add2;
    rt.tag2 += add2;
  }
  void push_down(int i) {
    T mn = min(ls.mn, rs.mn);
    push_tag(i<<1  , ls.mn == mn ? rt.tag1 : rt.tag2,
        ls.mn == mn ? rt.htag1 : rt.htag2, rt.tag2, rt.htag2);
    push_tag(i<<1|1, rs.mn == mn ? rt.tag1 : rt.tag2,
        rs.mn == mn ? rt.htag1 : rt.htag2, rt.tag2, rt.htag2);
    rt.tag1 = rt.htag1 = rt.tag2 = rt.htag2 = 0;
  }
  void build(int l, int r, int i = 1) {
    rt.l = l; rt.r = r;
    rt.tag1 = rt.htag1 = rt.tag2 = rt.htag2 = 0;
    if (l == r) {
      rt.hmn = rt.mn = a[l];
      rt.se = INF;
      return;
    }
    int mid = (l+r)>>1;
    build(l, mid, i<<1); build(mid+1, r, i<<1|1);
    push_up(i);
  }
  void update_add(int l, int r, T v, int i = 1) {
    if (rt.r < l || rt.l > r) return;
    if (l <= rt.l && rt.r <= r) return push_tag(i, v, v, v, v);
    push_down(i);
    update_add(l, r, v, i<<1); update_add(l, r, v, i<<1|1);
    push_up(i);
  }
  void update_max(int l, int r, T v, int i = 1) {
    if (rt.r < l || rt.l > r || v <= rt.mn) return;
    if (l <= rt.l && rt.r <= r && v < rt.se)
      return push_tag(i, v-rt.mn, v-rt.mn, 0, 0);
    push_down(i);
    update_max(l, r, v, i<<1); update_max(l, r, v, i<<1|1);
    push_up(i);
  }
  T query_min(int l, int r, int i = 1) {
    if (rt.r < l || rt.l > r) return INF;
    if (l <= rt.l && rt.r <= r) return rt.mn;
    push_down(i);
    return min(query_min(l, r, i<<1), query_min(l, r, i<<1|1));
  }
  T query_hmin(int l, int r, int i = 1) {
    if (rt.r < l || rt.l > r) return INF;
    if (l <= rt.l && rt.r <= r) return rt.hmn;
    push_down(i);
    return min(query_hmin(l, r, i<<1), query_hmin(l, r, i<<1|1));
  }
#undef rt
#undef ls
#undef rs
};
\end{lstlisting}

\section{ K-D Tree | KDT}

1. 将格子(x,y)加上v

2. 求(xl,yl)到(xr,yr)区间和

{\setmainfont{Consolas}
\begin{lstlisting}
struct Node {
  int x, y, v;
} s[N];
bool cmpx(int a, int b) { return s[a].x < s[b].x; }
bool cmpy(int a, int b) { return s[a].y < s[b].y; }
struct KDTree {
  double alpha = 0.725;
  int rt, cur, xl, yl, xr, yr; //rt根结点
  int d[N], siz[N], lc[N], rc[N]; //d=1竖着砍,sz子树大小
  int L[N], R[N], D[N], U[N]; //该子树的界线
  int sum[N]; //维护的二维区间信息(二维区间和)
  int g[N], gt;
  void pia(int x) { //将树还原成序列g
    if (!x) return;
    pia(lc[x]);
    g[++gt] = x;
    pia(rc[x]);
  }
  void push_up(int x) { //更新信息
    siz[x] = siz[lc[x]] + siz[rc[x]] + 1;
    sum[x] = sum[lc[x]] + sum[rc[x]] + s[x].v;
    L[x] = R[x] = s[x].x;
    D[x] = U[x] = s[x].y;
    if (lc[x]) {
      L[x] = min(L[x], L[lc[x]]);
      R[x] = max(R[x], R[lc[x]]);
      D[x] = min(D[x], D[lc[x]]);
      U[x] = max(U[x], U[lc[x]]);
    }
    if (rc[x]) {
      L[x] = min(L[x], L[rc[x]]);
      R[x] = max(R[x], R[rc[x]]);
      D[x] = min(D[x], D[rc[x]]);
      U[x] = max(U[x], U[rc[x]]);
    }
  }
  int build(int l, int r) { //以序列g[l..r]为模板重建树,返回根结点
    if (l > r) return 0;
    int mid = (l + r) >> 1;
    double ax = 0, ay = 0, sx = 0, sy = 0;
    for (int i = l; i <= r; i++) ax += s[g[i]].x, ay += s[g[i]].y;
    ax /= (r - l + 1);
    ay /= (r - l + 1);
    for (int i = l; i <= r; i++) {
      sx += (ax - s[g[i]].x) * (ax - s[g[i]].x);
      sy += (ay - s[g[i]].y) * (ay - s[g[i]].y);
    }
    if (sx > sy)
      nth_element(g + l, g + mid, g + r + 1, cmpx), d[g[mid]] = 1;
    else
      nth_element(g + l, g + mid, g + r + 1, cmpy), d[g[mid]] = 2;
    lc[g[mid]] = build(l, mid - 1);
    rc[g[mid]] = build(mid + 1, r);
    push_up(g[mid]);
    return g[mid];
  }
  void rebuild(int &x) {
    gt = 0;
    pia(x);
    x = build(1, gt);
  }
  bool bad(int x) {
    return alpha * siz[x] <= max(siz[lc[x]], siz[rc[x]]);
  }
  void insert(int x, int y, int v) { //在(x,y)处插入元素
    s[++cur] = {x, y, v};
    insert(rt, cur);
  }
  void insert(int &x, int v) {
    if (!x) return push_up(x = v);
    if (d[x]) insert(s[v].x <= s[x].x ? lc[x] : rc[x], v);
    else insert(s[v].y <= s[x].y ? lc[x] : rc[x], v);
    push_up(x);
    if (bad(x)) rebuild(x);
  }
  int query(int xl, int yl, int xr, int yr) { //查询[x1,x2]×[y1,y2]的区间和
    if (xl > xr) swap(xl, xr);
    if (yl > yr) swap(yl, yr);
    this->xl = xl; this->yl = yl; this->xr = xr; this->yr = yr;
    return query(rt);
  }
  int query(int x) {
    if (!x || xr < L[x] || xl > R[x] || yr < D[x] || yl > U[x]) return 0;
    if (xl <= L[x] && R[x] <= xr && yl <= D[x] && U[x] <= yr) return sum[x];
    int res = 0;
    if (xl <= s[x].x && s[x].x <= xr && yl <= s[x].y && s[x].y <= yr)
      res += s[x].v;
    return query(lc[x]) + query(rc[x]) + res;
  }
};
\end{lstlisting}


\section{ 树状数组}
\subsection{一维}
单点修改区间查询

区间修改单点查询

{\setmainfont{Consolas}
\begin{lstlisting}
template <typename T>
struct BinaryIndexedTree {
  int n;
  vector<T> tr;
  void init(const int &n) { this->n = n; tr = vector<T>(n+1, 0); }
  void add(const int &x, const T &v) { for (int i = x; i <= n; i += i&-i) tr[i] += v; }
  void add(const int &x, const int &y, const T &v) { add(x, v); add(y+1, -v); }
  T query(const int &x) { T res = 0; for (int i = x ; i; i -= i&-i) res += tr[i]; return res; }
  T query(const int &x, const int &y) { return query(y)-query(x-1); }
};
\end{lstlisting}

区间修改区间查询
{\setmainfont{Consolas}
\begin{lstlisting}
struct BIT{
	static const int SIZE=1e6+5;
	ll bit1[SIZE],bit2[SIZE];
	int limit;
	void init(int n=SIZE-1){
		limit=n;
		for(int i=1;i<=n;i++)bit1[i]=bit2[i]=0;
	}
	BIT(){init();}
	void add(ll *bit,int x,ll v){
		while(x<=limit)bit[x]+=v,x+=x&-x;
	}
	ll query(ll *bit,int x){
		ll res=0;
		while(x)res+=bit[x],x-=x&-x;
		return res;
	}
	void add(int l,int r,ll v){
		add(bit1,l,v);
		add(bit2,l,v*l);
		add(bit1,r+1,-v);
		add(bit2,r+1,-v*(r+1));
	}
	ll query(int l,int r){
		return (r+1)*query(bit1,r)-query(bit2,r)-l*query(bit1,l-1)+query(bit2,l-1);
	}
}bit;
\end{lstlisting}

\subsection{二维}
\subsubsection{单点修改区间查询}
{\setmainfont{Consolas}
\begin{lstlisting}
template <typename T>
struct BIT_2D {
  int n, m;
  T a[N][N], tr[N][N];
  BIT_2D() { memset(tr, 0, sizeof tr); }
  void init(const int &_n, const int &_m) {
    n = _n; m = _m;
    memset(a, 0, sizeof a);
    memset(tr, 0, sizeof tr);
  }
  void add(const int &x, const int &y, const T &k) {
    a[x][y] += k;
    for (int i = x; i <= n; i += i&-i)
      for (int j = y; j <= m; j += j&-j)
        tr[i][j] += k;
  }
  T query(const int &x, const int &y) {
    return a[x][y];
    // return query(x, y, x, y);
  }
  T query(int r1, int c1, int r2, int c2) {
    if (r1 > r2) swap(r1, r2);
    if (c1 > c2) swap(c1, c2);
    return _query(r2, c2)-_query(r1-1, c2)-_query(r2, c1-1)+_query(r1-1, c1-1);
  }
  T _query(const int &x, const int &y) {
    T res = 0;
    for (int i = x; i; i -= i&-i)
      for (int j = y; j; j -= j&-j)
        res += tr[i][j];
    return res;
  }
};
\end{lstlisting}

\section{ 可持久化数组}

{\setmainfont{Consolas}
\begin{lstlisting}
template <typename T>
struct PersistantArray {
  static const int NN = N*(log2(N)+3);
  int rt[N], ls[NN], rs[NN], val[NN], tot, n;
  void build(const int &n) {
    this->n = n;
    tot = 0;
    rt[0] = build(1, n);
  }
  int build(const int &l, const int &r) {
    int cur = ++tot; assert(tot < NN);
    if (l == r) return val[cur] = a[l], cur;
    int mid = (l+r)>>1;
    ls[cur] = build(l, mid);
    rs[cur] = build(mid+1, r);
    return cur;
  }
  void update(const int &cur, const int &pre, const int &x, const T &k) {
    rt[cur] = update(rt[pre], x, k, 1, n);
  }
  int update(const int &pre, const int &x, const T &k, const int &l, const int &r) {
    int cur = ++tot; assert(tot < NN);
    if (l == x && r == x) return val[cur] = k, cur;
    ls[cur] = ls[pre]; rs[cur] = rs[pre];
    int mid = (l+r)>>1;
    if (x <= mid) ls[cur] = update(ls[pre], x, k, l, mid);
    else rs[cur] = update(rs[pre], x, k, mid+1, r);
    return cur;
  }
  T query(const int &cur, const int &x) {
    return query(rt[cur], x, 1, n);
  }
  T query(const int &cur, const int &x, const int &l, const int &r) {
    if (l == x && r == x) return val[cur];
    int mid = (l+r)>>1;
    if (x <= mid) return query(ls[cur], x, l, mid);
    return query(rs[cur], x, mid+1, r);
  }
};
\end{lstlisting}


\section{ 可持久化线段树(主席树)}

{\setmainfont{Consolas}
\begin{lstlisting}
template <typename T>
struct PersistantSegmentTree {
  static const int NN = N*(log2(N)+5);
  int rt[N], sum[NN], ls[NN], rs[NN], tot, n;
  void build(const int &n) {
    this->n = n;
    tot = 0;
    rt[0] = _build(1, n);
  }
  void update(const int &cur, const int &pre, const T &k) {
    rt[cur] = _update(rt[pre], 1, n, k);
  }
  T query(const int &l, const int &r, const int &k) {
    return _query(rt[l-1], rt[r], 1, n, k);
  }
private:
  int _build(const int &l, const int &r) {
    int cur = ++tot;
    sum[cur] = 0;
    if (l >= r) return cur;
    int mid = (l+r)>>1;
    ls[cur] = _build(l, mid);
    rs[cur] = _build(mid+1, r);
    return cur;
  }
  int _update(const int &pre, const int &l, const int &r, const int &k) {
    int cur = ++tot;
    ls[cur] = ls[pre]; rs[cur] = rs[pre]; sum[cur] = sum[pre]+1;
    if (l >= r) return cur;
    int mid = (l+r)>>1;
    if (k <= mid) ls[cur] = _update(ls[pre], l, mid, k);
    else rs[cur] = _update(rs[pre], mid+1, r, k);
    return cur;
  }
  int _query(const int &u, const int &v, const int &l, const int &r, const int &k) {
    if (l >= r) return l;
    int num = sum[ls[v]]-sum[ls[u]], mid = (l+r)>>1;
    if (num >= k) return _query(ls[u], ls[v], l, mid, k);
    else return _query(rs[u], rs[v], mid+1, r, k-num);
  }
};
\end{lstlisting}

动态开点?oj可以跑本地莫名re
{\setmainfont{Consolas}
\begin{lstlisting}
template <class T>
class PersistantSegmentTree {
 public:
  void init(int n) {
    tot = 0;
    this->n = n;
    rt = vector<int>(1, 0);
    tr = vector<TreeNode>(1, TreeNode{0, 0, 0});
  }
  void update(int cur, int pre, int k, T v = 1) {
    rt.resize(cur+1); // assert rt.size()+1 <= cur
    rt[cur] = update(rt[pre], 1, n, k, v);
  }
  int update(int pre, int l, int r, int k, T v) {
    int cur = ++tot; // assert tot == tr.size()
    tr.emplace_back(tr[pre]);
    tr[cur].sum += v;
    if (l == r) return cur;
    int mid = (l+r)>>1;
    if (k <= mid) tr[cur].lc = update(tr[pre].lc, l, mid, k, v);
    else tr[cur].rc = update(tr[pre].rc, mid+1, r, k, v);
    return cur;
  }
  T query(int l, int r, int ql, int qr) {
    return query(rt[l-1], rt[r], 1, n, ql, qr);
  }
  // [u, v] 段 [ql, qr] 区间和
  T query(int u, int v, int l, int r, int ql, int qr) {
    if (!u && !v) return 0;
    if (l >= ql && r <= qr) return tr[v].sum-tr[u].sum;
    int mid = (l+r)>>1;
    if (qr <= mid) return query(tr[u].lc, tr[v].lc, l, mid, ql, qr);
    if (ql >  mid) return query(tr[u].rc, tr[v].rc, mid+1, r, ql, qr);
    return query(tr[u].lc, tr[v].lc, l, mid, ql, qr)+query(tr[u].rc, tr[v].rc, mid+1, r, ql, qr);
  }
 private:
  struct TreeNode {
    int lc, rc;
    T sum;
  };
  int tot, n;
  vector<int> rt;
  vector<TreeNode> tr;
};
\end{lstlisting}

dingbacode 高地

{\setmainfont{Consolas}
\begin{lstlisting}
template <typename T>
struct PersistantSegmentTree {
  static const int NN = N*(log2(N)+5);
  int rt[N], ls[NN], rs[NN], tot, n;
  T sum[NN];
  void build(const int &n) {
    this->n = n;
    tot = 0;
    rt[0] = _build(1, n);
  }
  void update(const int &cur, const int &pre, const int &k, const T &v) {
    rt[cur] = _update(rt[pre], 1, n, k, v);
  }
  T query(const int &l, const int &r, const int &ql, const int &qr) {
    return _query(rt[l-1], rt[r], 1, n, ql, qr);
  }
private:
  int _build(const int &l, const int &r) {
    int cur = ++tot;
    sum[cur] = 0;
    if (l >= r) return cur;
    int mid = (l+r)>>1;
    ls[cur] = _build(l, mid);
    rs[cur] = _build(mid+1, r);
    return cur;
  }
  int _update(const int &pre, const int &l, const int &r, const int &k, const T &v) {
    int cur = ++tot;
    ls[cur] = ls[pre]; rs[cur] = rs[pre]; sum[cur] = sum[pre]+v;
    if (l >= r) return cur;
    int mid = (l+r)>>1;
    if (k <= mid) ls[cur] = _update(ls[pre], l, mid, k, v);
    else rs[cur] = _update(rs[pre], mid+1, r, k, v);
    return cur;
  }
  // [u, v] 段 [ql, qr] 区间和
  T _query(const int &u, const int &v, const int &l, const int &r, const int &ql, const int &qr) {
    if (l >= ql && r <= qr) return sum[v]-sum[u];
    int mid = (l+r)>>1;
    if (qr <= mid) return _query(ls[u], ls[v], l, mid, ql, qr);
    if (ql >  mid) return _query(rs[u], rs[v], mid+1, r, ql, qr);
    return _query(ls[u], ls[v], l, mid, ql, qr)+_query(rs[u], rs[v], mid+1, r, ql, qr);
  }
};
\end{lstlisting}

\section{ 分块}
例题
{\setmainfont{Consolas}
\begin{lstlisting}
struct FenKuai {
  typedef long long T;
  int t; // 每组大小
  static const int NN = static_cast<int>(sqrt(N))+7;
  T a[N], sum[NN], add[NN];
  FenKuai() {
    memset(a, 0, sizeof a);
    memset(sum, 0, sizeof sum);
    memset(add, 0, sizeof add);
  }
  void init() {
    t = static_cast<int>(sqrt(n)+0.5);
    for (int i = 0; i < n; ++i) sum[i/t] += a[i];
  }
  void update(int x, T k) { a[x] += k; sum[x/t] += k; }
  void update(int x, int y, T k) {
    for ( ; x <= y && x%t; ++x) a[x] += k, sum[x/t] += k;
    for ( ; x+t-1 <= y; x += t) sum[x/t] += k*t, add[x/t] += k;
    for ( ; x <= y; ++x) a[x] += k, sum[x/t] += k;
  }
  T query(int x) { return a[x]+add[x/t]; }
  T query(int x, int y) {
    T res = 0;
    for ( ; x <= y && x%t; ++x) res += a[x]+add[x/t];
    for ( ; x+t-1 <= y; x += t) res += sum[x/t];
    for ( ; x <= y; ++x) res += a[x]+add[x/t];
    return res;
  }
} B;
\end{lstlisting}
{\setmainfont{Consolas}
\begin{lstlisting}
struct FenKuai {
  typedef int T;
  int t; // 每组大小
  T a[N], b[N], add[N];
  FenKuai() {
    memset(a, 0, sizeof a);
    memset(b, 0, sizeof b);
    memset(add, 0, sizeof add);
  }
  void build(int x) {
    for (int i = x*t; i < min(x*t+t, n); ++i) b[i] = a[i];
    sort(b+x*t, b+min(x*t+t, n));
  }
  void init() {
    t = static_cast<int>(sqrt(n)+0.5);
    for (int i = 0; i*t < n; ++i) build(i);
  }
  void update(int x, int y, T c) {
    int i = x;
    for ( ; i <= y && i%t; ++i) a[i] += c;
    build(x/t);
    for ( ; i+t-1 <= y; i += t) add[i/t] += c;
    for ( ; i <= y; ++i) a[i] += c;
    build(y/t);
  }
  T query(int x, int y, long long c) {
    T res = 0; int i = x;
    for ( ; i <= y && i%t; ++i) res += (a[i]+add[i/t] < c*c);
    for ( ; i+t-1 <= y; i += t) res += lower_bound(b+i, b+i+t, c*c-add[i/t])-(b+i);
    for ( ; i <= y; ++i) res += (a[i]+add[i/t] < c*c);
    return res;
  }
} B;
\end{lstlisting}
{\setmainfont{Consolas}
\begin{lstlisting}
struct FenKuai {
  typedef int T;
  int t, sz;
  static const int NN = static_cast<int>(sqrt(N))+7;
  T a[N];
  deque<int> q[NN];
  void init(int _n) {
    sz = _n;
    t = static_cast<int>(sqrt(sz*1.5)+0.5);
    for (int i = 0; i < sz; ++i) q[i/t].push_back(a[i]);
  }
  void update(int x, int k) {
    stack<int> tmp;
    for (int i = 0; i != x%t; ++i) {
      tmp.push(q[x/t].front());
      q[x/t].pop_front();
    }
    q[x/t].push_front(k);
    while (tmp.size()) {
      q[x/t].push_front(tmp.top());
      tmp.pop();
    }
    ++sz;
    if (sz/t == x/t) return;
    for (int i = x/t, val; i < sz/t; ++i) {
      val = q[i].back();
      q[i].pop_back();
      q[i+1].push_front(val);
    }
  }
  T query(int x) {
    stack<int> tmp;
    for (int i = 0; i != x%t; ++i) {
      tmp.push(q[x/t].front());
      q[x/t].pop_front();
    }
    int res = q[x/t].front();
    while (tmp.size()) {
      q[x/t].push_front(tmp.top());
      tmp.pop();
    }
    return res;
  }
} B;
\end{lstlisting}
\section{ 莫队}
$O(1)修改$ 一般取 $block = \frac{n}{\sqrt{m} }, O(n\sqrt{m})$

移动前两步先扩大区间 $l--,r++$ 后两步缩小区间 $l++,r--$

\subsection{奇偶性排序}
{\setmainfont{Consolas}
\begin{lstlisting}
template <typename T> bool cmp(const T &q1, const T &q2) {
  return q1.l/block != q2.l/block ? q1.l < q2.l :
      (q1.l/block)&1 ? q1.r < q2.r : q1.r > q2.r;
}
\end{lstlisting}

\subsection{带修改莫队}

以 $n^{\frac{2}{3} }$ 为一块,分成了 $n^{\frac{1}{3} }$ 块,第一关键字是左端点所在块,第二关键字是右端点所在块,第三关键字是时间. 复杂度 $O(n^{\frac{5}{3} })$

{\setmainfont{Consolas}
\begin{lstlisting}
template <typename T> bool cmp(const T &q1, const T &q2) {
  return q1.l/block != q2.l/block ? q1.l < q2.l :
      q1.r/block != q2.r/block ? q1.r < q2.r : q1.t < q2.t;
}
\end{lstlisting}

\subsection{值域分块}

维护块的前缀和以及块内部前缀和, $O(\sqrt{n})$ 修改, $O(1)$ 求区间和

{\setmainfont{Consolas}
\begin{lstlisting}
template <typename T>
struct PreSum {
  int n, block;
  T s[N], t[(int)sqrt(N)+3];
  void init(int n) {
    this->n = n;
    block = sqrt(n);
  }
  void add(int x, T k) {
    for (int i = x; i/block == x/block && i <= n; ++i) s[i] += k;
    for (int i = x/block+1; i <= n/block; ++i) t[i] += k;
  }
  T query(int x) {
    return t[x/block]+s[x];
  }
};

template <typename T>
struct SufSum {
  int n, block;
  T s[N], t[(int)sqrt(N)+3];
  void init(int n) {
    this->n = n;
    block = sqrt(n);
  }
  void add(int x, T k) {
    for (int i = x; i/block == x/block && i >= 1; --i) s[i] += k;
    for (int i = x/block-1; i >= 0; --i) t[i] += k;
  }
  T query(int x) {
    return t[x/block]+s[x];
  }
};
\end{lstlisting}

\subsection{二次离线莫队}
大概是一种需要维护信息具有可减性的莫队。只要具可减性,就可以容斥,就可以二次离线。所谓『二次离线』,大概是指由于普通莫队无法快速计算贡献,所以第一次离线把询问离线下来,第二次离线把莫队的转移过程离线下来。

由于信息具有可减性(比如常见的「点对数」),记 $(a,b)(c,d)$ 表示区间 $[a,b]$ 内的点和区间 $[c,d]$ 内的点对彼此产生的贡献(区间内部不算)。

$[l,r]\to[l+t,r],\sum\limits_{i=l}^{l+t−1}(i,i)(i+1,r)=\sum\limits_{i=l}^{l+t−1}(i,i)(1,r)−(i,i)(1,i)$

$[l,r]\to[l-t,r],\sum\limits_{i=l-t}^{l-1}(i,i)(i+1,r)=\sum\limits_{i=l-t}^{l-1}(i,i)(1,r)−(i,i)(1,i)$

$[l,r]\to[l,r+t],\sum\limits_{i=r+1}^{r+t}(i,i)(l,i-1)=\sum\limits_{i=r+1}^{r+t}(1,i-1)(i,i)-(1,l-1)(i,i)$

$[l,r]\to[l,r-t],\sum\limits_{i=r-t+1}^{r}(i,i)(l,i-1)=\sum\limits_{i=r-t+1}^{r}(1,i-1)(i,i)-(1,l-1)(i,i)$


对于 $(1,i-1)(i,i)$ 没什么好说,暴力处理前缀和

对于 $(1,l-1)(i,i)$ 由于莫队的复杂度,至多有 $n\sqrt{m}$ 个不同询问,把每个询问 打标记到左端点(比如 $[l,r]\to [l,r-t]$ 就打到 $l-1$ 上), 最后扫一遍全部 $i \in [1,n]$ ,处理出询问值, 因为此时 $i$ 枚举 $O(n)$ 次,可以用『值域分块』技巧。这样最终复杂度 $O(n\sqrt n+n\sqrt{n})$

求区间逆序对

{\setmainfont{Consolas}
\begin{lstlisting}
struct Query {
  int id, l, r;
  Query() {}
  Query(int i, int _l, int _r) : id(i), l(_l), r(_r) {}
};

int n, m, block;
int a[N];
long long res[N], sumil[N], sumir[N], ans[N];
Query q[N];
SufSum<int> suml;
PreSum<int> sumr;
vector<Query> ql[N], qr[N];

inline void calc_sumi() {
  static BinaryIndexedTree<int> tree;
  tree.init(n);
  for (int i = 1; i <= n; ++i) {
    sumil[i] = sumil[i-1]+i-1-tree.query(a[i]);
    tree.add(a[i], 1);
  }
  tree.clear();
  for (int i = n; i; --i) {
    sumir[i] = sumir[i+1]+tree.query(a[i]-1);
    tree.add(a[i], 1);
  }
}

signed main() {
  read(n); read(m);
  for (int i = 1; i <= n; ++i) read(a[i]);
  discrete();
  block = n/sqrt(m);
  for (int i = 1; i <= m; ++i) {
    q[i].id = i;
    read(q[i].l);
    read(q[i].r);
  }
  sort(q+1, q+m+1, cmp);
  calc_sumi();
  q[0] = Query(0, 1, 0);
  for (int i = 1, ul, vl, ur, vr; i <= m; ++i) {
    ul = q[i-1].l; ur = q[i-1].r;
    vl = q[i].l; vr = q[i].r;
    res[i] = sumil[vr]-sumil[ur]+sumir[vl]-sumir[ul];
    if (vl < ul) qr[vr+1].emplace_back(-i, vl, ul-1);
    if (vl > ul) qr[vr+1].emplace_back(+i, ul, vl-1);
    if (vr < ur) ql[ul-1].emplace_back(+i, vr+1, ur);
    if (vr > ur) ql[ul-1].emplace_back(-i, ur+1, vr);
  }
  suml.init(n+1);
  for (int i = 1; i <= n; ++i) {
    suml.add(a[i], 1);
    for (auto &qq : ql[i]) {
      for (int j = qq.l; j <= qq.r; ++j) {
        if (qq.id > 0) res[qq.id] += suml.query(a[j]+1);
        else res[-qq.id] -= suml.query(a[j]+1);
      }
    }
  }
  sumr.init(n);
  for (int i = n; i; --i) {
    sumr.add(a[i], 1);
    for (auto &qq : qr[i]) {
      for (int j = qq.l; j <= qq.r; ++j) {
        if (qq.id > 0) res[qq.id] += sumr.query(a[j]-1);
        else res[-qq.id] -= sumr.query(a[j]-1);
      }
    }
  }
  for (int i = 1; i <= m; ++i) {
    res[i] += res[i-1];
    ans[q[i].id] = res[i];
  }
  for (int i = 1; i <= m; ++i) write(ans[i]), putchar('\n');
  return 0;
}
\end{lstlisting}

\section{ ST表}
\subsection{一维}
{\setmainfont{Consolas}
\begin{lstlisting}
template <typename T, typename U = std::greater<T>>
struct ST {
  static const int NN = 31-__builtin_clz(N)+3;
  static const T INF = 1e9;
  int lg2[N];
  U cmp = U();
  T rmq[N][NN];
  ST() {
    fill(rmq[0], rmq[0]+N*NN, cmp(-INF, +INF) ? INF : -INF);
    for (int i = 2; i < N; ++i) lg2[i] = lg2[i>>1]+1;
  }
  T& operator [] (const int &i) { return rmq[i][0]; }
  void init(const T &val = 0) { fill(rmq[0], rmq[0]+N*NN, val); }
  T mv(const T &x, const T &y) { return cmp(x, y) ? x : y; }
  // rmq[i][j] ==> [i, i+2^j-1]
  void build(T a[], const int &n) {
    for (int i = n; i; --i) {
      rmq[i][0] = a[i];
      for (int j = 1; j <= lg2[n-i+1]; ++j)
        rmq[i][j] =  mv(rmq[i][j-1], rmq[i+(1<<(j-1))][j-1]);
    }
  }
  T query(const int &l, const int &r) {
    if (l > r) return query(r, l);
    int k = lg2[r-l+1];
    return mv(rmq[l][k], rmq[r-(1<<k)+1][k]);
  }
};
\end{lstlisting}

\subsection{二维}
$O(nm\log n \log m)$

{\setmainfont{Consolas}
\begin{lstlisting}
template <typename T, typename U = std::greater<T>>
struct ST {
  static const int NN = (int)log2(N)+3;
  static const T INF = 1e9;
  U cmp = U();
  T rmq[N][N][NN][NN]; // rmq[i][j][k][l] [i, j] [i+2^k-1, j+2^l-1]
  ST() { init(); }
  ST(const T &val) { init(val); }
  T& operator [] (const int &i) { return rmq[i][0]; }
  void init(){ fill(rmq[0][0][0], rmq[0][0][0]+N*N*NN*NN, cmp(-INF, +INF) ? INF : -INF); }
  void init(const T &val) { fill(rmq[0][0][0], rmq[0][0][0]+N*N*NN*NN, val); }
  T mv(const T &x, const T &y) { return cmp(x, y) ? x : y; }
  void build(T a[N][N], const int &n, const int &m) {
    for (int k = 0; k <= log_2[n]; ++k)
    for (int l = 0; l <= log_2[m]; ++l)
    for (int i = 1; i+(1<<k)-1 <= n; ++i)
    for (int j = 1; j+(1<<l)-1 <= m; ++j) {
      T &cur = rmq[i][j][k][l];
      if (!k && !l) cur = a[i][j];
      else if (!l) cur = mv(rmq[i][j][k-1][l], rmq[i+(1<<(k-1))][j][k-1][l]);
      else cur = mv(rmq[i][j][k][l-1], rmq[i][j+(1<<(l-1))][k][l-1]);
    }
  }
  T query(const int &r1, const int &c1, const int &r2, const int &c2) {
    int k = log_2[r2-r1+1], l = log_2[c2-c1+1];
    return mv(mv(rmq[r1][c1][k][l], rmq[r2-(1<<k)+1][c2-(1<<l)+1][k][l]),
          mv(rmq[r2-(1<<k)+1][c1][k][l], rmq[r1][c2-(1<<l)+1][k][l]));
  }
};
\end{lstlisting}

\subsection{反向ST}
{\setmainfont{Consolas}
\begin{lstlisting}
template <typename T, typename U = std::greater<T>>
struct rST {
  static const int NN = (int)log2(N)+3;
  static const T INF = 1e9;
  int n;
  int lg2[N];
  U cmp = U();
  T rmq[N][NN]; // rmq[i][j] ==> [i, i+2^j-1]
  rST() { for (int i = 2; i < N; ++i) lg2[i] = lg2[i>>1]+1; }
  T& operator [] (const int &i) { return rmq[i][0]; }
  T mv(const T &x, const T &y) { return cmp(x, y) ? x : y; }
  void init(const int &_n, const T &val = 0) { 
    n = _n;
    for (int i = 1; i <= n; ++i) fill(rmq[i], rmq[i]+NN, val);
  }
  void update(const int &l, const int &r, const T &k) {
    if (l > r) return void(update(r, l, k));
    int b = lg2[r-l+1];
    rmq[l][b] = mv(rmq[l][b], k);
    rmq[r-(1<<b)+1][b] = mv(rmq[r-(1<<b)+1][b], k);
  }
  void build() {
    for (int i = lg2[n]; i >= 0; --i) {
      for (int l = 1, r; l <= n; ++l) {
        r = l+(1<<i);
        if (r <= n) rmq[r][i] = mv(rmq[r][i], rmq[l][i+1]);
        rmq[l][i] = mv(rmq[l][i], rmq[l][i+1]);
      }
    }
  }
  T query(const int &l, const int &r) {
    if (l > r) return query(r, l);
    int b = lg2[r-l+1];
    return mv(rmq[l][b], rmq[r-(1<<b)+1][b]);
  }
};
\end{lstlisting}


\section{ 并查集}

\subsection{路径压缩}

{\setmainfont{Consolas}
\begin{lstlisting}
struct DSU {
  vector<int> fa;
  void init(int n) { fa = vector<int>(n+1); iota(fa.begin(), fa.end(), 0); }
  int get(int s) { return s == fa[s] ? s : fa[s] = get(fa[s]); }
  int& operator [] (int i) { return fa[get(i)]; }
  bool merge(int x, int y) { // merge x to y
    x = get(x); y = get(y);
    return x == y ? false : (fa[x] = y, true);
  }
};
\end{lstlisting}

\subsection{按秩合并}

秩的意思就是树的高度,按秩合并过后并查集的结构为树形结构,最坏情况为 $O(m \log n)$

{\setmainfont{Consolas}
\begin{lstlisting}
struct DSU {
  vector<int> fa, rk;
  void init(int n) { fa = rk = vector<int>(n+1, 0); iota(fa.begin(), fa.end(), 0); }
  int get(int s) { return s == fa[s] ? s : get(fa[s]); }
  int operator ; }
  bool merge(int x, int y) {
    x = get(x); y = get(y);
    if (x == y) return false;
    if (rk[x] < rk[y]) fa[x] = y;
    else fa[y] = x, rk[x] += rk[x] == rk[y];
    return true;
  }
};
\end{lstlisting}

\subsection{带权并查集}

{\setmainfont{Consolas}
\begin{lstlisting}
template <typename T = int> struct DSU {
  vector<int> fa;
  vector<T> w;
  void init(int n, T v = 1) {
    fa = vector<int>(n+1);
    iota(fa.begin(), fa.end(), 0);
    w = vector<T>(n+1, v);
  }
  void init(int n, T a[]) {
    fa = vector<int>(n+1);
    iota(fa.begin(), fa.end(), 0);
    w = vector<T>(a, a+n+1);
  }
  int get(int s) { return s == fa[s] ? s : fa[s] = get(fa[s]); }
  int& operator [] (int i) { return fa[get(i)]; }
  bool merge(int x, int y) { // merge x to y
    x = get(x); y = get(y);
    return x == y ? false : (w[y] += w[x], fa[x] = y, true);
  }
};
\end{lstlisting}

\subsection{扩展域并查集}

例题:关押罪犯,食物链

{\setmainfont{Consolas}
\begin{lstlisting}
struct DSU {
  int n;
  vector<int> fa; // [1, n] partner, [n+1, 2n] enemy
  void init(int n) {
    this->n = n;
    fa = vector<int>(2*n+1, 0);
    iota(fa.begin(), fa.end(), 0);
  }
  int get(int s) { return s == fa[s] ? s : fa[s] = get(fa[s]); }
  int& operator [] (int i) { return fa[get(i)]; }
  void merge(int x, int y) { fa[get(x)] = get(y); }
  bool update(int x, int y) {
    if (get(x) == get(y)) return false;
    merge(x+n, y);
    merge(x, y+n);
    return true;
  }
};
\end{lstlisting}

\subsection{可撤销并查集}

用一个栈维护每次操作

\subsubsection{按秩合并}

{\setmainfont{Consolas}
\begin{lstlisting}
struct DSU {
  vector<int> fa, rk;
  stack<pair<int&, int>> stk;
  void init(int n) {
    stk = stack<pair<int&, int>>();
    fa = rk = vector<int>(n+1, 0);
    iota(fa.begin(), fa.end(), 0);
  }
  int get(int s) { while (s != fa[s]) s = fa[s]; return s; }
  int& operator ]; }
  int merge(int x, int y) { // return the number push in stack
    x = get(x); y = get(y);
    if (x == y) return 0;
    if (rk[x] > rk[y]) swap(x, y);
    stk.push({fa[x], fa[x]});
    fa[x] = y;
    return rk[x] == rk[y] ? stk.push({rk[y], rk[y]++}), 2 : 1;
  }
  bool undo() {
    if (stk.empty()) return false;
    stk.top().first = stk.top().second;
    stk.pop();
    return true;
  }
};
\end{lstlisting}

\subsubsection{启发式合并}
{\setmainfont{Consolas}
\begin{lstlisting}
struct DSU {
  vector<int> fa, sz;
  stack<pair<int, int>> stk;
  void init(int n) {
    stk = stack<pair<int, int>>();
    fa = sz = vector<int>(n+1, 1);
    iota(fa.begin(), fa.end(), 0);
  }
  int get(int s) { while (s != fa[s]) s = fa[s]; return s; }
  int& operator ]; }
  int merge(int x, int y) {
    x = get(x); y = get(y);
    if (x == y) return 0;
    if (sz[x] > sz[y]) swap(x, y);
    stk.push({x, y});
    fa[x] = y;
    sz[y] += sz[x];
    return 1;
  }
  bool undo() {
    if (stk.empty()) return false;
    int x = stk.top().first, y = stk.top().y;
    stk.pop();
    fa[x] = x;
    sz[y] -= sz[x];
    return true;
  }
}
\end{lstlisting}

\subsubsection{可撤销扩展域并查集}

{\setmainfont{Consolas}
\begin{lstlisting}
struct DSU {
  int n;
  vector<int> fa, rk; // [1, n] partner, [n+1, 2n] enemy
  stack<pair<int&, int>> stk;
  void init(int n) {
    this->n = n;
    stk = stack<pair<int&, int>>();
    fa = rk = vector<int>(2*n+1, 0);
    iota(fa.begin(), fa.end(), 0);
  }
  int& operator [] (int i) { return fa[get(i)]; }
  int get(int s) { while (s != fa[s]) s = fa[s]; return s; }
  void undo() { stk.top().first = stk.top().second; stk.pop(); }
  void merge(int x, int y) {
    x = get(x); y = get(y);
    if (x == y) return;
    if (rk[x] > rk[y]) swap(x, y);
    stk.push({fa[x], fa[x]});
    stk.push({rk[y], rk[y]});
    fa[x] = y;
    rk[y] += rk[x] == rk[y];
  }
  bool update(int x, int y) {
    if (get(x) == get(y)) return false;
    merge(x+n, y);
    merge(x, y+n);
    return true;
  }
};
\end{lstlisting}

\subsection{可持久化并查集}

{\setmainfont{Consolas}
\begin{lstlisting}
struct PersistantUnionSet {
  static const int NN = N*(log2(N)+3);
  int rt[N], ls[NN], rs[NN], fa[NN], dep[NN], n, tot;
  void build(const int &n) {
    this->n = n;
    tot = 0;
    rt[0] = build(1, n);
  }
  int build(const int &l, const int &r) {
    int cur = ++tot; assert(tot < NN);
    if (l == r) return fa[cur] = l, dep[cur] = 0, cur;
    int mid = (l+r)>>1;
    ls[cur] = build(l, mid);
    rs[cur] = build(mid+1, r);
    return cur;
  }
  bool query(const int &cur, const int &x, const int &y) {
    return fa[getf(rt[cur], x)] == fa[getf(rt[cur], y)];
  }
  // return the id of fa[], dep[]
  int query(const int &cur, const int &x, const int &l, const int &r) {
    if (l == r) return cur;
    int mid = (l+r)>>1;
    if (x <= mid) return query(ls[cur], x, l, mid);
    else return query(rs[cur], x, mid+1, r);
  }
  // return the id of fa[], dep[]
  int getf(const int &cur, int x) {
    int fi;
    while (fa[(fi = query(cur, x, 1, n))] != x) x = fa[fi];
    return fi;
  }
  void merge(const int &cur, const int &pre, const int &x, const int &y) {
    rt[cur] = rt[pre];
    int fx = getf(rt[cur], x), fy = getf(rt[cur], y);
    if (fa[fx] == fa[fy]) return;
    if (dep[fx] > dep[fy]) swap(fx, fy);
    rt[cur] = update(rt[pre], fa[fx], fa[fy], 1, n);
    if (dep[fx] == dep[fy]) add(rt[cur], fa[fy], 1, n);
  }
  // update fa, merge x to y
  int update(const int &pre, const int &x, const int &y, const int &l, const int &r) {
    int cur = ++tot; assert(tot < NN);
    if (l == r) return fa[cur] = y, dep[cur] = dep[pre], cur;
    ls[cur] = ls[pre]; rs[cur] = rs[pre];
    int mid = (l+r)>>1;
    if (x <= mid) ls[cur] = update(ls[pre], x, y, l, mid);
    else rs[cur] = update(rs[pre], x, y, mid+1, r);
    return cur;
  }
  // add dep
  void add(const int &cur, const int &x, const int &l, const int &r) {
    if (l == r) return ++dep[cur], void();
    int mid = (l+r)>>1;
    if (x <= mid) add(ls[cur], x, l, mid);
    else add(rs[cur], x, mid+1, r);
  }
};
\end{lstlisting}

\section{ 左偏树|可并堆}

1 x y:将第 x 个数和第 y 个数所在的小根堆合并(若第 x 或第 y 个数已经被删除或第 x 和第 y 个数在用一个堆内,则无视此操作)。

2 x:输出第 x 个数所在的堆最小数,并将这个最小数删除(若有多个最小数,优先删除先输入的;若第 x 个数已经被删除,则输出 -1 并无视删除操作)。

{\setmainfont{Consolas}
\begin{lstlisting}
template <typename T> struct Tree { // 左偏树|可并堆
#define ls tr[x].son[0]
#define rs tr[x].son[1]
  struct TreeNode {
    T val;
    int dis, rt, son[2];
  };
  vector<TreeNode> tr;
  template <typename TT> void build(TT a[], int n) {
    tr.resize(n+1);
    tr[0].dis = -1;
    for (int i = 1; i <= n; ++i) {
      tr[i].val = a[i];
      tr[i].rt = i;
    }
  }
  int get(int x) {
    return tr[x].rt == x ? x : tr[x].rt = get(tr[x].rt);
  }
  void merge(int x, int y) {
    if (tr[x].val == -1 || tr[y].val == -1) return;
    x = get(x); y = get(y);
    if (x != y) tr[x].rt = tr[y].rt = _merge(x, y);
  }
  int _merge(int x, int y) {
    if (!x || !y) return x+y;
    if (tr[x].val > tr[y].val || (tr[x].val == tr[y].val && x > y)) swap(x, y);
    rs = _merge(rs, y);
    if(tr[ls].dis < tr[rs].dis) swap(ls, rs);
    tr[ls].rt = tr[rs].rt = tr[x].rt = x;
    tr[x].dis = tr[rs].dis+1;
    return x;
  }
  T pop(int x) {
    if (tr[x].val == -1) return -1;
    x = get(x);
    T res = tr[x].val;
    tr[x].val = -1;
    tr[ls].rt = ls;
    tr[rs].rt = rs;
    tr[x].rt = _merge(ls, rs);
    return res;
  }
#undef ls
#undef rs
};
\end{lstlisting}


\part{字符串}
\section{ 回文字符串|manacher算法}
从 0 开始,第 i 位对应 p[i*2+2]
{\setmainfont{Consolas}
\begin{lstlisting}
inline int manacher(const char *str, char *buf, int *p) {
  int str_len = strlen(str), buf_len = 2;
  buf[0] = buf[1] = '#';
  for(int i = 0; i < str_len; ++i)
    buf[buf_len++] = str[i], buf[buf_len++] = '#';

  int mx = 0, id, ans = 0;
  for(int i = 1; i < buf_len; ++i) {
    if(i <= mx) p[i] = min(p[id*2-i], mx-i);
    else p[i] = 1;
    while(buf[i-p[i]] == buf[i+p[i]]) p[i]++;
    if(i+p[i] > mx) mx = i+p[i], id = i;
    ans = max(ans, p[i]-1);
  }
  return ans;
}
\end{lstlisting}
\subsection{判断s[l, r]是否为回文}
{\setmainfont{Consolas}
\begin{lstlisting}
p[l+r+2]-1 >= r-l+1
\end{lstlisting}


\section{ KMP}
{\setmainfont{Consolas}
\begin{lstlisting}
inline void get_next(const char *s, int nex[]) {
  nex[0] = nex[1] = 0;
  for (int i = 1, j = 0, l = strlen(s); i < l; ++i) {
    while (j && s[i] != s[j]) j = nex[j];
    nex[i+1] = s[i] == s[j] ? ++j : 0;
  }
}

inline void kmp(const char *s1, const char *s2, int nex[]) {
  for (int i = 0, j = 0, l1 = strlen(s1), l2 = strlen(s2); i < l1; ++i){
    while (j && s1[i] != s2[j]) j = nex[j];
    if (s1[i] == s2[j]) ++j;
    if (j == l2) {
      cout << i-l2+2 << endl;
      j = nex[j];
    }
  }
}
\end{lstlisting}

nex 数组往上跳为公差为 i-nex[i] 的等差数列直到 i/2

\section{ 扩展KMP|Z函数}
{\setmainfont{Consolas}
\begin{lstlisting}
vector<int> z_function(string s) {
  int n = (int)s.length();
  vector<int> z(n);
  for (int i = 1, l = 0, r = 0; i < n; ++i) {
    if (i <= r && z[i - l] < r - i + 1) {
      z[i] = z[i - l];
    } else {
      z[i] = max(0, r - i + 1);
      while (i + z[i] < n && s[z[i]] == s[i + z[i]]) ++z[i];
    }
    if (i + z[i] - 1 > r) l = i, r = i + z[i] - 1;
  }
  return z;
}
\end{lstlisting}

\section{ 字典树}
{\setmainfont{Consolas}
\begin{lstlisting}
struct TireTree {
  static const int NN = 5e5+7;
  static const int SZ = 26;
  char beg;
  int nex[NN][SZ], num[NN], cnt;
  bool exist[NN];
  TireTree(char _beg = 'a') : beg(_beg) { clear(); }
  void clear() {
    memset(nex, 0, sizeof(nex[0])*(cnt+1));
    memset(num, 0, sizeof(int)*(cnt+1));
    memset(exist, 0, sizeof(bool)*(cnt+1));
    cnt = 0;
  }
  void insert(const char *s) {
    int len = strlen(s), p = 0;
    for (int i = 0, c; i < len; ++i) {
      c = s[i]-beg;
      if (!nex[p][c]) nex[p][c] = ++cnt;
      p = nex[p][c];
      ++num[p];
    }
    exist[p] = true;
  }
  bool find(const char *s) {
    int len = strlen(s), p = 0;
    for (int i = 0, c; i < len; ++i) {
      c = s[i]-beg;
      if (!nex[p][c]) return false;
      p = nex[p][c];
    }
    return exist[p];
  }
  int count(const char *s) {
    int len = strlen(s), p = 0;
    for (int i = 0, c; i < len; ++i) {
      c = s[i]-beg;
      if (!nex[p][c]) return 0;
      p = nex[p][c];
    }
    return num[p];
  }
  void insert(const string &s) { insert(s.c_str()); }
  bool find(const string &s) { return find(s.c_str()); }
  int count(const string &s) { return count(s.c_str()); }
};
\end{lstlisting}

\subsection{求异或}
最大异或
{\setmainfont{Consolas}
\begin{lstlisting}
struct TireTree {
  static const int SZ = 2;
  static const int B = 30;
  static const int NN = N*B;
  int nex[NN][SZ], cnt;
  void init() { clear(); }
  void clear() {
    memset(nex, 0, sizeof(int)*(cnt+1)*SZ);
    cnt = 0;
  }
  void insert(int x) {
    for (int i = B, c, p = 0; i >= 0; --i) {
      c = (x>>i)&1;
      if (!nex[p][c]) nex[p][c] = ++cnt;
      p = nex[p][c];
    }
  }
  int max_xor(int x) const {
    int ans = 0;
    for (int i = B, c, p = 0; i >= 0; --i) {
      c = (x>>i)&1;
      if (!nex[p][c^1]) p = nex[p][c];
      else p = nex[p][c^1], ans |= 1<<i;
    }
    return ans;
  }
};
\end{lstlisting}
动态开点+支持合并
{\setmainfont{Consolas}
\begin{lstlisting}
struct TireTree {
  static const int SZ = 2;
  static const int B = 30;
  typedef array<int, SZ> T;
  vector<T> nex;
  TireTree() { init(); }
  void init() { nex.assign(1, T()); /* nex.reserve(N); */ }
  void clear() { nex = vector<T>(); }
  size_t size() const { return nex.size(); }
  void extend(int &x) {
    if (x != 0) return;
    x = nex.size();
    nex.emplace_back(T());
  }
  void insert(int x) {
    for (int i = B, c, p = 0; i >= 0; --i) {
      c = (x>>i)&1;
      extend(nex[p][c]);
      p = nex[p][c];
    }
  }
  int max_xor(int x) const {
    int ans = 0;
    for (int i = B, c, p = 0; i >= 0; --i) {
      c = (x>>i)&1;
      if (!nex[p][c^1]) p = nex[p][c];
      else p = nex[p][c^1], ans |= 1<<i;
    }
    return ans;
  }
  void dfs(const TireTree &t, int p = 0, int pt = 0) {
    for (int c = 0; c < SZ; ++c) {
      if (t.nex[pt][c] == 0) continue;
      extend(nex[p][c]);
      dfs(t, nex[p][c], t.nex[pt][c]);
    }
  }
  void join(TireTree &t) {
    if (t.size() < size()) swap(*this, t);
    dfs(t);
    t.clear();
  }
};
\end{lstlisting}

\section{ AC自动机}
AC 自动机是 以 Trie 的结构为基础,结合 KMP 的思想 建立的。

将所有模式串构成一棵 Trie, 再对所有结点构造失配指针

如需构造可重建AC自动机,每次构造建一个nex数组的拷贝
{\setmainfont{Consolas}
\begin{lstlisting}
struct Aho_Corasick_Automaton {
  static const int NN = 5e6+7;
  static const int SZ = 26;
  char beg;
  int nex[NN][SZ], num[NN], fail[NN], cnt;
  Aho_Corasick_Automaton(const char &_beg = 'a') : beg(_beg) {}
  void clear() {
    memset(nex, 0, sizeof(nex[0])*(cnt+1));
    memset(num, 0, sizeof(int)*(cnt+1));
    memset(fail, 0, sizeof(int)*(cnt+1));
    cnt = 0;
  }
  void insert(const char *s) {
    int len = strlen(s), p = 0;
    for (int i = 0, c; i < len; ++i) {
      c = s[i]-beg;
      if (!nex[p][c]) nex[p][c] = ++cnt;
      p = nex[p][c];
    }
    ++num[p];
  }
  void build() {
    static queue<int> q;
    for (int i = 0; i < SZ; ++i) if (nex[0][i]) q.push(nex[0][i]);
    while (q.size()) {
      int u = q.front();
      q.pop();
      for (int i = 0; i < SZ; ++i) {
        if (nex[u][i]) {
          fail[nex[u][i]] = nex[fail[u]][i];
          q.push(nex[u][i]);
        } else {
          nex[u][i] = nex[fail[u]][i];
        }
      }
    }
  }
  int query(const char *s) {
    int len = strlen(s), p = 0, res = 0;
    for (int i = 0; i < len; ++i) {
      p = nex[p][s[i]-beg];
      for (int t = p; t && ~num[t]; t = fail[t]) {
        res += num[t];
        num[t] = -1;
      }
    }
    return res;
  }
};
\end{lstlisting}

{\setmainfont{Consolas}
\begin{lstlisting}
struct Aho_Corasick_Automaton {
  static const int NN = 2e5+7;
  static const int SZ = 26;
  char beg;
  int cnt;
  int nex[NN][SZ], fail[NN], vis[NN];
  Aho_Corasick_Automaton(const char &_beg = 'a') : beg(_beg) {}
  void clear() {
    memset(nex, 0, sizeof(nex[0])*(cnt+1));
    memset(fail, 0, sizeof(int)*(cnt+1));
    memset(vis, 0, sizeof(int)*(cnt+1));
    cnt = 0;
  }
  int insert(const char *s) {
    int len = strlen(s), p = 0;
    for (int i = 0, c; i < len; ++i) {
      c = s[i]-beg;
      if (!nex[p][c]) nex[p][c] = ++cnt;
      p = nex[p][c];
    }
    return p;
  }
  void build() {
    static queue<int> q;
    for (int i = 0; i < SZ; ++i) if (nex[0][i]) q.push(nex[0][i]);
    while (q.size()) {
      int u = q.front();
      q.pop();
      for (int i = 0; i < SZ; ++i) {
        if (nex[u][i]) {
          fail[nex[u][i]] = nex[fail[u]][i];
          q.push(nex[u][i]);
        } else {
          nex[u][i] = nex[fail[u]][i];
        }
      }
    }
  }
  void query(char *s) {
    static int deg[NN];
    static queue<int> q;
    int len = strlen(s);
    for (int i = 0, p = 0; i < len; ++i) {
      p = nex[p][s[i]-beg];
      ++vis[p];
      // for (int t = p; t; t = fail[t]) ++vis[t];
    }
    for (int i = 1; i <= cnt; ++i) ++deg[fail[i]];
    for (int i = 1; i <= cnt; ++i) if (!deg[i]) q.push(i);
    while (q.size()) {
      int u = q.front();
      q.pop();
      vis[fail[u]] += vis[u];
      if (--deg[fail[u]] == 0) q.push(fail[u]);
    }
  }
} ac;
\end{lstlisting}


\section{ 后缀数组|SA}

$sa[i]$ 表示将所有后缀排序后第 $i$ 小的后缀的编号

$rk[i]$ 表示后缀 $i$ 的排名

性质:$sa[rk[i]]=rk[sa[i]]=i$

$lcp(i, j)$ 表示后缀 $i$ 和后缀 $j$ 的最长公共前缀(的长度)

$height[i]=lcp(sa[i], sa[i-1])$

引理 $height[rk[i]] \geq height[rk[i-1]]-1$

$lcp(sa[i],sa[j])=\min\{height[i+1\cdots j]\}$

不同子串数目:$\frac{n(n+1)}{2}-\sum\limits_{i=2}^{n}{height[i]}$

\subsection{$O(nlogn)$}
{\setmainfont{Consolas}
\begin{lstlisting}
int sa[N], rk[N<<1], height[N];
template <typename T> // s start from 1
inline void SA(const T *s, const int &n) {
#define cmp(x, y, w) oldrk[x] == oldrk[y] && oldrk[x + w] == oldrk[y + w]
  static int oldrk[N<<1], id[N], px[N], cnt[N], m;
  // memset(oldrk+n+1, 0, sizeof(int)*n); // multi testcase
  memset(cnt, 0, sizeof(int) * (m = 128));
  for (int i = 1; i <= n; ++i) ++cnt[rk[i] = s[i]];
  for (int i = 1; i <= m; ++i) cnt[i] += cnt[i - 1];
  for (int i = n; i; --i) sa[cnt[rk[i]]--] = i;
  for (int w = 1, p, i; w <= n; w <<= 1, m = p) {
    for (p = 0, i = n; i > n - w; --i) id[++p] = i;
    for (i = 1; i <= n; ++i) if (sa[i] > w) id[++p] = sa[i] - w;
    memset(cnt + 1, 0, sizeof(int) * m);
    for (i = 1; i <= n; ++i) ++cnt[px[i] = rk[id[i]]];
    for (i = 1; i <= m; ++i) cnt[i] += cnt[i - 1];
    for (i = n; i; --i) sa[cnt[px[i]]--] = id[i];
    swap(oldrk, rk); // memcpy(oldrk+1, rk+1, sizeof(int)*n);
    for (p = 0, i = 1; i <= n; ++i) rk[sa[i]] = cmp(sa[i], sa[i - 1], w) ? p : ++p;
  }
  for (int i = 1, k = 0; i <= n; ++i) {
    if (k) --k;
    while (s[i+k] == s[sa[rk[i]-1]+k]) ++k;
    height[rk[i]] = k;
  }
#undef cmp
}
\end{lstlisting}

\subsection{$O(n)$}
{\setmainfont{Consolas}
\begin{lstlisting}
namespace SuffixArray {

int sa[N], rk[N], ht[N];
bool t[N << 1];

inline bool islms(const int i, const bool *t) { return i > 0 && t[i] && !t[i - 1]; }

template <class T>
inline void sort(T s, int *sa, const int len, const int sz, const int sigma, bool *t, int *b, int *cb, int *p) {
  memset(b, 0, sizeof(int) * sigma);
  memset(sa, -1, sizeof(int) * len);
  for (register int i = 0; i < len; i++) b[static_cast<int>(s[i])]++;
  cb[0] = b[0];
  for (register int i = 1; i < sigma; i++) cb[i] = cb[i - 1] + b[i];
  for (register int i = sz - 1; i >= 0; i--) sa[--cb[static_cast<int>(s[p[i]])]] = p[i];
  for (register int i = 1; i < sigma; i++) cb[i] = cb[i - 1] + b[i - 1];
  for (register int i = 0; i < len; i++)
    if (sa[i] > 0 && !t[sa[i] - 1])
      sa[cb[static_cast<int>(s[sa[i] - 1])]++] = sa[i] - 1;
  cb[0] = b[0];
  for (register int i = 1; i < sigma; i++) cb[i] = cb[i - 1] + b[i];
  for (register int i = len - 1; i >= 0; i--)
    if (sa[i] > 0 && t[sa[i] - 1])
      sa[--cb[static_cast<int>(s[sa[i] - 1])]] = sa[i] - 1;
}

template <class T>
inline void sais(T s, int *sa, const int len, bool *t, int *b, int *b1, const int sigma) {
  register int i, j, x, p = -1, cnt = 0, sz = 0, *cb = b + sigma;
  for (t[len - 1] = 1, i = len - 2; i >= 0; i--) t[i] = s[i] < s[i + 1] || (s[i] == s[i + 1] && t[i + 1]);
  for (i = 1; i < len; i++)
    if (t[i] && !t[i - 1])
      b1[sz++] = i;
  sort(s, sa, len, sz, sigma, t, b, cb, b1);
  for (i = sz = 0; i < len; i++)
    if (islms(sa[i], t))
      sa[sz++] = sa[i];
  for (i = sz; i < len; i++) sa[i] = -1;
  for (i = 0; i < sz; i++) {
    for (x = sa[i], j = 0; j < len; j++) {
      if (p == -1 || s[x + j] != s[p + j] || t[x + j] != t[p + j]) {
        cnt++, p = x;
        break;
      } else if (j > 0 && (islms(x + j, t) || islms(p + j, t))) {
        break;
      }
    }
    sa[sz + (x >>= 1)] = cnt - 1;
  }
  for (i = j = len - 1; i >= sz; i--)
    if (sa[i] >= 0)
      sa[j--] = sa[i];
  register int *s1 = sa + len - sz, *b2 = b1 + sz;
  if (cnt < sz)
    sais(s1, sa, sz, t + len, b, b1 + sz, cnt);
  else
    for (i = 0; i < sz; i++) sa[s1[i]] = i;
  for (i = 0; i < sz; i++) b2[i] = b1[sa[i]];
  sort(s, sa, len, sz, sigma, t, b, cb, b2);
}

template <class T>
inline void getHeight(T s, int n) {
  for (register int i = 1; i <= n; i++) rk[sa[i]] = i;
  register int j = 0, k = 0;
  for (register int i = 0; i < n; ht[rk[i++]] = k)
    for (k ? k-- : 0, j = sa[rk[i] - 1]; s[i + k] == s[j + k]; k++)
      ;
}

template <class T>  // s start from 0
inline void init(T s, const int len, const int sigma = 128) {
  sais(s, sa, len + 1, t, rk, ht, sigma);
  getHeight(s, len);
  for (int i = 1; i <= len; ++i) ++sa[i];
  for (int i = len; i; --i) rk[i] = rk[i-1];
}

}  // namespace SuffixArray
\end{lstlisting}

\subsection{树上SA}

树上可能出现完全相同的字符串,增加上一轮的有序状态rk为"第三关键字"

{\setmainfont{Consolas}
\begin{lstlisting}
struct SAonTree {
  int n, d[N], cnt[N], sa[N], rk[N<<1], _rk[N<<1], _oldrk[N<<1], tp[N<<1];
  template <typename T>
  void tsort(int *sa, T *rk, int *tp, int m) {
    memset(cnt, 0, sizeof(int)*(m+1));
    for (int i = 1; i <= n; ++i) ++cnt[rk[i]];
    for (int i = 1; i <= m; ++i) cnt[i] += cnt[i-1];
    for (int i = n; i; --i) sa[cnt[rk[tp[i]]]--] = tp[i];
  }
  template <typename T>
  void build(int *f, const T *a, const int n) {
    this->n = n;
    int p = 128, i; // p = n
    iota(tp+1, tp+n+1, 1);
    tsort(sa, a, tp, p);
    for (i = 1, p = 0; i <= n; ++i) {
      _rk[sa[i]] = a[sa[i-1]] == a[sa[i]] ? p : ++p;
      rk[sa[i]] = i;
    }
    for (int w = 1, t = 0; w < n; w <<= 1, ++t) {
      for (i = 1; i <= n; ++i) _oldrk[i] = rk[f[i]];
      tsort(tp, _oldrk, sa, n);
      tsort(sa, _rk, tp, p);
      swap(_rk, tp);
      for (i = 1, p = 0; i <= n; ++i) {
        _rk[sa[i]] = tp[sa[i-1]] == tp[sa[i]]
            && tp[f[sa[i-1]]] == tp[f[sa[i]]] ? p : ++p;
        rk[sa[i]] = i;
      }
      for (int i = n; i; --i) f[i] = f[f[i]]; // attention special tree
    }
  }
};
\end{lstlisting}

\section{ 后缀平衡树}

\section{ 后缀自动机|SAM}

一个状态表示一个 $endpos$ 的等价类

$len(v)$ 为该状态最长的字符串长度

后缀链接 $link(v)$ 连接到对应于该状态最长字符串的最长后缀的另一个 $endpos$ 等价类的状态。

\subsection{代码}
{\setmainfont{Consolas}
\begin{lstlisting}
struct SAM { // root 0
  static const int A = 26;
  static const int M = N<<1;
  static const char C = 'a';
  int sz, last, len[M], link[M], nex[M][A];
  void init() {
    memset(nex, 0, sizeof(int)*A*sz);
    link[0] = -1; sz = 1; last = 0;
  }
  int extend(int c) {
    int cur = sz++, p = last;
    len[cur] = len[last]+1;
    for (; ~p && !nex[p][c]; p = link[p]) nex[p][c] = cur;
    if (p == -1) return link[cur] = 0, cur;
    int q = nex[p][c];
    if (len[p]+1 == len[q]) return link[cur] = q, cur;
    int clone = sz++;
    memcpy(nex[clone], nex[q], sizeof nex[q]);
    link[clone] = link[q];
    len[clone] = len[p]+1;
    for (; ~p && nex[p][c] == q; p = link[p]) nex[p][c] = clone;
    link[q] = link[cur] = clone;
    return cur;
  }
  void insert(const string &s) {
    init(); for (char ch : s) last = extend(ch-C);
  }
};
\end{lstlisting}

{\setmainfont{Consolas}
\begin{lstlisting}
  void build() { // topo on parent tree
    static int t[M], rk[M];
    memset(t, 0, sizeof(int)*sz);
    for (int i = 0; i < sz; ++i) ++t[len[i]];
    for (int i = 1; i < sz; ++i) t[i] += t[i-1];
    for (int i = 0; i < sz; ++i) rk[--t[len[i]]] = i;
    for (int _ = sz-1, i, j; _; --_) { // assert(rk[0] == 0);
      i = rk[_];
      j = link[i];
      cnt[j] += cnt[i];
    }
  }
\end{lstlisting}
\subsection{检查字符串是否出现}

丢进去转移。这个算法还找到了模式串在文本串中出现的最大前缀长度。

\subsection{不同子串个数}
不同子串的个数等于自动机中以 $t_0$ 为起点的不同路径的条数-1(空串)。令 $d_{v}$ 为从状态 $v$ 开始的路径数量(包括长度为零的路径)

$d_{v}=1+\sum\limits_{w:(v,w,c)\in DAWG}d_{w}$

或利用上述后缀自动机树形结构统计节点对应的子串数量 $len(i)-len(link(i))$

ps:若新增一个字符,其增量为$len(cur)-len(link(cur))$ 不包括 $clone$ 结点

\subsection{所有不同子串的总长度}

$ans_{v}=\sum\limits_{w:(v,w,c)\in DAWG}d_{w}+ans_{w}$

法二:每个节点对应的所有后缀长度是 $\frac{len(i)\times (len(i)+1)}{2}$,减去其 $link$ 节点的对应值就是该节点的净贡献

\subsection{字典序第 k 大子串}

不同位置的相同子串算作一个,每个非 clone 状态的数量记为1

不同位置的相同子串算作多个,每个状态的数量为 parent 树上求子树和

在 SAM 的 DAG 求和然后字典序搞搞

\subsection{两个字符串的最长公共子串}

{\setmainfont{Consolas}
\begin{lstlisting}
  int lcs(const string &s) {
    int u = 0, l = 0, res = 0, c;
    for (char ch : s) {
      c = ch-C;
      while (u && !nex[u][c]) l = len[u = link[u]];
      if (nex[u][c]) u = nex[u][c], ++l;
      res = max(res, l);
    }
    return res;
  }
\end{lstlisting}

\subsection{多个字符串间的最长公共子串}

记录 $f[i][j]$ 为第 $i$ 个字符串在 sam 上状态 $j$ 的匹配长度

$ans = \max\limits_{j}{(\min\limits_{i}{f[i][j]})}$

\subsection{后缀的最长公共前缀|height}
求两个后缀的最长公共前缀,显然就是两个后缀的节点在Parent树上的LCA

\section{ 广义后缀自动机}

广义后缀自动机 (General Suffix Automaton) 是将后缀自动机整合到字典树中来解决对于多个字符串的子串问题

\subsection{离线构造}
{\setmainfont{Consolas}
\begin{lstlisting}
struct generalSAM {
  static const int A = 26;
  static const int M = N<<1;
  static const char C = 'a';
  int sz, len[M], link[M], nex[M][A];
  generalSAM() { init(); }
  void init() {
    // memset(nex, 0, sizeof(int)*A*sz);
    link[0] = -1; sz = 1;
  }
  int insertSAM(int last, int c) {
    int cur = nex[last][c];
    if (len[cur]) return cur;
    len[cur] = len[last]+1;
    int p = link[last];
    for (; ~p && !nex[p][c]; p = link[p]) nex[p][c] = cur;
    if (p == -1) return link[cur] = 0, cur;
    int q = nex[p][c];
    if (len[p]+1 == len[q]) return link[cur] = q, cur;
    int clone = sz++;
    for (int i = 0; i < A; ++i)
      nex[clone][i] = len[nex[q][i]] ? nex[q][i] : 0;
    len[clone] = len[p]+1;
    for (; ~p && nex[p][c] == q; p = link[p]) nex[p][c] = clone;
    link[clone] = link[q];
    link[q] = link[cur] = clone;
    return cur;
  }
  int insertTrie(int cur, int c) {
    return nex[cur][c] ? nex[cur][c] : nex[cur][c] = sz++;
  }
  void insert(const string &s) {
    int last = 0; for (char ch : s) last = insertTrie(last, ch-C);
  }
  void insert(const char *s, int n) {
    for (int i = 0, last = 0; i < n; ++i) last = insertTrie(last, s[i]-C);
  }
  void build() {
    queue<pair<int, int>> q;
    for (int i = 0; i < A; ++i)
      if (nex[0][i]) q.push({0, i});
    while (!q.empty()) {
      auto item = q.front(); q.pop();
      int last = insertSAM(item.first, item.second);
      for (int i = 0; i < A; ++i)
        if (nex[last][i]) q.push({last, i});
    }
  }
};
\end{lstlisting}
\subsection{在线构造}

{\setmainfont{Consolas}
\begin{lstlisting}
struct generalSAM {
  static const int A = 26;
  static const int M = N<<1;
  static const char C = 'a';
  int sz, len[M], link[M], nex[M][A];
  generalSAM() { init(); }
  void init() {
    memset(nex, 0, sizeof(int)*A*sz);
    link[0] = -1; sz = 1;
  }
  int extend(int last, int c) {
    if (nex[last][c]) {
      int p = last, cur = nex[p][c];
      if (len[p]+1 == len[cur]) return cur;
      int q = sz++;
      len[q] = len[p]+1;
      memcpy(nex[q], nex[cur], sizeof nex[q]);
      for ( ; ~p && nex[p][c] == cur; p = link[p]) nex[p][c] = q;
      link[q] = link[cur];
      link[cur] = q;
      return q;
    }
    int cur = sz++, p = last;
    len[cur] = len[p]+1;
    for ( ; ~p && !nex[p][c]; p = link[p]) nex[p][c] = cur;
    if (p == -1) return link[cur] = 0, cur;
    int q = nex[p][c];
    if (len[p]+1 == len[q]) return link[cur] = q, cur;
    int clone = sz++;
    len[clone] = len[p]+1;
    memcpy(nex[clone], nex[q], sizeof nex[q]);
    for (; ~p && nex[p][c] == q; p = link[p]) nex[p][c] = clone;
    link[clone] = link[q];
    link[q] = link[cur] = clone;
    return cur;
  }
  void insert(const string &s) {
    int last = 0; for (char ch : s) last = extend(last, ch-C);
  }
  void insert(const char *s) {
    for (int i = 0, last = 0; s[i]; ++i) last = extend(last, s[i]-C);
  }
};
\end{lstlisting}

\subsection{多个字符串间的最长公共子串}

设有 $k$ 个字符串,每个结点建立长度为 $k$ 的标记,在 parent 树自底向上合并,若满足所有标记,则记录

(对于本题而言,可以仅为标记数组,若需要求出此子串的个数,则需要改成计数数组)(可用二进制或bitset)

\subsection{根号暴力}

在 parent 树上从字符串的每一个前缀的状态暴力往上跳(须标记vis)

例如可用此法记录上述的标记数组

证明:对于第 $i$ 个字符串,它最多会贡献 $\min{(n,\lvert s_i \rvert^2)}, n=\sum{\lvert s_i \rvert},O(n\sqrt{n})$

{\setmainfont{Consolas}
\begin{lstlisting}
  void jump(const string &s, int id) {
    int x = 0;
    for (char ch : s) {
      x = nex[x][ch-C];
      for (int y = x; y && vis[y] != id; y = link[y]) {
        vis[y] = id;
        ++k[y]; // 记录信息
      }
    }
  }
\end{lstlisting}

\subsection{树上本质不同路径数}

一颗无根树上任意一条路径必定可以在以某个叶节点为根时,变成一条从上到下的路径

暴力把所有叶子结点为根的树加入自动机,就这?

{\setmainfont{Consolas}
\begin{lstlisting}
void dfs(int u, int fa = 0, int last = 0) {
  int nex = gsam.extend(last, a[u]);
  for (int v : e[u]) if (v != fa) dfs(v, u, nex);
}
\end{lstlisting}

\subsection{循环同构问题}

CF235C:SAM 读入字符串是支持删除首字符的! 将一个字符串所有的循环同构串插入到 sam, 外加 vis 去重

{\setmainfont{Consolas}
\begin{lstlisting}
ll cyclic_query(const string &s) { // 循环同构
    static int id = 0, vis[M];
    ++id;
    ll res = 0;
    for (int i = 0, c, m = s.size(), u = 0, l = 0; i < m*2-1; ++i) {
      c = s[i%m]-C;
      while (u && !nex[u][c]) l = len[u = link[u]];
      if (nex[u][c]) u = nex[u][c], ++l;
      if (l > m && --l == len[link[u]]) u = link[u]; // 删除首字符
      if (l == m && vis[u] != id) res += cnt[u], vis[u] = id; // 去重,记录答案
    }
    return res;
  }
\end{lstlisting}

\section{ 最小表示法}

$S[i\cdots n]+S[1\cdots i-1] = T$ 则称 $S$ 与 $T$ **循环同构**

字符串 $S$ 的**最小表示**为与 $S$ 循环同构的所有字符串中字典序最小的字符串

$O(n)$

{\setmainfont{Consolas}
\begin{lstlisting}
int min_cyclic_string(const string &s) {
  int k = 0, i = 0, j = 1, n = s.size();
  while (k < n && i < n && j < n) {
    if (s[(i + k) % n] == s[(j + k) % n]) {
      k++;
    } else {
      s[(i + k) % n] > s[(j + k) % n] ? i = i + k + 1 : j = j + k + 1;
      if (i == j) i++;
      k = 0;
    }
  }
  return min(i, j);
}
\end{lstlisting}

\section{ Lyndon分解}

Lyndon 串:对于字符串 $s$,如果 $s$ 的字典序严格小于 $s$ 的所有后缀的字典序,我们称 $s$ 是简单串,或者 Lyndon 串。

Lyndon 分解:串 $s$ 的 Lyndon 分解记为 $s=w_1w_2\cdots w_k$,其中所有 $w_i$ 为简单串,并且他们的字典序按照非严格单减排序,即 $w_1\ge w_2\ge\cdots\ge w_k$。可以发现,这样的分解存在且唯一。

Duval 可以在 $O(n)$ 的时间内求出一个串的 Lyndon 分解。

{\setmainfont{Consolas}
\begin{lstlisting}
// duval_algorithm
vector<string> duval(string const& s) {
  int n = s.size(), i = 0;
  vector<string> factorization;
  while (i < n) {
    int j = i + 1, k = i;
    while (j < n && s[k] <= s[j]) {
      if (s[k] < s[j])
        k = i;
      else
        k++;
      j++;
    }
    while (i <= k) {
      factorization.push_back(s.substr(i, j - k));
      i += j - k;
    }
  }
  return factorization;
}
\end{lstlisting}


\section{ 回文树|回文自动机|PAM}
定理:对于一个字符串 $s$,它的本质不同回文子串个数最多只有 $\lvert s \rvert$ 个。

状态数,复杂度 $O(n)$

由于回文树的构造过程中,节点本身就是按照拓扑序插入,因此可以按序枚举所有状态实现树遍历

{\setmainfont{Consolas}
\begin{lstlisting}
struct PAM {
  static const int A = 26;
  static const char C = 'a';
  int sz, tot, last;
  int ch[N][A], len[N], fail[N];
  char s[N];
  PAM() { init(); }
  int node(int l) {
    ++sz;
    memset(ch[sz], 0, sizeof ch[sz]);
    len[sz] = l;
    fail[sz] = 0;
    return sz;
  }
  void init() {
    sz = -1;
    last = 0;
    s[tot = 0] = '#';
    node(0);
    fail[0] = node(-1);
  }
  int getfail(int x) {
    while (s[tot-len[x]-1] != s[tot]) x = fail[x];
    return x;
  }
  void insert(char c) {
    s[++tot] = c;
    int now = getfail(last);
    if (!ch[now][c-C]) {
      int x = node(len[now]+2);
      fail[x] = ch[getfail(fail[now])][c-C];
      ch[now][c-C] = x;
    }
    last = ch[now][c-C];
  }
};
\end{lstlisting}

\subsection{特殊log性质}

记字符串 $s$ 长度为 $i$ 的前缀为 $pre(s,i)$,长度为 $i$ 的后缀为 $suf(s,i)$

周期:若 $0< p \le |s|$,$\forall 1 \le i \le |s|-p,s[i]=s[i+p]$,就称 $p$ 是 $s$ 的周期。

border:若 $0 \le r < |s|$,$pre(s,r)=suf(s,r)$,就称 $pre(s,r)$ 是 $s$ 的 border。

周期和 border 的关系:$t$ 是 $s$ 的 border,当且仅当 $|s|-|t|$ 是 $s$ 的周期。

引理 $1$:$t$ 是回文串 $s$ 的后缀,$t$ 是 $s$ 的 border 当且仅当 $t$ 是回文串。

引理 $2$:$t$ 是回文串 $s$ 的 border ($|s|\le 2|t|$),$s$ 是回文串当且仅当 $t$ 是回文串。

引理 $3$:$t$ 是回文串 $s$ 的 border,则 $|s|-|t|$ 是 $s$ 的周期,$|s|-|t|$ 为 $s$ 的最小周期,当且仅当 $t$ 是 $s$ 的最长回文真后缀。

引理 $4$:$x$ 是一个回文串,$y$ 是 $x$ 的最长回文真后缀,$z$ 是 $y$ 的最长回文真后缀。令 $u,v$ 分别为满足 $x=uy,y=vz$ 的字符串,则有下面三条性质

$|u| \ge |v|$;

如果 $|u| > |v|$,那么 $|u| > |z|$;

如果 $|u| = |v|$,那么 $u=v$。

推论:$s$ 的所有回文后缀按照长度排序后,可以划分成 $\log |s|$ 段等差数列。

回文树上的每个节点 $u$ 需要多维护两个信息,$diff[u]$ 和 $slink[u]$。$diff[u]$ 表示节点 $u$ 和 $fail[u]$ 所代表的回文串的长度差,即 $len[u]-len[fail[u]]$。$slink[u]$ 表示 $u$ 一直沿着 fail 向上跳到第一个节点 $v$,使得 $diff[v] \neq diff[u]$,也就是 $u$ 所在等差数列中长度最小的那个节点。

{\setmainfont{Consolas}
\begin{lstlisting}
      diff[x] = len[x]-len[fail[x]];
      slink[x] = diff[x] == diff[fail[x]] ? slink[fail[x]] : fail[x];
\end{lstlisting}

{\setmainfont{Consolas}
\begin{lstlisting}
for (int i = 2; i <= sz; ++i)
  for (int j = i, k = slink[i]; j; j = k, k = slink[j])
    // 等差数列[len[k], len[j]] d = diff[j]
\end{lstlisting}

\subsection{最小回文划分}

问题描述:求最小的 $k$,使得字符串能分成 $k$ 段且每段都是回文串

暴力:$dp[i]=1+\min\limits_{s[j+1\cdots i]为回文串}dp[j]$

$g[v]$ 表示 $v$ 所在等差数列的 $dp$ 值之和,且 $v$ 是这个等差数列中长度最长的节点,则 $g[v]=\sum_{x,slink[x]=v} dp[i-len[x]]$,这里 $i$ 是当前枚举到的下标。(该题改求和为取min)

假设当前枚举到第 $i$ 个字符,回文树上对应节点为 $x$。$g[x]=g[fail[x]]+dp[i-(len[slink[x]]+diff[x])]$

{\setmainfont{Consolas}
\begin{lstlisting}
  void solve() {
    dp[0] = 1;
    for (int x = 0, i = 1; i <= tot; ++i) {
      while (s[i-len[x]-1] != s[i]) x = fail[x];
      x = ch[x][s[i]-C];
      for (int j = x; j > 1; j = slink[j]) {
        g[j] = dp[i-len[slink[j]]-diff[j]];
        if (diff[j] == diff[fail[j]]) update(g[j], g[fail[j]]);
        update(dp[i], g[j]);
      }
    }
  }
\end{lstlisting}

\subsection{回文级数优化回文树上dp}

问题描述:把字符串 $s$ 划分成若干个字符串 $t_1t_2\cdots t_k$,使得 $t_i=t_{k-i+1}$,求方案数

问题转化:将字符串转为 $s_1s_ns_2s_{n-1}\cdots$,原划分方案恰好对应了对新串进行偶数长度的回文划分的方案

tips:只需要考虑偶数下标位置的 $dp$ 值即可

\section{ 序列自动机}
序列自动机是接受且仅接受一个字符串的子序列的自动机。

{\setmainfont{Consolas}
\begin{lstlisting}
struct SeqAutomaton { // suppose string [1, n]
  static const int A = 26;
  static const char C = 'a';
  int nex[N][A];
  void build(const string &s) {
    memset(nex[s.size()], 0, sizeof nex[0]);
    for (int i = s.size(); i; --i) {
      memcpy(nex[i-1], nex[i], sizeof nex[0]);
      nex[i-1][s[i-1]-C] = i;
    }
  }
};
\end{lstlisting}

\section{ Main-Lorentz 算法}

我们将一个字符串连续写两遍所产生的新字符串称为 重串 (tandem repetition)。将被重复的这个字符串称为原串。

如果一个重串的原串不是重串,则我们称这个重串为 本原重串 (primitive repetition)。可以证明,本原重串最多有 $O(n \log n)$ 个。

如果我们把一个重串用 Crochemore 三元组 $(i, p, r)$ 进行压缩,其中 $i$ 是重串的起始位置,$p$ 是该重串某个循环节的长度(注意不是原串长度!),$r$ 为这个循环节重复的次数。则某字符串的所有重串可以被 $O(n \log n)$ 个 Crochemore 三元组表示。

\part{图论|树论}
\section{ DFS树}
\section{ 树的重心}
{\setmainfont{Consolas}
\begin{lstlisting}
void treedp(int cur, int fa) {
  s[cur] = c[cur];
  for(int i = fir[cur]; i; i = nex[i]) {
    if(e[i] == fa) continue;
    treedp(e[i], cur);
    s[cur] += s[e[i]];
    maxs[cur] = max(maxs[cur], s[e[i]]);
  }
  maxs[cur] = max(maxs[cur], sum-s[cur]);
}
\end{lstlisting}
\section{ 最大团}
最大独立集数=补图的最大团
{\setmainfont{Consolas}
\begin{lstlisting}
struct MaxClique {
  vector<int> res, tmp, cnt;
  bool dfs(int p) {
    for (int i = p+1, flag; i <= n; ++i) {
      if (cnt[i]+tmp.size() <= res.size()) return false;
      if (!g[p][i]) continue;
      flag = 1;
      for (int j : tmp)
        if (!g[i][j]) flag = 0;
      if (!flag) continue;
      tmp.push_back(i);
      if (dfs(i)) return true;
      tmp.pop_back();
    }
    if (tmp.size() > res.size()) {
      res = tmp;
      return true;
    }
    return false;
  }
  void solve() {
    vector<int>(n+1, 0).swap(cnt);
    vector<int>().swap(res);
    for (int i = n; i; --i) {
      vector<int>(1, i).swap(tmp);
      dfs(i);
      cnt[i] = res.size();
    }
  }
} MC;
\end{lstlisting}


\section{ 稳定婚姻匹配}
{\setmainfont{Consolas}
\begin{lstlisting}
template <typename T = int> struct Stable_Marriage {
  int t[N], b[N], g[N], rkb[N][N], rkg[N][N];
  T wb[N][N], wg[N][N];
  queue<int> q;
  void init(const int &n) {
    queue<int>().swap(q);
    memset(t, 0, sizeof(int)*(n+3));
    memset(b, 0, sizeof(int)*(n+3));
    memset(g, 0, sizeof(int)*(n+3));
    for (int i = 1; i <= n; ++i) {
      q.push(i);
      for (int j = 1; j <= n; ++j)
        rkb[i][j] = rkg[i][j] = j;
      sort(rkb[i]+1, rkb[i]+n+1,
         [&](const int &x, const int &y) { return wb[i][y] < wb[i][x]; });
    }
  }
  bool match(const int &x, const int &y) {
    if (g[y]) {
      if (wg[y][x] < wg[y][g[y]]) return false;
      b[g[y]] = 0;
      q.push(g[y]);
    }
    b[x] = y; g[y] = x;
    return true;
  }
  void gale_shapely(const int &n) {
    init(n);
    while (q.size()) {
      int x = q.front(); q.pop();
      int y = rkb[x][++t[x]];
      if (!match(x, y)) q.push(x);
    }
  }
};
\end{lstlisting}


\section{ 最小生成树}
Prim
{\setmainfont{Consolas}
\begin{lstlisting}
inline void prim() {
  fill(dis, dis+n+1, INF);
  dis[1] = 0;
  for(int t = 1; t <= n; ++t)
  {
    int mini = 0;
    for(int i = 1; i <= n; ++i)
      if(!vis[i] && dis[i] < dis[mini])
        mini = i;
    vis[mini] = 1;
    ans += dis[mini];
    for(int i = 1; i <= n; ++i)
      if(!vis[i]) dis[i] = min(dis[i], calc(mini, i));
  }
}
\end{lstlisting}

\section{ 二分图}
\subsection{二分图最大匹配}
增广路算法 Augmenting Path Algorithm $O(nm)$
{\setmainfont{Consolas}
\begin{lstlisting}
bool check(int u) {
  for (int v : e[u]) {
    if (vis[v]) continue;
    vis[v] = 1;
    if (!co[v] || check(co[v])) {
      co[v] = u;
      return true;
    }
  }
  return false;
}

inline int solve() {
  int res = 0;
  memset(co, 0, sizeof co);
  for (int i = 1; i <= n; ++i) {
    memset(vis, 0, sizeof(int)*(n+3));
    res += check(i);
  }
  return res;
}
\end{lstlisting}
网络流 $O(\sqrt{n}m)$
\subsection{二分图最大权匹配}
Hungarian Algorithm (Kuhn-Munkres Algorithm)
匈牙利算法又称为 KM 算法,可以在 $O(n^3)$ 时间内求出二分图的 最大权完美匹配。
\subsection{二分图最小顶点覆盖}
定义:假如选了一个点就相当于覆盖了以它为端点的所有边。最小顶点覆盖就是选择最少的点来覆盖所有的边。

定理:最小顶点覆盖等于二分图的最大匹配。
\subsection{最大独立集}

定义:选出一些顶点使得这些顶点两两不相邻,则这些点构成的集合称为独立集。找出一个包含顶点数最多的独立集称为最大独立集。

定理:最大独立集 = 所有顶点数 - 最小顶点覆盖 = 所有顶点数 - 最大匹配



\section{ 最近公共祖先|LCA}
\subsection{倍增}
{\setmainfont{Consolas}
\begin{lstlisting}
struct LCA {
  static const int NN = (int)log2(N)+3;
  int f[N][NN], d[N], lg2[N];
  LCA() { for (int i = 2; i < N; ++i) lg2[i] = lg2[i>>1]+1; }
  template <typename TT>
  void build(const TT e[], const int &u = 1, const int &fa = 0) {
    d[u] = d[fa]+1;
    f[u][0] = fa;
    for (int i = 1; (1<<i) <= d[u]; ++i)
      f[u][i] = f[f[u][i-1]][i-1];
    for (auto v : e[u]) if (v != fa)
      build(e, v, u);
  }
  int get(int x, int y) {
    if (d[x] < d[y]) swap(x, y);
    while (d[x] > d[y])
      x = f[x][lg2[d[x]-d[y]]];
    if (x == y) return x;
    for (int i = lg2[d[x]]; i >= 0; --i)
      if(f[x][i] != f[y][i])
        x = f[x][i], y = f[y][i];
    return f[x][0];
  }
};
\end{lstlisting}

\subsection{树剖}
{\setmainfont{Consolas}
\begin{lstlisting}
struct HLD {
  int fa[N], d[N], num[N], son[N], tp[N];
  vector<int> *e;
  void build(vector<int> *e, const int &rt = 1) {
    this->e = e;
    fa[rt] = 0;
    dfs1(rt);
    dfs2(rt);
  }
  void dfs1(const int &u = 1) {
    d[u] = d[fa[u]]+1;
    num[u] = 1;
    son[u] = 0;
    for (const int &v : e[u]) if (v != fa[u]) {
      fa[v] = u;
      dfs1(v);
      num[u] += num[v];
      if (num[v] > num[son[u]]) son[u] = v;
    }
  }
  void dfs2(const int &u = 1) {
    tp[u] = son[fa[u]] == u ? tp[fa[u]] : u;
    if (son[u]) dfs2(son[u]);
    for (const int &v : e[u]) if (v != son[u] && v != fa[u])
      dfs2(v);
  }
  int lca(int x, int y) {
    while (tp[x] != tp[y]) {
      if (d[tp[x]] < d[tp[y]]) swap(x, y);
      x = fa[tp[x]];
    }
    return d[x] < d[y] ? x : y;
  }
};
\end{lstlisting}

\subsection{欧拉序}

dfs 时进入一个节点加入序列,回溯回来也加一次

lca 转变为区间深度最小的点

\subsection{带权LCA}
{\setmainfont{Consolas}
\begin{lstlisting}
template <typename T>
struct LCA {
  static const int NN = (int)log2(N)+3;
  int f[N][NN], d[N], lg2[N];
  T w[N][NN], init_val = 0;
  LCA() {
    for (int i = 2; i < N; ++i) lg2[i] = lg2[i>>1]+1;
    init();
  }
  // set sum or min or max, and don't forget to set init_val
  T update(const T &x, const T &y) { return x+y; }
  void init(const int &n = N-1) {
    fill(w[0], w[0]+(n+1)*NN, init_val);
  }
  template <typename TT>
  void build(const TT e[], const int &u = 1, const int &fa = 0) {
    d[u] = d[fa]+1;
    f[u][0] = fa;
    for (int i = 1; (1<<i) <= d[u]; ++i) {
      f[u][i] = f[f[u][i-1]][i-1];
      w[u][i] = update(w[u][i-1], w[f[u][i-1]][i-1]);
    }
    for (auto v : e[u]) if (v.first != fa) {
      w[v.first][0] = v.second;
      build(e, v.first, u);
    }
  }
  T get(int x, int y) {
    T res = init_val;
    if (d[x] < d[y]) swap(x, y);
    while (d[x] > d[y]) {
      res = update(res, w[x][lg2[d[x]-d[y]]]);
      x = f[x][lg2[d[x]-d[y]]];
    }
    if (x == y) return res;
    for (int i = lg2[d[x]]; i >= 0; --i)
      if(f[x][i] != f[y][i]) {
        res = update(res, w[x][i]);
        res = update(res, w[y][i]);
        x = f[x][i], y = f[y][i];
      }
    return update(res, update(w[x][0], w[y][0]));
  }
};
\end{lstlisting}
\section{ 树上差分}
{\setmainfont{Consolas}
\begin{lstlisting}
template <typename T>
struct Tree {
  T val[N];
  void update_point(const int &x, const int &y, const T &k) {
    int _lca = lca(x, y);
    val[x] += k; val[y] += k;
    val[_lca] -= k; val[f[_lca][0]] -= k;
  }
  void update_edge(const int &x, const int &y, const T &k) {
    int _lca = lca(x, y);
    val[x] += k; val[y] += k; val[_lca] -= 2*k;
  }
  void dfs(const int &u = 1, const int &fa = 0) {
    for (int v : e[u]) if (v != fa) {
      dfs(v, u);
      val[u] += val[v];
    }
  }
};
\end{lstlisting}
\section{ 树链剖分}
{\setmainfont{Consolas}
\begin{lstlisting}
template <typename T>
struct HLD {
  int dfn;
  int fa[N], d[N], num[N], son[N], id[N], tp[N];
  T init_val[N];
  vector<int> *e;
  SegmentTree ST;
  void build(vector<int> *e, T *a, int n, int rt = 1) {
    this->e = e;
    fa[rt] = dfn = 0;
    dfs1(rt);
    dfs2(rt);
    for (int i = 1; i <= n; ++i)
      init_val[id[i]] = a[i];
    ST.build(init_val, n);
  }
  void dfs1(const int &u = 1) {
    d[u] = d[fa[u]]+1;
    num[u] = 1;
    son[u] = 0;
    for (const int &v : e[u]) if (v != fa[u]) {
      fa[v] = u;
      dfs1(v);
      num[u] += num[v];
      if (num[v] > num[son[u]]) son[u] = v;
    }
  }
  void dfs2(const int &u = 1) {
    tp[u] = son[fa[u]] == u ? tp[fa[u]] : u;
    id[u] = ++dfn;
    if (son[u]) dfs2(son[u]);
    for (const int &v : e[u]) if (v != son[u] && v != fa[u])
      dfs2(v);
  }
  void add_sons(const int &x, const T &k) { ST.add(id[x], id[x]+num[x]-1, k); }
  void add(int x, int y, const T &k, const int &is_edge = 0) {
    while (tp[x] != tp[y]) {
      if (d[tp[x]] < d[tp[y]]) swap(x, y);
      ST.add(id[tp[x]], id[x], k);
      x = fa[tp[x]];
    }
    if (d[x] > d[y]) swap(x, y);
    ST.add(id[x], id[y], k);
    if (is_edge) ST.add(id[x], -k);
  }
  T query_sons(const int &x) { return ST.query(id[x], id[x]+num[x]-1); }
  T query(const int &x) { return ST.query(id[x]); }
  T query(int x, int y) {
    T res = 0;
    while (tp[x] != tp[y]) {
      if (d[tp[x]] < d[tp[y]]) swap(x, y);
      res += ST.query(id[tp[x]], id[x]);
      x = fa[tp[x]];
    }
    if (d[x] > d[y]) swap(x, y);
    return res+ST.query(id[x], id[y]);
  }
};
\end{lstlisting}


\section{ 网络流}
\subsection{最大流}
\subsubsection{Dinic}
普通情况下 $O(n^2m)$
二分图中 $O(\sqrt{n}m)$
{\setmainfont{Consolas}
\begin{lstlisting}
template <typename T>
struct Dinic {
  struct EDGE {
    int v, nex;
    T w;
    EDGE(const int &_v, const int &_nex, const T &_w) : v(_v), nex(_nex), w(_w) {}
  };
  vector<EDGE> e;
  int n, s, t;
  int fir[N], dep[N], cur[N];
  Dinic() { e.reserve(N<<2); }
  T work(const int &_s, const int &_t) {
    s = _s; t = _t;
    T maxflow = 0, flow;
    while (bfs())
      while ((flow = dfs(s, INF)))
        maxflow += flow;
    return maxflow;
  }
  void init(const int &_n) {
    n = _n;
    e.clear();
    memset(fir, -1, sizeof(int)*(n+3));
  }
  void add_edge(const int &u, const int &v, const T &w) {
    e.emplace_back(v, fir[u], w); fir[u] = e.size()-1;
    e.emplace_back(u, fir[v], 0); fir[v] = e.size()-1;
  }
  bool bfs() {
    queue<int> q;
    memset(dep, 0, sizeof(int)*(n+3));
    q.push(s);
    dep[s] = 1;
    for (int i = 0; i <= n; ++i) cur[i] = fir[i];
    while (q.size()) {
      int u = q.front();
      q.pop();
      for (int i = fir[u], v; i != -1; i = e[i].nex) {
        v = e[i].v;
        if (dep[v] || !e[i].w) continue;
        dep[v] = dep[u]+1;
        if (v == t) return true;
        q.push(v);
      }
    }
    return false;
  }
  T dfs(const int &u, const T &flow) {
    if (!flow || u == t) return flow;
    T rest = flow, now;
    for (int &i = cur[u], v; i != -1; i = e[i].nex) {
      v = e[i].v;
      if (dep[v] != dep[u]+1 || !e[i].w) continue;
      now = dfs(v, min(rest, e[i].w));
      if (!now) {
        dep[v] = 0;
      } else {
        e[i].w -= now;
        e[i^1].w += now;
        rest -= now;
        if (rest == flow) break;
      }
    }
    return flow-rest;
  }
};
\end{lstlisting}
\subsection{费用流}
\subsubsection{ZKW\_SPFA}
{\setmainfont{Consolas}
\begin{lstlisting}
template <typename T>
struct ZKW_SPFA {
  struct Edge {
    int v, nex;
    T w, c; // edge wight and cost
    Edge(const int &_v, const int &_nex, const T &_w, const T &_c) \
    : v(_v), nex(_nex), w(_w), c(_c) {}
  };
  vector<Edge> e;
  int n, s, t;
  int fir[N], vis[N];
  T maxflow, mincost;
  T dis[N];
  ZKW_SPFA() { e.reserve(N<<4); }
  void init(const int &_n) {
    n = _n;
    maxflow = mincost = 0;
    e.clear();
    memset(fir, -1, sizeof(int)*(n+3));
  }
  void add_edge(const int &u, const int &v, const T &w = 1, const T &c = 0) {
    e.emplace_back(v, fir[u], w, c); fir[u] = e.size()-1;
    e.emplace_back(u, fir[v], 0, -c); fir[v] = e.size()-1;
  }
  pair<T, T> work(const int &_s, const int &_t) {
    s = _s; t = _t;
    while (spfa()) {
      vis[t] = 1;
      while (vis[t]) {
        memset(vis, 0, sizeof(int)*(n+3));
        maxflow += dfs(s, INF);
      }
    }
    return {maxflow, mincost};
  }
  private:
  bool spfa() {
    memset(dis, 0x3f, sizeof(T)*(n+3));
    memset(vis, 0, sizeof(int)*(n+3));
    deque<int> q;
    q.push_back(t);
    dis[t] = 0;
    vis[t] = 1;
    while (q.size()) {
      int u = q.front(); q.pop_front();
      for (int i = fir[u], v; ~i; i = e[i].nex) {
        v = e[i].v;
        if (!e[i^1].w || dis[v] <= dis[u]+e[i^1].c) continue;
        dis[v] = dis[u]+e[i^1].c;
        if (vis[v]) continue;
        vis[v] = 1;
        if (q.size() && dis[v] < dis[q.front()]) q.push_front(v);
        else q.push_back(v);
      }
      vis[u] = 0;
    }
    return dis[s] < INF;
  }
  T dfs(const int &u, const T &flow) {
    vis[u] = 1;
    if (u == t || flow <= 0) return flow;
    T res, used = 0;
    for (int i = fir[u], v; ~i; i = e[i].nex) {
      v = e[i].v;
      if (vis[v] || !e[i].w || dis[u] != dis[v]+e[i].c) continue;
      res = dfs(v, min(e[i].w, flow-used));
      if (!res) continue;
      mincost += res*e[i].c;
      e[i].w -= res;
      e[i^1].w += res;
      used += res;
      if (used == flow) break;
    }
    return used;
  }
};
\end{lstlisting}
\subsection{上下界网络流}

\subsection{全局最小割StoerWagner}

{\setmainfont{Consolas}
\begin{lstlisting}
template <typename T>
T sw(int n, T dis[N][N]) {
  int s, t;
  T res = INF;
  vector<int> dap(n+1, 0), ord(n+1, 0), vis;
  vector<T> w;
  function<T(int)> proc = & {
    vis = vector<int>(n+1, 0);
    w = vector<T>(n+1, 0);
    w[0] = -1;
    for (int i = 1; i <= n-x+1; ++i) {
      int mx = 0;
      for (int j = 1; j <= n; ++j) {
        if (!dap[j] && !vis[j] && w[j] > w[mx]) mx = j;
      }
      vis[mx] = 1;
      ord[i] = mx;
      for (int j = 1; j <= n; ++j) {
        if (!dap[j] && ! vis[j]) w[j] += dis[mx][j];
      }
    }
    s = ord[n-x];
    t = ord[n-x+1];
    return w[t];
  };
  for (int i = 1; i < n; ++i) {
    res = min(res, proc(i));
    dap[t] = 1;
    for (int j = 1; j <= n; ++j) {
      dis[s][j] += dis[t][j];
      dis[j][s] += dis[j][t];
    }
  }
  return res;
}
\end{lstlisting}


\section{ 最短路}
\subsection{SPFA}
{\setmainfont{Consolas}
\begin{lstlisting}
inline void SPFA() {
  fill(dis+1, dis+n+1, INT_MAX);
  dis[S] = 0;
  head = tail = 0;
  q[++tail] = S;
  while(head < tail) {
    int cur = q[++head];
    for(int i = fir[cur], to, tmp; i; i = nex[i]) {
      to = ver[i];
      tmp = dis[cur]+w[i];
      if(tmp >= dis[to]) continue;
      dis[to] = tmp;
      q[++tail] = to;
    }
  }
}
\end{lstlisting}


\section{ 负环}
{\setmainfont{Consolas}
\begin{lstlisting}
// 返回true有负环,返回false没负环
inline bool SPFA() {
  q[++tail] = 1;
  vis[1] = 1;
  cnt[1] = 1;
  dis[1] = 0;
  while(head < tail) {
    int cur = q[(++head)%Maxn];
    vis[cur] = 0;
    for(int i = fir[cur], to; i; i = nex[i]) {
      to = ver[i];
      if(dis[cur]+w[i] < dis[to]) {
        dis[to] = dis[cur]+w[i];
        if(!vis[to]) {
          q[(++tail)%Maxn] = to;
          vis[to] = 1;
          if(++cnt[to] > n) return true;
        }
      }
    }
  }
  return false;
}
\end{lstlisting}


\section{ 割点}
{\setmainfont{Consolas}
\begin{lstlisting}
void tarjan(int cur, int fa) {
  dfn[cur] = low[cur] = ++_dfn;
  int child = 0;
  for(auto i : e[cur]) {
    if(!dfn[i]) {
      child++;
      tarjan(i, fa);
      low[cur] = min(low[cur], low[i]);
      if(cur != fa && low[i] >= dfn[cur]) flag[cur] = 1;
    }
    low[cur] = min(low[cur], dfn[i]);
  }
  if(cur == fa && child >= 2) flag[cur] = 1;
}
\end{lstlisting}


\section{ SCC强连通分量|Tarjan}
\subsection{递归版本}
{\setmainfont{Consolas}
\begin{lstlisting}
int _dfn, _col, _top;
int dfn[N], low[N], vis[N], col[N], sta[N];

void tarjan(const int &u) {
  dfn[u] = low[u] = ++_dfn;
  vis[u] = 1;
  sta[++_top] = u;
  for (int v : e[u]) {
    if (!dfn[v]) {
      tarjan(v);
      low[u] = min(low[u], low[v]);
    } else if (vis[v]) {
      low[u] = min(low[u], low[v]);
    }
  }
  if (dfn[u] == low[u]) {
    ++_col;
    do {
      col[sta[_top]] = _col;
      vis[sta[_top]] = 0;
    } while (sta[_top--] != u);
  }
}
\end{lstlisting}

\section{ 缩点}
{\setmainfont{Consolas}
\begin{lstlisting}
void tarjan(int u) {
  dfn[u] = low[u] = ++_dfn;
  vis[u] = 1;
  sta[++top] = u;
  for (int v : e[u]) {
    if (!dfn[v]) {
      tarjan(v);
      low[u] = min(low[u], low[v]);
    } else if (vis[v]) {
      low[u] = min(low[u], low[v]);
    }
  }
  if (dfn[u] == low[u]) {
    w_col[++_col] = 0;
    do {
      col[sta[top]] = _col;
      vis[sta[top]] = 0;
      w_col[_col] += w[sta[top]];
    } while (sta[top--] != u);
  }
}

inline void suodian() {
  for (int i = 1; i <= n; ++i) {
    if (!dfn[i]) tarjan(i);
  }
  for (int i = 1; i <= n; ++i) {
    for (int j : e[i]) {
      if (col[i] == col[j]) continue;
      e_col[col[i]].push_back(col[j]);
    }
  }
}
\end{lstlisting}
\section{ 2-SAT}
\subsection{SCC Tarjan}
$O(n+m)$
{\setmainfont{Consolas}
\begin{lstlisting}
struct TWO_SAT { // node start from 0
  int top, _dfn, _scc;
  int dfn[N<<1], low[N<<1], stk[N<<1], scc[N<<1], res[N];
  vector<int> e[N<<1];
  void init(const int &n) {
    top = 0;
    memset(dfn, 0, sizeof(int)*n*2);
    memset(low, 0, sizeof(int)*n*2);
    memset(scc, 0, sizeof(int)*n*2);
    for (int i = 0; i < n<<1; ++i) vector<int>().swap(e[i]);
  }
  // if u then v
  void add_edge(const int &u, const int &v) {
    e[u].emplace_back(v);
  }
  void add_edge(const int &u, const int &uv, const int &v, const int &vv) {
    e[u<<1^uv].emplace_back(v<<1^vv);
  }
  // pt i ==> i<<1 && i<<1|1 ==> 0 && 1
  inline bool work(const int &n) {
    for (int i = 0; i <= n<<1; ++i)
      if (!dfn[i]) tarjan(i);
    for (int i = 0; i < n; ++i) {
      if (scc[i<<1] == scc[i<<1|1]) return false;
      res[i] = scc[i<<1] > scc[i<<1|1];
    }
    return true;
  }
  void tarjan(const int &u) {
    dfn[u] = low[u] = ++_dfn;
    stk[++top] = u;
    for (int &v : e[u]) {
      if (!dfn[v]) {
        tarjan(v);
        low[u] = min(low[u], low[v]);
      } else if (!scc[v]) {
        low[u] = min(low[u], dfn[v]);
      }
    }
    if (dfn[u] == low[u]) {
      ++_scc;
      do {
        scc[stk[top]] = _scc;
      } while (stk[top--] != u);
    }
  }
};
\end{lstlisting}
\subsection{DFS}
$O(nm)$
所求结果字典序最小
{\setmainfont{Consolas}
\begin{lstlisting}
struct TWO_SAT {
  int n, cnt;
  int res[N], mem[N<<1], mark[N<<1];
  vector<int> e[N<<1];
  void init(const int &_n) {
    n = _n;
    memset(mark, 0, sizeof(int)*n*2);
    for (int i = 0; i < n<<1; ++i) vector<int>().swap(e[i]);
  }
  // if u then v
  void add_edge(const int &u, const int &v) {
    e[u].emplace_back(v);
  }
  // pt i ==> i<<1 && i<<1|1 ==> 0 && 1
  void add_edge(const int &u, const int &uv, const int &v, const int &vv) {
    e[u<<1|uv].emplace_back(v<<1|vv);
  }
  // tag 0 any 1 smallest
  bool work() {
    for (int i = 0; i < n; ++i) {
      if (mark[i<<1] || mark[i<<1|1]) continue;
      cnt = 0;
      if (!dfs(i<<1)) {
        while (cnt) mark[mem[cnt--]] = 0;
        if (!dfs(i<<1|1)) return false;
      }
    }
    for (int i = 0; i < n<<1; ++i) if (mark[i]) res[i>>1] = i&1;
    return true;
  }
  bool dfs(const int &u) {
    if (mark[u^1]) return false;
    if (mark[u]) return true;
    mark[mem[++cnt] = u] = 1;
    for (int v : e[u]) if (!dfs(v)) return false;
    return true;
  }
};
\end{lstlisting}

\section{ 虚树}

{\setmainfont{Consolas}
\begin{lstlisting}
vector<int> ve[N];
void virtual_tree_clear(const int &u = 1) {
  for (const int &v : ve[u]) virtual_tree_clear(v);
  ve[u].clear();
}

// return the root of virtual tree
int virtual_tree_build(int vset[], const int &k) {
  static int stk[N], top;
  // id ==> dfn rank, d ==> depth
  int *id = hld.id, *d = hld.d;
  sort(vset+1, vset+k+1, & {
    return id[x] < id[y];
  });
  top = 0;
  int x, z;
  for (int i = 1; i <= k; ++i) {
    if (top && (z = hld.lca(vset[i], stk[top])) != stk[top]) {
      x = stk[top--];
      while (top && d[stk[top]] > d[z]) {
        ve[stk[top]].emplace_back(x);
        x = stk[top--];
      }
      ve[z].emplace_back(x);
      if (!top || stk[top] != z) stk[++top] = z;
    }
    stk[++top] = vset[i];
  }
  x = stk[top--];
  while (top) {
    ve[stk[top]].emplace_back(x);
    x = stk[top--];
  }
  // if (x != 1) ve[1].emplace_back(x); // force root at 1
  return x;
}
\end{lstlisting}

\section{ 线段树优化建图}

{\setmainfont{Consolas}
\begin{lstlisting}
template <typename T>
struct SegmentTreeGarph {
  struct TreeNode {
    int l, r;
    int ls, rs;
  } tr[N*3];
  vector<pair<int, T>> *e;
  int tot, root[2];
  // op [down, 0] [up, 1]
  template <typename E>
  void build(const int &n, E *_e) {
    tot = n;
    e = _e;
    for (int i = 1; i <= n; ++i) tr[i].l = tr[i].r = i;
    build(1, n, root[0], 0);
    build(1, n, root[1], 1);
  }
  void build(const int &l, const int &r, int &i, const int &op) {
    if (l == r) return i = l, void();
    i = ++tot;
    tr[i].l = l; tr[i].r = r;
    int mid = (l+r)>>1;
    build(l, mid, tr[i].ls, op);
    build(mid+1, r, tr[i].rs, op);
    e[op ? tr[i].ls : i].emplace_back(op ? i : tr[i].ls, 0);
    e[op ? tr[i].rs : i].emplace_back(op ? i : tr[i].rs, 0);
  }
  void insert(const int &o, const int &l, const int &r, const T &w,
      const int &op) {
    if (l == r) e[op ? l : o].emplace_back(op ? o : l, w);
    else insert(o, l, r, w, op, root[op]);
  }
  void insert(const int &o, const int &l, const int &r, const T &w,
      const int &op, const int &i) {
    if (tr[i].l >= l && tr[i].r <= r) {
      e[op ? i : o].emplace_back(op ? o : i, w);
      return;
    }
    int mid = (tr[i].l+tr[i].r)>>1;
    if (l <= mid) insert(o, l, r, w, op, tr[i].ls);
    if (r >  mid) insert(o, l, r, w, op, tr[i].rs);
  }
};
\end{lstlisting}

\section{ 矩阵树定理|Kirchhoff}

解决一张图的生成树个数计数问题(详情见oi-wiki)

\section{ 斯坦纳树}

给定连通图 $G$ 中的 $n$ 个点 $m$ 条边与 $k$ 个关键点,连接 $k$ 个关键点,使得生成树的所有边的权值和最小。

我们使用状态压缩动态规划来求解。用 $f(i,S)$ 表示以 $i$ 为根的一棵树,包含集合 $S$ 中所有点的最小边权值和。

\subsection{边权最小}

- 首先对连通的子集进行转移, $f(i,S)\leftarrow \min(f(i,S),f(i,T)+f(i,S-T))$ 。

- 在当前的子集连通状态下进行边的松弛操作, $f(i,S)\leftarrow \min(f(i,S),f(j,S)+w(j,i))$ 

复杂度 $O(n\times 3^k+m\log m\times 2^k)$

{\setmainfont{Consolas}
\begin{lstlisting}
int dp[N][1<<K];
vector<pair<int, int>> e[N]; // e[u] = {w, v}
priority_queue<pair<int, int>> q;

void dijkstra(int s) {
  while (q.size()) {
    int ud = -q.top().first;
    int u = q.top().second;
    q.pop();
    if (ud > dp[u][s]) continue;
    for (int i = 0, v, w; i < (int)e[u].size(); ++i) {
      w = e[u][i].first;
      v = e[u][i].second;
      if (dp[v][s] <= dp[u][s]+w) continue;
      dp[v][s] = dp[u][s]+w;
      q.push({-dp[v][s], v});
    }
  }
}

int steiner_tree(int *p) { // p[] is key point
  memset(dp, 0x3f, sizeof dp);
  for (int i = 0; i < k; ++i) dp[p[i]][1<<i] = 0;
  for (int s = 1; s < 1<<k; ++s) {
    for (int i = 1; i <= n; ++i) {
      for (int t = s&(s-1); t; t = s&(t-1))
        dp[i][s] = min(dp[i][s], dp[i][s^t]+dp[i][t]);
      if (dp[i][s] != INF) q.push({-dp[i][s], i});
    }
    dijkstra(s);
  }
  return dp[p[0]][(1<<k)-1];
}
\end{lstlisting}

\subsection{点权最小}

- $f(i,S)\leftarrow \min(f(i,S),f(i,T)+f(i,S-T)-a_i)$ 。由于此处合并时同一个点 $a_i$ ,会被加两次,所以减去。

- $f(i,S)\leftarrow \min(f(i,S),f(j,S)+w(j,i))$ 。

\subsection{路径输出}

{\setmainfont{Consolas}
\begin{lstlisting}
void dfs(int u, int s){
    if(!pre[u][s].second) return;
    // print
    if(pre[u][s].first == u) dfs(u, s^pre[u][s].second);
    dfs(pre[u][s].first, pre[u][s].second);
}
\end{lstlisting}

\section{ 树上背包}

时间复杂度 $O(n^2)$

{\setmainfont{Consolas}
\begin{lstlisting}
void dfs(int u) {
  num[u] = 1;
  for (int &v : e[u]) {
    dfs(v);
    for (int i = min(m, num[u]+num[v]); j; --j) {
      for (int j = max(0, i-num[u]); j <= min(num[v], i); ++j) { // i-j >= num[u]
        dp[u][i] = max(dp[u][i], dp[u][i-j]+dp[v][j]);
      }
    }
    num[u] += num[v];
  }
}
\end{lstlisting}

\section{ 仙人掌}

两点之间最短路
{\setmainfont{Consolas}
\begin{lstlisting}
namespace Cactus {

#define log(x) (31-__builtin_clz(x))
typedef long long ll;
typedef pair<int, ll> pil;
const int NN = (int)log2(N)+3;
int n, _dfn, cnt;
int f[N<<1][NN], dep[N<<1], dfn[N], from[N];
// od 为仙人掌上距离, dis 为圆方树上距离, cir 为环上边权和
ll od[N], dis[N<<1], cir[N];
vector<pil> *e, ce[N<<1];

bool dfs(int u) {
  dfn[u] = ++_dfn;
  for (const pil &edge : e[u]) {
    int v = edge.first;
    if (v == f[u][0]) continue;
    ll w = edge.second;
    if (!dfn[v]) {
      f[v][0] = u;
      dep[v] = dep[u]+1;
      od[v] = od[u]+w;
      if (!dfs(v)) return false;
      if (!from[v]) ce[u].emplace_back(v, w);
    } else if (dfn[v] < dfn[u]) {
      cir[++cnt] = od[u]-od[v]+w;
      ce[v].emplace_back(n+cnt, 0);
      for (int x = u; x != v; x = f[x][0]) {
        if (from[x]) return false;
        from[x] = cnt;
        ll ww = od[x]-od[v];
        ce[n+cnt].emplace_back(x, min(ww, cir[cnt]-ww));
      }
    }
  }
  return true;
}
// 带权倍增LCA
void build_lca(int u) {
  for (int i = 1; (1<<i) <= dep[u]; ++i)
    f[u][i] = f[f[u][i-1]][i-1];
  for (const pil &edge : ce[u]) {
    int v = edge.first;
    dep[v] = dep[u]+1;
    dis[v] = dis[u]+edge.second;
    f[v][0] = u;
    build_lca(v);
  }
}

bool build(int _n, vector<pil> *_e) {
  n = _n; e = _e;
  _dfn = cnt = 0;
  if (!dfs(1)) return false;
  dep[1] = 1;
  build_lca(1);
  return true;
}

ll query(int u, int v) {
  if (dep[u] < dep[v]) swap(u, v);
  int x = u, y = v;
  while (dep[x] > dep[y]) x = f[x][log(dep[x]-dep[y])];
  if (x == y) return dis[u]-dis[v];
  for (int i = log(dep[x]); i >= 0; --i)
    if(f[x][i] != f[y][i]) x = f[x][i], y = f[y][i];
  if (f[x][0] <= n) return dis[u]+dis[v]-2*dis[f[x][0]];
  ll d = abs(od[x]-od[y]);
  return dis[u]-dis[x]+dis[v]-dis[y]+min(d, cir[f[x][0]-n]-d);
}

} /* namespace Cactus */
\end{lstlisting}

\subsection{仙人掌DP}

{\setmainfont{Consolas}
\begin{lstlisting}
// from[i] 节点i与父亲的边所在环的编号
// tp[i]环i深度最小的节点编号, bm[i]环i深度最大的节点编号
int _dfn, cnt, fa[N], dfn[N], from[N], tp[N], bm[N];

bool dfs(int u) {
  dfn[u] = ++_dfn;
  for (const int &v : e[u]) {
    if (v == fa[u]) continue;
    if (!dfn[v]) {
      fa[v] = u;
      dfs(v);
      // dp
    } else if (dfn[v] < dfn[u]) {
      ++cnt;
      tp[cnt] = v; bm[cnt] = u;
      for (int x = u; x != v; x = fa[x]) {
        if (from[x]) return false;
        from[x] = cnt;
      }
    }
  }
}
\end{lstlisting}
\section{ 补图DFS}
{\setmainfont{Consolas}
\begin{lstlisting}
void bfs(int S = 1) {
  static set<int> st[2]; // 存储未访问的点
  static queue<int> q;
  memset(dis+1, -1, sizeof(int)*n);
  dis[S] = 0;
  q.push(S);
  st[0].clear(); st[1].clear();
  for (int i = 1; i <= n; ++i) if (i != S) st[0].insert(i);
  while (q.size()) {
    int u = q.front();
    q.pop();
    for (int v : e[u]) if (st[0].count(v)) {
      st[0].erase(v);
      st[1].insert(v);
    }
    for (auto v : st[0]) if (dis[v] == -1) {
      dis[v] = dis[u]+1;
      q.push(v);
    }
    swap(st[0], st[1]);
    st[1].clear();
  }
}
\end{lstlisting}

\section{ 浅谈图模型上的随机游走问题}

\subsection{网格图}

$$f(i)=
\begin{cases}
p_1f(i-1,j)+p_2f(i,j-1)+p_3f(i+1,j)+p_4f(i,j+1)+1,i^2+j^2\leq R \\\\
0,i^2+j^2<R
\end{cases}$$

\subsubsection{高斯消元 $O(R^6)$}

\subsubsection{直接消元法 $O(R^4)$}

\subsubsection{主元法 $O(R^3)$}

\section{ 树分治}
\subsection{点分治}
按树的重心分治
{\setmainfont{Consolas}
\begin{lstlisting}
namespace DFZ {
int vis[N], siz[N], mxs[N], cnt[K];
vector<int> pset;

void calc_size(int u, int fa = 0) { // 找重心
  siz[u] = 1; mxs[u] = 0;
  pset.emplace_back(u);
  for (auto p : e[u]) {
    int v = p.first;
    if (v == fa || vis[v]) continue;
    calc_size(v, u);
    siz[u] += siz[v];
    mxs[u] = max(mxs[u], siz[v]);
  }
}

void query(int u, int fa, int dis) {
  for (int i = 1; i <= m; ++i) if (q[i] >= dis) ans[i] |= cnt[q[i]-dis];
  for (auto p : e[u]) {
    int v = p.first;
    if (v == fa || vis[v]) continue;
    query(v, u, dis+p.second);
  }
}

void update(int u, int fa, int dis, int k) {
  if (dis < K) cnt[dis] += k;
  for (auto p : e[u]) {
    int v = p.first;
    if (v == fa || vis[v]) continue;
    update(v, u, dis+p.second, k);
  }
}

void dfz(int u = 1) {
  pset.clear();
  calc_size(u);
  for (int v : pset) { // get centre
    mxs[v] = max(mxs[v], (int)pset.size()-siz[v]);
    if (mxs[v] < mxs[u]) u = v;
  }
  cnt[0] = 1;
  for (auto p : e[u]) {
    int v = p.first;
    if (vis[v]) continue;
    query(v, u, p.second);
    update(v, u, p.second, 1);
  }
  update(u, 0, 0, -1); // clear
  vis[u] = 1;
  for (auto p : e[u]) {
    int v = p.first;
    if (vis[v]) continue;
    dfz(v);
  }
}
} // namespace 点分治
\end{lstlisting}
\subsection{边分治}


\part{数论}
\section{ 快排}
{\setmainfont{Consolas}
\begin{lstlisting}
void quick_sort(int l, int r) {
  if(l >= r) return;
  swap(a[l], a[l+rand()%(r-l)]);
  int i = l, j = r, mid = a[l];
  while(i < j) {
    while(i < j && a[j] >= mid) --j;
    swap(a[i], a[j]);
    while(i < j && a[i] < mid) ++i;
    swap(a[i], a[j]);
  }
  quick_sort(l, i-1);
  quick_sort(i+1, r);
}
\end{lstlisting}


\section{ 求第K大数}
{\setmainfont{Consolas}
\begin{lstlisting}
nth_element(a+1, a+k+1, a+n+1);
\end{lstlisting}


\section{ 求逆序对(归并排序)}
{\setmainfont{Consolas}
\begin{lstlisting}
void merge_sort(int l, int r) {
  if(l == r) return;
  int mid = (l+r)>>1;
  merge_sort(l, mid);
  merge_sort(mid+1, r);
  int i = l, j = mid+1, k = l;
  while(k <= r) {
    if(j <= r && (i > mid || a[j] < a[i])) {
      ans += mid-i+1;
      b[k++] = a[j++];
    }
    else b[k++] = a[i++];
  }
  memcpy(a+l, b+l, sizeof(int)*(r-l+1));
}
\end{lstlisting}


\section{ 线性基}
求最大值Luogu3812

求第k大数HDOJ3949
{\setmainfont{Consolas}
\begin{lstlisting}
template <typename T>
struct LinearBase {
  int sz = sizeof(T)*8, zero;
  T tot;
  vector<T> b, rb, p;
  LinearBase(){ init(); }
  void init() {
    tot = zero = 0;;
    vector<T>(sz, 0).swap(b);
    vector<T>().swap(rb);
    vector<T>().swap(p);
  }
  template <typename TT>
  void build(TT a[], const int &n) {
    init();
    for (int i = 1; i <= n; ++i) insert(a[i]);
  }
  void merge(const LinearBase xj) {
    for (int i : xj.b) if (i) insert(i);
  }
  void insert(T x) {
    for (int i = sz-1; i >= 0; --i) if ((x>>i)&1) {
      if (!b[i]) { b[i] = x; return; }
      x ^= b[i];
    }
    zero = 1;
  }
  bool find(T x) {
    for (int i = sz-1; i >= 0; --i) if ((x>>i)&1) {
      if (!b[i]) { return false; }
      x ^= b[i];
    }
    return true;
  }
  T max_xor() {
    T res = 0;
    for (int i = sz-1; i >= 0; --i)
      if (~(res>>i)&1) res ^= b[i];
      // res = max(res, res^b[i]);
    return res;
  }
  T min_xor() {
    if (zero) return 0;
    for (int i = 0; i < sz; ++i)
      if (b[i]) return b[i];
  }
  void rebuild() {
    rb = b;
    vector<T>().swap(p);
    for (int i = sz-1; i >= 0; --i)
      for (int j = i-1; j >= 0; --j)
        if ((rb[i]>>j)&1) rb[i] ^= rb[j];
    for (int i = 0; i < sz; ++i)
      if (rb[i]) p.emplace_back(rb[i]);
    tot = ((T)1<<p.size())+zero;
  }
  T kth_min(T k) {
    if (k >= tot || k < 1) return -1;
    if (zero && k == 1) return 0;
    if (zero) --k;
    T res = 0;
    for (int i = (int)p.size()-1; i >= 0; --i)
      if ((k>>i)&1) res ^= p[i];
    return res;
  }
  T kth_max(const T &k) {
    return kth_min(tot-k);
  }
};
\end{lstlisting}
前缀和线性基
vector跑贼鸡儿慢
{\setmainfont{Consolas}
\begin{lstlisting}
template <class T>
struct PreSumLB {
  int tot, sz = sizeof(T)*8;
  vector<T> b[N];
  vector<int> p[N];
  PreSumLB() { init(); }
  void init() {
    tot = 0;
    vector<T>(sz, 0).swap(b[0]);
    vector<int>(sz, 0).swap(p[0]);
  }
  void append(T val) {
    int pos = ++tot;
    vector<T> &bb = b[tot];
    vector<int> &pp = p[tot];
    pp = p[tot-1];
    bb = b[tot-1];
    for (int i = sz-1; i >= 0; --i) if ((val>>i)&1) {
      if (bb[i]) {
        if (pos > pp[i]) swap(pos, pp[i]), swap(val, bb[i]);
        val ^= bb[i];
      } else {
        bb[i] = val;
        pp[i] = pos;
        return;
      }
    }
  }
  T query(const int &l, const int &r) {
    T res = 0;
    vector<T> &bb = b[r];
    vector<int> &pp = p[r];
    for (int i = sz-1; i >= 0; --i)
      if (pp[i] >= l) res = max(res, res^bb[i]);
    return res;
  }
};
\end{lstlisting}


\section{ 矩阵}
\subsection{矩阵快速幂}
\subsection{矩阵求逆}
{\setmainfont{Consolas}
\begin{lstlisting}
template <typename T>
struct Martix {
  int n, m;
  T a[N][N];
  Martix(){}
  Martix(const int &_n) : n(_n), m(_n) { init(); }
  Martix(const int &_n, const int &_m) : n(_n), m(_m) { init(); }
  T* operator [] (const int &i) { return a[i]; }
  void init(const int &tag = 0) {
    for (int i = 1; i <= n; ++i) memset(a[i], 0, sizeof(T)*(m+1));
    if (tag) for (int i = 1; i <= n; ++i) a[i][i] = tag;
  }
  friend Martix operator * (const Martix &m1, const Martix &m2) {
    Martix res(m1.n, m2.m);
    for (int i = 1; i <= res.n; ++i)
      for (int j = 1; j <= res.m; ++j)
        for (int k = 1; k <= m1.m; ++k)
          res.a[i][j] = (res.a[i][j]+m1.a[i][k]*m2.a[k][j])%MOD;
    return res;
  }
  Martix& operator *= (const Martix &mx) { return *this = *this*mx; }
  Martix& operator + (const Martix &mx) const { Martix res(n, m); return res += mx; }
  Martix& operator += (const Martix &mx) {
    assert(n == mx.n && m == mx.m);
    for (int i = 1; i <= n; ++i)
      for (int j = 1; j <= m; ++j)
        a[i][j] += mx.a[i][j];
    return *this;
  }
  Martix& operator - (const Martix &mx) const { Martix res(n, m); return res -= mx; }
  Martix& operator -= (const Martix &mx) {
    assert(n == mx.n && m == mx.m);
    for (int i = 1; i <= n; ++i)
      for (int j = 1; j <= m; ++j)
        a[i][j] -= mx.a[i][j];
    return *this;
  }
  template <typename TT>
  Martix pow(const TT &p) const {
    Martix res(n, m), a = *this;
    res.init(1);
    for (TT i = p; i; i >>= 1, a *= a) if (i&1) res *= a;
    return res;
  }
  Martix inv() const {
    Martix res = *this;
    vector<int> is(n+1), js(n+1);
    for (int k = 1; k <= n; ++k) {
      for (int i = k; i <= n; ++i)
        for (int j = k; j <= n; ++j) if (res.a[i][j]) {
          is[k] = i; js[k] = j; break;
        }
      for (int i = 1; i <= n; ++i) swap(res.a[k][i], res.a[is[k]][i]);
      for (int i = 1; i <= n; ++i) swap(res.a[i][k], res.a[i][js[k]]);
      if (!res.a[k][k]) return Martix(0);
      res.a[k][k] = mul_inverse(res.a[k][k]); // get inv of number
      for (int j = 1; j <= n; ++j) if (j != k)
        res.a[k][j] = res.a[k][j]*res.a[k][k]%MOD;
      for (int i = 1; i <= n; ++i) if (i != k)
        for (int j = 1; j <= n; ++j) if (j != k)
          res.a[i][j] = (res.a[i][j]+MOD-res.a[i][k]*res.a[k][j]%MOD)%MOD;
      for (int i = 1; i <= n; ++i) if (i != k)
        res.a[i][k] = (MOD-res.a[i][k]*res.a[k][k]%MOD)%MOD;
    }
    for (int k = n; k; --k) {
      for (int i = 1; i <= n; ++i) swap(res.a[js[k]][i], res.a[k][i]);
      for (int i = 1; i <= n; ++i) swap(res.a[i][is[k]], res.a[i][k]);
    }
    return res;
  }
  T det() {
    long long res = 1;
    Martix cpy = *this;
    for (int i = 1; i <= n; ++i) {
      for (int j = i+1; j <= n; ++j) while (cpy.a[j][i]) {
        long long t = cpy.a[i][i]/cpy.a[j][i];
        for (int k = i; k <= n; ++k)
          cpy.a[i][k] = (cpy.a[i][k]+MOD-t*cpy.a[j][k]%MOD)%MOD;
        swap(cpy.a[i], cpy.a[j]);
        res = -res;
      }
      res = res*cpy.a[i][i]%MOD;
    }
    return (res+MOD)%MOD;
  }
  friend ostream& operator << (ostream &os, const Martix<T> &mx) {
    for (int i = 1; i <= mx.n; ++i)
      for (int j = 1; j <= mx.m; ++j)
        os << mx.a[i][j] << " \n"[j==mx.m];
    return os;
  }
};
\end{lstlisting}
\subsection{伍德伯里矩阵恒等式|Woodbury matrix identity}

解决矩阵修改求逆问题 hdoj6994

$(A+UCV)^{-1}=A^{-1}-A^{-1}U(C^{-1}+VA^{-1}U)^{-1}VA^{-1}$

$A\in R^{n\times n},U\in R^{n\times k},C\in R^{k\times k},V\in R^{k\times n}$

矩阵求逆 $O(n^3)$, 单次修改 $O(n^2)$

例如给矩阵 $A$ 的第 $i$ 行第 $j$ 列增加 $\Delta$ , 若 $A_{3*3}, (i,j)=(2,3)$

$$UCV=
\begin{bmatrix}
{0}&{0}&{0}\\\\
{0}&{0}&{\Delta}\\\\
{0}&{0}&{0}
\end{bmatrix}$$

则有

$U_{n\times 1}=\begin{bmatrix}{0}&{1}&{0}\end{bmatrix}^T,U_i=1$

$C_{1\times 1}=\begin{bmatrix}{\Delta}\end{bmatrix}$

$V_{1\times n}=\begin{bmatrix}{0}&{0}&{1}\end{bmatrix},V_j=1$

注意运算顺序$(A^{-1}U)(C^{-1}+VA^{-1}U)^{-1}(VA^{-1})$


\section{ 高斯消元}
{\setmainfont{Consolas}
\begin{lstlisting}
struct GaussElimination {
  double a[N][N];
  void init() { memset(a, 0, sizeof a); }
  void init(const int &n) {
    for (int i = 1; i <= n; ++i)
      for (int j = 1; j <= n+1; ++j)
        a[i][j] = 0;
  }
  // ans is a[i][n+1]
  bool solve(const int &n) {
    for (int i = 1, j, k; i <= n; ++i) {
      for (j = i+1, k = i; j <= n; ++j)
        if (abs(a[j][i]) > abs(a[k][i])) k = j;
      if (abs(a[k][i]) < eps) return false;
      swap(a[k], a[i]);
      for (j = 1; j <= n; ++j) if (i != j) {
        double d = a[j][i]/a[i][i];
        for (k = i+1; k <= n+1; ++k)
          a[j][k] -= d*a[i][k];
      }
    }
    for (int i = 1; i <= n; ++i) a[i][n+1] /= a[i][i];
    return true;
  }
};
\end{lstlisting}
\subsection{异或方程组}
a[i][j] 第i个是否对j有影响

a[i][n+1] 第i个最后被翻转与否
{\setmainfont{Consolas}
\begin{lstlisting}
// -1 : no solution, 0 : multi , 1 : one
template <typename T>
int XorGauss(T a[N], const int &n) {
  for (int i = 1, j, k; i <= n; ++i) {
    for (k = i; !a[k][i] && k <= n; ++k) {}
    if (k <= n) swap(a[k], a[i]);
    for (j = 1; j <= n; ++j) if (i != j && a[j][i])
      for (k = i; k <= n+1; ++k) a[j][k] ^= a[i][k];
      // a[j] ^= a[i]; // bitset<N> a[N]
  }
  for (int i = 1; i <= n; ++i) if (!a[i][i] && a[i][n+1]) return -1;
  for (int i = 1; i <= n; ++i) if (!a[i][i]) return 0;
  return 1;
}
// dfs(n, 0)
void dfs(const int &u, const int &num) {
  if (num >= res) return;
  if (u <= 0) { res = num; return; }
  if (a[u][u]) {
    int t = a[u][n+1];
    for (int i = u+1; i <= n; ++i) {
      if (a[u][i]) t ^= used[i];
    }
    dfs(u-1, num+t);
  } else { // 自由元
    dfs(u-1, num);
    used[u] = 1;
    dfs(u-1, num+1);
    used[u] = 0;
  }
}
\end{lstlisting}


\section{ 拉格朗日插值}
{\setmainfont{Consolas}
\begin{lstlisting}
template <typename T, typename H, typename P>
long long Largrange(const T &k, const int &n, const H x[], const P y[]) {
  k %= MOD;
  long long res = 0, s1 = 1, s2 = 1;
  for (int i = 1; i <= n; ++i, s1 = s2 = 1) {
    for (int j = 1; j <= n; ++j) if (i != j) {
      s1 = s1*(x[i]-x[j]+MOD)%MOD;
      s2 = s2*(k-x[j]+MOD)%MOD;
    }
    res = (res+y[i]*s2%MOD*mul_inverse(s1)%MOD)%MOD;
  }
  return res;
}
\end{lstlisting}
{\setmainfont{Consolas}
\begin{lstlisting}
template <typename T, typename P> // x[i] = i -> y[i] = f(i)
long long Largrange(const T &k, const int &n, const P y[]) {
  if (k <= n) return y[k];
  static long long pre[N], suf[N];
  long long res = 0;
  k %= MOD;
  pre[0] = suf[n+1] = 1;
  for (int i = 1; i <= n; ++i) pre[i] = pre[i-1]*(k-i)%MOD;
  for (int i = n; i >= 1; --i) suf[i] = suf[i+1]*(k-i)%MOD;
  for (int i = 1; i <= n; ++i) {
    res = (res+y[i]*(pre[i-1]*suf[i+1]%MOD)%MOD
      *mul_inverse(((n-i)&1 ? -1 : 1)*fac[i-1]*fac[n-i]%MOD)%MOD)%MOD;
  }
  return (res+MOD)%MOD;
}
\end{lstlisting}


\section{ 快速幂}
{\setmainfont{Consolas}
\begin{lstlisting}
template <typename T, typename H>
inline T qpow(const T &a, const H &p, const int &mo = MOD) {
  long long res = 1, x = a;
  for (H i = p; i; i >>= 1, x = x*x%mo)
    if (i&1) res = res*x%mo;
  return static_cast<T>(res);
}
\end{lstlisting}


\section{ 快速乘}
{\setmainfont{Consolas}
\begin{lstlisting}
inline long long qmul(long long x, long long y, long long mo) {
  // x*y - floor(x*y/mo)*mo
  typedef unsigned long long ull;
  typedef long double ld;
  return ((ull)x*y-(ull)((ld)x/mo*y)*mo+mo)%mo;
}
\end{lstlisting}


\section{ 复数}
{\setmainfont{Consolas}
\begin{lstlisting}
struct comp {
  typedef double T; // maybe long double ?
  T real, imag;
  comp (const double &_real = 0, const double &_imag = 0) : real(_real), imag(_imag) {}
  friend comp operator + (const comp &c1, const comp &c2) { return comp(c1.real+c2.real, c1.imag+c2.imag); }
  friend comp operator - (const comp &c1, const comp &c2) { return comp(c1.real-c2.real, c1.imag-c2.imag); }
  friend comp operator * (const comp &c1, const comp &c2) { return comp(c1.real*c2.real-c1.imag*c2.imag, c1.real*c2.imag+c1.imag*c2.real); }
  comp& operator += (const comp &c) { return *this = *this+c; }
  comp& operator -= (const comp &c) { return *this = *this-c; }
  comp& operator *= (const comp &c) { return *this = *this*c; }
  friend istream& operator >> (istream &is, comp &c) { return is >> c.real >> c.imag; }
  friend ostream& operator << (ostream &os, comp &c) { return os << c.real << setiosflags(ios::showpos) << c.imag << "i";}
  comp conjugate() { return comp(real, -imag); }
  friend comp conjugate(const comp &c) { return comp(c.real, -c.imag); }
};
\end{lstlisting}


\section{ 快速傅里叶变换|FFT}
{\setmainfont{Consolas}
\begin{lstlisting}
namespace FFT { // array [0, n)
const int SIZE = (1<<18)+3;
int len, bit;
int rev[SIZE];
// #define comp complex<long double>
void fft(comp a[], int flag = 1) {
  for (int i = 0; i < len; ++i)
    if (i < rev[i]) swap(a[i], a[rev[i]]);
  for (int base = 1; base < len; base <<= 1) {
    comp w, wn = {cos(PI/base), flag*sin(PI/base)};
    for (int i = 0; i < len; i += base*2) {
      w = { 1.0, 0.0 };
      for (int j = 0; j < base; ++j) {
        comp x = a[i+j], y = w*a[i+j+base];
        a[i+j] = x+y;
        a[i+j+base] = x-y;
        w *= wn;
      }
    }
  }
}
void work(comp f[], const int &n, comp g[], const int &m) {
  len = 1; bit = 0;
  while (len < n+m) len <<= 1, ++bit;
  // multi-testcase
  for (int i = n; i < len; ++i) f[i] = 0;
  for (int i = m; i < len; ++i) g[i] = 0;
  for (int i = 0; i < len; ++i)
    rev[i] = (rev[i>>1]>>1)|((i&1)<<(bit-1));
  fft(f, 1); fft(g, 1);
  for (int i = 0; i < len; ++i) f[i] *= g[i];
  fft(f, -1);
  for (int i = 0; i < n+m; ++i) f[i].real /= len;
}
} // namespace FFT
\end{lstlisting}


\section{ 快速数论变换|NTT}
{\setmainfont{Consolas}
\begin{lstlisting}
namespace NTT { // array [0, n)
const int SIZE = (1<<18)+3;
const int G = 3;
int len, bit;
int rev[SIZE];
long long f[SIZE], g[SIZE];
template <class T>
void ntt(T a[], int flag = 1) {
  for (int i = 0; i < len; ++i)
    if (i < rev[i]) swap(a[i], a[rev[i]]);
  for (int base = 1; base < len; base <<= 1) {
    long long wn = qpow(G, (MOD-1)/(base*2)), w;
    if (flag == -1) wn = qpow(wn, MOD-2);
    for (int i = 0; i < len; i += base*2) {
      w = 1;
      for (int j = 0; j < base; ++j) {
        long long x = a[i+j], y = w*a[i+j+base]%MOD;
        a[i+j] = (x+y)%MOD;
        a[i+j+base] = (x-y+MOD)%MOD;
        w = w*wn%MOD;
      }
    }
  }
}
template <class T>
void work(T a[], const int &n, T b[], const int &m) {
  len = 1; bit = 0;
  while (len < n+m) len <<= 1, ++bit;
  for (int i = 0; i < n; ++i) f[i] = a[i];
  for (int i = n; i < len; ++i) f[i] = 0;
  for (int i = 0; i < m; ++i) g[i] = b[i];
  for (int i = m; i < len; ++i) g[i] = 0;
  for (int i = 0; i < len; ++i)
    rev[i] = (rev[i>>1]>>1)|((i&1)<<(bit-1));
  ntt(f, 1); ntt(g, 1);
  for (int i = 0; i < len; ++i) f[i] = f[i]*g[i]%MOD;
  ntt(f, -1);
  long long inv = qpow(len, MOD-2);
  for (int i = 0; i < n+m-1; ++i) f[i] = f[i]*inv%MOD;
}
} // namespace NTT
\end{lstlisting}
\section{ 任意模数NTT|MTT}
{\setmainfont{Consolas}
\begin{lstlisting}
namespace MTT {
const int SIZE = (1<<18)+7;
const int Mod = MOD;
comp w[SIZE];
int bitrev[SIZE];
long long f[SIZE];
void fft(comp *a, const int &n) {
  for (int i = 0; i < n; ++i) if (i < bitrev[i]) swap(a[i], a[bitrev[i]]);
  for (int i = 2, lyc = n >> 1; i <= n; i <<= 1, lyc >>= 1)
    for (int j = 0; j < n; j += i) {
      comp *l = a + j, *r = a + j + (i >> 1), *p = w;
      for (int k = 0; k < i>>1; ++k) {
        comp tmp = *r * *p;
        *r = *l - tmp, *l = *l + tmp;
        ++l, ++r, p += lyc;
      }
    }
}
template <class T>
inline void work(T *x, const int &n, T *y, const int &m) {
  static int bit, L;
  static comp a[SIZE], b[SIZE];
  static comp dfta[SIZE], dftb[SIZE];

  for (L = 1, bit = 0; L < n+m-1; ++bit, L <<= 1);
  for (int i = 0; i < L; ++i) bitrev[i] = bitrev[i >> 1] >> 1 | ((i & 1) << (bit - 1));
  for (int i = 0; i < L; ++i) w[i] = comp(cos(2 * PI * i / L), sin(2 * PI * i / L));

  for (int i = 0; i < n; ++i) (x[i] += Mod) %= Mod, a[i] = comp(x[i] & 32767, x[i] >> 15);
  for (int i = n; i < L; ++i) a[i] = 0;
  for (int i = 0; i < m; ++i) (y[i] += Mod) %= Mod, b[i] = comp(y[i] & 32767, y[i] >> 15);
  for (int i = m; i < L; ++i) b[i] = 0;
  fft(a, L), fft(b, L);
  for (int i = 0; i < L; ++i) {
    int j = (L - i) & (L - 1);
    static comp da, db, dc, dd;
    da = (a[i] + conjugate(a[j])) * comp(.5, 0);
    db = (a[i] - conjugate(a[j])) * comp(0, -.5);
    dc = (b[i] + conjugate(b[j])) * comp(.5, 0);
    dd = (b[i] - conjugate(b[j])) * comp(0, -.5);
    dfta[j] = da*dc + da*dd*comp(0, 1);
    dftb[j] = db*dc + db*dd*comp(0, 1);
  }
  for (int i = 0; i < L; ++i) a[i] = dfta[i];
  for (int i = 0; i < L; ++i) b[i] = dftb[i];
  fft(a, L), fft(b, L);
  for (int i = 0; i < L; ++i) {
    int da = (long long)(a[i].real / L + 0.5) % Mod;
    int db = (long long)(a[i].imag / L + 0.5) % Mod;
    int dc = (long long)(b[i].real / L + 0.5) % Mod;
    int dd = (long long)(b[i].imag / L + 0.5) % Mod;
    f[i] = (da + ((long long)(db + dc) << 15) + ((long long)dd << 30)) % Mod;
  }
  for (int i = 0; i < n+m-1; ++i) (f[i] += Mod) %= Mod;
}
} // namespace MTT

\end{lstlisting}
\section{ 分治FFT}
{\setmainfont{Consolas}
\begin{lstlisting}
// give g[1, n) ask f[0, n)
// f[i] = sigma f[i-j]*g[j] (1 <= j <= i)
template <class T> // [l, r]
void cdq_fft(T f[], T g[], const int &l, const int &r) {
  if (r-l <= 1) return;
  int mid = (l+r)>>1;
  cdq_fft(f, g, l, mid);
  NTT::work(f+l, mid-l, g, r-l);
  for (int i = mid; i < r; ++i)
    (f[i] += NTT::f[i-l]) %= MOD;
  cdq_fft(f, g, mid, r);
}
// f[0] = 1; cdq_fft(f, g, 0, n);
\end{lstlisting}

\section{ 快速沃尔什变换|FWT}

推导详解

公式参考

洛谷例题

复杂度 $O(n\log n) | O(n2^n)$

$FWT(A\pm B)=FWT(A)\pm FWT(B)$

$FWT(cA)=cFWT(A)$

定义 $\bigoplus$ 为任意集合运算

$FWT(A\bigoplus B)=FWT(A)\times FWT(B)$

求 $C_i = \sum\limits_{i=j\bigoplus k}{a_j b_k}$

\subsection{或运算}
$FWT(A)[i] = \sum\limits_{j|i=i}{A[j]}$

$FWT(A) = [FWT(A_0),FWT(A_0+A_1)]$

$IFWT(A) = [IFWT(A_0),IFWT(A_1)-IFWT(A_0)]$


\subsection{与运算}
$FWT(A)[i] = \sum\limits_{i\&j=j}{A[j]}$

$FWT(A) = [FWT(A_0+A_1),FWT(A_1)]$

$IFWT(A) = [IFWT(A_0)-IFWT(A_1),IFWT(A_1)]$

\subsection{异或运算}
令 $d(x)$ 为 $x$ 在二进制下拥有的1的数量

$FWT(A)[i] = \sum\limits_{j}(-1)^{d(i\&j)}A[j]$

$FWT(A) = [FWT(A_0+A_1),FWT(A_0-A_1)]$

$IFWT(A) = [\frac{IFWT(A_1-A_0)}{2},\frac{IFWT(A_1+A_0)}{2}]$

\subsection{code}

{\setmainfont{Consolas}
\begin{lstlisting}
namespace FWT {
#define forforfor for (int l = 2; l <= len; l <<= 1)\
                  for (int i = 0, k = l>>1; i < len; i += l)\
                  for (int j = 0; j < k; ++j)

  const int SIZE = (1<<17)+3;
  int len;
  int f[SIZE], g[SIZE];
  template <class T> void init(T a[], const int &n, T b[], const int &m) {
    len = 1;
    while (len < max(n, m)) len <<= 1;
    for (int i = 0; i < n; ++i) f[i] = a[i];
    for (int i = n; i < len; ++i) f[i] = 0;
    for (int i = 0; i < m; ++i) g[i] = b[i];
    for (int i = m; i < len; ++i) g[i] = 0;
  }
  template <class T> void fwt_or(T a[], const int x = 1) {
    forforfor a[i+j+k] = (a[i+j+k]+1ll*a[i+j]*x)%MOD;
  }
  template <class T> void fwt_and(T a[], const int x = 1) {
    forforfor a[i+j] = (a[i+j]+1ll*a[i+j+k]*x)%MOD;
  }
  template <class T> void fwt_xor(T a[], const int x = 1) {
    forforfor {
      (a[i+j] += a[i+j+k]) %= MOD;
      a[i+j+k] = (a[i+j]-2*a[i+j+k]%MOD+MOD)%MOD;
      a[i+j] = 1ll*a[i+j]*x%MOD; a[i+j+k] = 1ll*a[i+j+k]*x%MOD;
    }
  }
  template <class T> void work_or(const T a[], const int &n, const T b[], const int &m) {
    init(a, n, b, m); fwt_or(f); fwt_or(g);
    for (int i = 0; i < len; ++i) f[i] = 1ll*f[i]*g[i]%MOD;
    fwt_or(f, MOD-1); // fwt_or(x, -1)
  }
  template <class T> void work_and(const T a[], const int &n, const T b[], const int &m) {
    init(a, n, b, m); fwt_and(f); fwt_and(g);
    for (int i = 0; i < len; ++i) f[i] = 1ll*f[i]*g[i]%MOD;
    fwt_and(f, MOD-1); // fwt_and(x, -1)
  }
  template <class T> void work_xor(const T a[], const int &n, const T b[], const int &m) {
    init(a, n, b, m); fwt_xor(f); fwt_xor(g);
    for (int i = 0; i < len; ++i) f[i] = 1ll*f[i]*g[i]%MOD;
    fwt_xor(f, mul_inverse(2)); // fwt_xor(x, 1/2)
  }
#undef forforfor
} // namespace FWT
\end{lstlisting}

\section{ 快速子集变换(子集卷积)|FST}

$C_k = \sum\limits_{i\&j=0,i|j=k}{A_i B_j}$

复杂度 $O(n\log^2 n) | O(n^22^n)$

{\setmainfont{Consolas}
\begin{lstlisting}
namespace FST {
  const int W = 20;
  const int N = 1<<W;
  int len, bit;
  int f[W+1][N], g[W+1][N], h[W+1][N], res[N];
  template <class T> void fwt(T a[], const int x = 1) {
    for (int l = 2; l <= len; l <<= 1)
    for (int i = 0, k = l>>1; i < len; i += l)
    for (int j = 0; j < k; ++j)
      a[i+j+k] = (a[i+j+k]+1ll*a[i+j]*x)%MOD;
  }
  template <class T> void work(const T a[], const int &n, const T b[], const int &m) {
    len = 1; bit = 0;
    while (len < max(n, m)) len <<= 1, ++bit;
    for (int i = 0; i <= bit; ++i)
      for (int j = 0; j < len; ++j)
        f[i][j] = g[i][j] = h[i][j] = 0;
    for (int i = 0; i < n; ++i) f[__builtin_popcount(i)][i] = a[i];
    for (int i = 0; i < m; ++i) g[__builtin_popcount(i)][i] = b[i];
    for (int i = 0; i <= bit; ++i) {
      fwt(f[i]); fwt(g[i]);
      for (int j = 0; j <= i; ++j)
        for (int k = 0; k < len; ++k)
          h[i][k] = (h[i][k]+1ll*f[j][k]*g[i-j][k])%MOD;
      fwt(h[i], MOD-1); // fwt(h[i], -1)
    }
    for (int i = 0; i < len; ++i) res[i] = h[__builtin_popcount(i)][i];
  }
} // namespace FST
\end{lstlisting}

\subsection{倍增子集卷积}

设多项式 $A = \sum\limits_{i=0}^{2^n-1}{a_i x^i},B=\sum\limits_{i=0}^{2^n-1}{b_i x^i}$

求 $C = A*B = \sum\limits_{i=0}^{2^n-1}{x^i \sum\limits_{d\subseteq i}{a_d b_{i-d} }}$

按照每个状态的最高位进行分组,然后卷 $n$ 次

复杂度 $O(\sum\limits_{i=1}^{n}{i^2 2^i}) = O(n^2 2^n)$

{\setmainfont{Consolas}
\begin{lstlisting}
template <typename T> void vip_fst(T a[], const int &n) { // return a
  static int b[1<<B]; // warning: the type of b
  int len = 1; while (len < n) len <<= 1;
  memcpy(b, a, sizeof(T)*len);
  memset(a, 0, sizeof(T)*len); a[0] = 1;
  for (int i = 1; i < len; i <<= 1) {
    FST::work(a, i, b+i, i);
    for (int j = 0; j < i; ++j)
      a[i+j] = FST::h[__builtin_popcount(j)][j];
  }
}
\end{lstlisting}


\section{ 第二类斯特林数}
{\setmainfont{Consolas}
\begin{lstlisting}
inline void stirling(const int &n) {
  S[0][0] = 1;
  for (int i = 1; i <= n; ++i)
    for (int j = 1; j <= i; ++j)
      S[i][j] = S[i-1][j-1]+S[i-1][j]*j;
}
\end{lstlisting}
{\setmainfont{Consolas}
\begin{lstlisting}
void stirling(const int &n) {
  inv[0] = inv[1] = 1;
  for(int i = 2; i <= n; ++i)
    inv[i] = MOD-MOD/i*inv[MOD%i]%MOD;
  for (int i = 1; i <= n; ++i)
    inv[i] = inv[i-1]*inv[i]%MOD;
  while (len <= (n<<1)) len <<= 1, ++bit;
  for (int i = 0; i < len; ++i)
    rev[i] = (rev[i>>1]>>1)|((i&1)<<(bit-1));
  for (int i = 0, one = 1; i <= n; ++i, one = MOD-one) {
    f[i] = one*inv[i]%MOD;
    g[i] = qpow(i, n)*inv[i]%MOD;
  }
  NTT(f, 1); NTT(g, 1);
  for (int i = 0; i < len; ++i) f[i] = f[i]*g[i]%MOD;
  NTT(f, -1);
  long long invv = qpow(len, MOD-2);
  for (int i = 0; i <= n; ++i)
    printf("%lld%c", f[i]*invv%MOD, " \n"[i==n]);
}
\end{lstlisting}


\section{ 约瑟夫环}
\subsection{O(n)}
{\setmainfont{Consolas}
\begin{lstlisting}
int solve(int n, int v) { return n == 1 ? 0 : (solve(n-1, v)+v)%n; }
// res = solve(num, step)+1
\end{lstlisting}


\section{ 最小公倍数 lcm}
$LCM(\frac{a}{b},\frac{c}{d})=\frac{LCM(a, c)}{GCD(b,d)}$

$LCM(\frac{a_1}{b_1},\frac{a_2}{b_2},...)=\frac{LCM(a1, a2,...)}{GCD(b1, b2,...)}$
\section{ 扩展欧几里得(同余方程)}
{\setmainfont{Consolas}
\begin{lstlisting}
template <typename T>
T exgcd(const T a, const T b, T &x, T &y) {
  if (!b) return x = 1, y = 0, a;
  T d = exgcd(b, a%b, y, x);
  y -= a/b*x;
  return d;
}
\end{lstlisting}


\section{ 乘法逆元}
\subsection{拓展欧几里得}
{\setmainfont{Consolas}
\begin{lstlisting}
template <typename T>
inline T mul_inverse(const T &a, const T &mo = MOD) {
  T x, y;
  exgcd(a, mo, x, y);
  return (x%mo+mo)%mo;
}
\end{lstlisting}
\subsection{线性递推}
{\setmainfont{Consolas}
\begin{lstlisting}
template <typename T>
inline void mul_inverse(T *inv, int mod = MOD) {
  inv[0] = inv[1] = 1;
  for(int i = 2; i <= n; ++i)
    inv[i] = 1ll*(mod-mod/i)*inv[mod%i]%mod;
}
\end{lstlisting}


\section{ 中国剩余定理}
\subsection{中国剩余定理CRT(m互质)}
{\setmainfont{Consolas}
\begin{lstlisting}
inline long long CRT(int a[], int m[]) {
  long long res = 0, M = 1;
  for (int i = 1; i <= n; ++i)
    M *= m[i];
  for (int i = 1; i <= n; ++i)
    res = (res + a[i]*(M/m[i])*mul_inverse(M/m[i], m[i]))%M;
  return (res+M)%M;
}
\end{lstlisting}
\subsection{扩展中国剩余定理EXCRT(m不互质)}
{\setmainfont{Consolas}
\begin{lstlisting}
inline long long EXCRT(long long a[], long long m[]) {
  // M*x + m[i]*y = a[i]-res (mod m[i])
  // res = res+x*M;
  long long M = m[1], res = a[1], x, y, c, d;
  for (int i = 2; i <= n; ++i) {
    d = exgcd(M, m[i], x, y);
    c = (a[i]-res%m[i]+m[i])%m[i];
    if (c%d != 0) return -1;
    x = (c/d)*x%(m[i]/d);
    res += x*M;
    M *= m[i]/d;
    res = (res%M+M)%M;
  }
  return res;
}
\end{lstlisting}


\section{ 排列组合}

\subsection{奇偶性}
C(n,k) 当 \`n\&k == k\` 为奇数反之偶数


\section{ 欧拉函数}
{\setmainfont{Consolas}
\begin{lstlisting}
template <typename T> inline T phi(T x) {
  T res = x;
  for (T i = 2; i*i <= x; ++i) {
    if (x%i) continue;
    res = res/i*(i-1);
    while (x%i == 0) x /= i;
  }
  if (x > 1) res = res/x*(x-1);
  return res;
}
\end{lstlisting}
\subsection{筛法}
{\setmainfont{Consolas}
\begin{lstlisting}
struct Euler {
  int phi[N], check[N];
  vector<int> prime;
  void init(int sz) {
    for (int i = 1; i <= sz; ++i) check[i] = 1;
    phi[1] = 1; check[1] = 0;
    for (int i = 2; i <= sz; ++i) {
      if (check[i]) {
        prime.emplace_back(i);
        phi[i] = i-1;
      }
      for (int j : prime) {
        if (i*j > sz) break;
        check[i*j] = 0;
        if (i%j) {
          phi[i*j] = (j-1)*phi[i];
        } else {
          phi[i*j] = j*phi[i];
          break;
        }
      }
    }
  }
} E;
\end{lstlisting}

\subsection{欧拉定理}

a 与 m 互质时,$a^{\phi(m)}  \equiv 1 \mod m$ 

\subsection{扩展欧拉定理}

无需 a,m 互质 $b > \phi(m),a^b \equiv a^{(b \mod \phi(m))+\phi(m)} \mod m$

\section{ 莫比乌斯函数}

{\setmainfont{Consolas}
\begin{lstlisting}
template <typename T> inline T miu(T x) {
  int t = 0;
  for (T i = 2, k; i*i <= x; ++i) {
    if (x%i) continue;
    for (k = 0, ++t; x %i == 0; x /= i, ++k) {}
    if (k >= 2) return 0;
  }
  if (x > 1) ++t;
  return t&1 ? -1 : 1;
}
\end{lstlisting}


\section{ 线性筛素数}
{\setmainfont{Consolas}
\begin{lstlisting}
struct Euler {
  vector<int> prime;
  vector<bool> check;
  bool operator  { return check[i]; }
  void init(int n) {
    prime.clear();
    check = vector<bool>(n+1, true);
    check[1] = false;
    for (int i = 2; i <= n; ++i) {
      if (check[i]) prime.emplace_back(i);
      for (const int &j : prime) {
        if (i*j > n) break;
        check[i*j] = false;
        if (i%j == 0) break;
      }
    }
  }
} E;
\end{lstlisting}


\section{ 判断素数(质数)}
\subsection{Miller-Rabin 素性测试}
{\setmainfont{Consolas}
\begin{lstlisting}
inline bool MillerRabin(int x) {
  static const int test_time = 10;
  if (x < 3) return x == 2;
  int a = x-1, b = 0;
  while (!(a&1)) a >>= 1, ++b;
  for (int i = 1, j, v; i <= test_time; ++i) {
    v = (qpow(rnd()%(x-2)+2, a, x));
    if (v == 1 || v == x-1) continue;
    for (j = 0; j < b && v != x-1; ++j)
      v = static_cast<int>(1ll*v*v%x);
    if (j >= b) return false;
  }
  return true;
}
\end{lstlisting}

\section{ 线性筛GCD}
{\setmainfont{Consolas}
\begin{lstlisting}
inline void gcd_init(const int &n) {
  for (int i = 1; i <= n; ++i)
  for (int j = 1; j <= n; ++j) if (!g[i][j])
  for (int k = 1; k <= n/i; ++k)
    g[k*i][k*j] = k;
}
\end{lstlisting}


\section{ BSGS}

求解关于 $t$ 的方程 $a^t \equiv x(mod m),\gcd(a, m) = 1$

{\setmainfont{Consolas}
\begin{lstlisting}
inline long long BSGS(long long a, long long x, long long m) { // a^n = x
  static map<long long, int> mmp; mmp.clear();
  long long t = sqrt(m)+1, b = 1, c = 1, res;
  for(int i = 0; i < t; ++i, b = b*a%m) mmp[x*b%m] = i;
  for(int i = 0; i <= t; ++i, c = c*b%m) { // b = a^t
    if(mmp.count(c)) {
      res = t*i-mmp[c];
      if(res > 0) return res;
    }
  }
  return -1;
}
\end{lstlisting}

\section{ 拓展BSGS}

$\gcd(a, m) \neq 1$

{\setmainfont{Consolas}
\begin{lstlisting}
namespace EXBSGS {

inline ll BSGS(ll a, ll b, ll mod, ll qaq){
  unordered_map<ll, int> H; H.clear();
  ll Q, p = ceil(sqrt(mod)), x, y; 
  exgcd(qaq, mod, x, y), b = (b * x % mod + mod) % mod, 
  Q = qpow(a, p, mod), exgcd(Q, mod, x, y), Q = (x % mod + mod) % mod ;
  for (ll i = 1, j = 0 ; j <= p ; ++ j, i = i * a % mod)  if (!H.count(i)) H[i] = j ;
  for (ll i = b, j = 0 ; j <= p ; ++ j, i = i * Q % mod)  if (H[i]) return j * p + H[i];
  return -1 ;
}

inline ll exBSGS(ll N, ll P, ll M){
  ll qaq = 1, k = 0, qwq = 1; 
  if (M == 1) return 0 ; 
  while ((qwq = __gcd(N, P)) > 1) {
    if (M%qwq) return -1 ;
    ++k, M /= qwq, P /= qwq, qaq = qaq*(N/qwq)%P ;
    if (qaq == M) return k ;
  }
  return (qwq = BSGS(N, M, P, qaq)) == -1 ? -1 : qwq+k ;
}

} using EXBSGS::exBSGS;
\end{lstlisting}

\section{ 错排}

$D_1 = 0$

$D_2 = 1$

$D_n = (n-1)(D_{n-1}+D_{n-2})$

\section{ 原根}

复杂度 $O(\sqrt{m}+g\times\log^2m)$

{\setmainfont{Consolas}
\begin{lstlisting}
inline int getG(const int &m) {
  static int q[100000+7];
  int _phi = phi(m), x = _phi, tot = 0;
  for (int i = 2; i*i <= _phi; ++i) {
    if (x%i) continue;
    q[++tot] = _phi/i;
    while (x%i == 0) x /= i;
  }
  if (x > 1) x = q[++tot] = _phi/x;
  for (int g = 2, flag; ; ++g) {
    flag = 1;
    if (qpow(g, _phi, m) != 1) continue;
    for (int i = 1; i <= tot; ++i) {
      if (qpow(g, q[i], m) == 1) {
        flag = 0;
        break;
      }
    }
    if (flag) return g;
  }
}
\end{lstlisting}

\subsection{单位根反演}

$[k|n]=\frac{1}{k}\sum\limits_{i=0}^{k-1}w_{k}^{ni}$

$[a\equiv b(\mod n)]=[a-b \equiv 0(\mod n)]=\frac{1}{n}\sum\limits_{i=0}^{n-1}w_n^{(a-b)k}=\frac{1}{n}\sum\limits_{i=0}^{n-1}w_n^{ak}w_n^{-bk}$

\subsection{单位根卷积}

$\sum\limits_{i=0}^{n}[i\%k=0]f(i)=\sum\limits_{i=0}^{n}\frac{1}{k}\sum\limits_{j=0}^{k-1}(w_k^i)^jf(i)$

\section{ 全排列和逆序对}

\subsection{根据逆序数推排列数}

已知一个n元排列的逆序数为m,这样的n元排列有多少种?

1. 对任意n>=2且0<=m<=C(n,2)时f(n,m)>=1;当m>C(n,2)时,f(n,m)=0

2. f(n,m)=f(n,C(n,2)-m)

3. f(n+1,m)=f(n,m)+f(n,m-1)+…+f(n,m-n)

4. f(n,0)=f(n,C(n,2))=1

5. f(n,1)=f(n,C(n,2)-1)=n-1(n>1)

6. f(n,2)=f(n,C(n,2)-2)=C(n,2)-1(n>2)

\subsection{根据每个数的逆序数求出原排列}

\subsection{根据逆序数求最小排列}

1. 对于n的全排列,在它完全倒序的时候(也就是n,n-1,…,2,1的时候)逆序数最多。

2. 对于一个形如1,2,3,…,i-1,i,n,…i+1的排列q(如n=5时的1,2,5,4,3),即在数n前保证首项为1且严格以公差为1递增而数n之后排列任意的数列

  - 当数n之后是递减的时候q的逆序数最多,为t=C(n-i,2)。 

  - 排列q是出现逆序数为t的最小排列。 

3. 在上一条所设定的排列q的基础上,我们将数n后面的第k小数与数n的前一个数(即i)交换,然后使数n后面保持逆序。这样得到的新排列所含的逆序数为t=C(n-i,2)+k,且这个排列是逆序数为t的最小排列。 

\subsection{第k个字典序每个数的逆序对}

{\setmainfont{Consolas}
\begin{lstlisting}
// n个数排列的第k个(字典序)的逆序对
int f(int k) {
  int res = 0;
  for (int j = n; j; --j) {
    res += k/fac[j];
    k %= fac[j];
  }
  return res;
}
\end{lstlisting}

\section{ 二次剩余}
{\setmainfont{Consolas}
\begin{lstlisting}
namespace Cipolla {

int mod, _x0, _x1;
long long I_mul_I; // 虚数单位的平方

struct complex {
  long long real, imag;
  complex(long long real = 0, long long imag = 0): real(real), imag(imag) { }
};
inline bool operator == (complex x, complex y) {
  return x.real == y.real and x.imag == y.imag;
}
inline complex operator * (complex x, complex y) {
  return complex((x.real * y.real + I_mul_I * x.imag % mod * y.imag) % mod,
      (x.imag * y.real + x.real * y.imag) % mod);
}

complex power(complex x, int k) {
  complex res = 1;
  while(k) {
    if(k & 1) res = res * x;
    x = x * x;
    k >>= 1;
  }
  return res;
}

bool check_if_residue(int x) {
  return power(x, (mod - 1) >> 1) == 1;
}

int solve(int n, int p, int &x0 = _x0, int &x1 = _x1) {
  mod = p;
  if (power(n, p>>1) == p-1) return x0 = -1; // 无解
  if (power(n, p>>1) == 0) return x0 = x1 = 0;

  long long a = rand() % mod;
  while(!a or check_if_residue((a * a + mod - n) % mod))
    a = rand() % mod;
  I_mul_I = (a * a + mod - n) % mod;

  x0 = int(power(complex(a, 1), (mod + 1) >> 1).real);
  x1 = mod - x0;
  return x0;
}
} // namespace Cipolla
\end{lstlisting}



\part{动态规划 DP}
\section{ 线性DP}
\subsection{最长公共子序列LCS}
{\setmainfont{Consolas}
\begin{lstlisting}
f[i][j] = max{  f[i-1][j],
                f[i][j-1],
                f[i-1][j-1]+1 (if A[i] == B[j])}
\end{lstlisting}
\section{ 状压DP}
\subsection{枚举子集}
{\setmainfont{Consolas}
\begin{lstlisting}
for (int i = s; i; i = (i-1)&s) {}
\end{lstlisting}
\subsection{枚举n个元素,大小为k的二进制子集}
{\setmainfont{Consolas}
\begin{lstlisting}
int s=(1<<k)-1;
while(s<(1<<n)){
  work(s);
  int x=s&-s,y=s+x;
  s=((s&~y)/x>>1)|y; //这里有一个位反~
}
\end{lstlisting}

\section{ 背包问题}
\subsection{多重背包}
二进制拆分
{\setmainfont{Consolas}
\begin{lstlisting}
for(int i = 1, cnt, vi, wi, m; i <= n; ++i) {
  scanf("%d%d%d", &vi, &wi, &m);
  cnt = 1;
  while(m-cnt > 0) {
    m -= cnt;
    v.push_back(vi*cnt);
    w.push_back(wi*cnt);
    cnt <<= 1;
  }
  v.push_back(vi*m);
  w.push_back(wi*m);
}
for(int i = 0; i < w.size(); ++i)
  for(int j = W; j >= w[i]; --j)
    b[j] = max(b[j], b[j-w[i]]+v[i]);
\end{lstlisting}
单调队列
{\setmainfont{Consolas}
\begin{lstlisting}
for(int i = 1; i <= n; ++i) {
  scanf("%d%d%d", &v, &w, &m);
  for(int u = 0; u < w; ++u) {
    int maxp = (W-u)/w;
    head = 1; tail = 0;
    for(int k = maxp-1; k >= max(0, maxp-m); --k) {
      while(head <= tail && calc(u, q[tail]) <= calc(u, k)) tail--;
      q[++tail] = k;
    }
    for(int p = maxp; p >= 0; --p) {
      while(head <= tail && q[head] >= p) head++;
      if(head <= tail) f[u+p*w] = max(f[u+p*w], p*v+calc(u, q[head]));
      if(p-m-1 < 0) continue;
      while(head <= tail && calc(u, q[tail]) <= calc(u, p-m-1)) tail--;
      q[++tail] = p-m-1;
    }
  }
}
int ans = 0;
for(int i = 1; i <= W; ++i)
  ans = max(ans, f[i]);
\end{lstlisting}


\section{ SOS DP}
\section{ WQS二分|DP凸优化}
题目给了一个选物品的限制条件,要求刚好选m个,让你最大化(最小化)权值, 其特点是函数的斜率单调

例:给你一个N个点M条边无向带权连通图,每条边是黑色或白色。让你求一棵最小权的恰好有K条白色边的生成树。

记 $g(i)$ 是选了 $i$ 条白边的最小生成树值, 发现 $g(i)$ 斜率单调不增 $g(i)-g(i-1) \leq g(i+1)-g(i)$

则二分斜率 k, 求切点(截距最大)

设$f(x)$为我在没有固定选多少个点(但是我已经选了x个点)时的答案(也就是截距), $f(x)=g(x)-k*x$

只要把每个数的$h(x)−=k$然后正常求一下在选任意个数的情况下最大$f(x)$是多少 $O(n\log n)$

\section{ 斜率优化}

若dp方程为 $dp[i]=a[i] \cdot b[j]+c[i]+d[j]$ 时,由于存在$a[i] \cdot b[j]$ 这个既有 $i$ 又有 $j$ 的项,就需要使用斜率优化

\subsection{「HNOI2008」玩具装箱 TOY}

$dp[i]=min(dp[j]+(sum[i]+i−sum[j]−j−L−1)^2)(j<i)$

令$a[i]=sum[i]+i,b[i]=sum[i]+i+L+1$

$dp[i]=dp[j]+(a[i]-b[j])^2$

$dp [ i ] = dp [ j ] + a [ i ] ^ 2 - 2 a [ i ] b [ j ] + b [ j ] ^ 2$

$2 a [ i ] b [ j ] + dp [ i ] - a [ i ] ^ 2 = dp [ j ] + b [ j ] ^ 2$

将 $b[j]$ 看作 $x,dp[j]+b[j]^2$ 看作 $y$,这个式子就可以看作一条斜率为 $2 a[i]$ 的直线

而对于每个 $i$ 来说, $a[i]$ 都是确定的, 类似线性规划

$dp[i]$ 的含义转化为:当上述直线过点 $P(b[j],dp[j]+b[j]^2)$ 时,直线在 $y$ 轴的截距加上 $a[i]^2$ (一个定值) 而题目即为找这个截距的最小值

\section{ 四边形不等式}

\subsection{2D1D}

$f_{l,r}=\min\limits_{k=l}^{r-1} \{f_{l,k}+f_{k+1,r}\}+w(l,r) \ \ (1\leq l \leq r \leq n)$

当 $w(l,r)$ 满足特定性质

- 区间包含单调性 :如果对于任意 $l\leq l' \leq r' \leq r$ ,均有 $w(l',r')\leq w(l,r)$ 成立,则称函数 $w$ 对于区间包含关系具有单调性。
  
- 四边形不等式 :如果对于任意 $l_1 \leq l_2 \leq r_1 \leq r_2$ ,均有 $w(l_1,r_1)+w(l_2,r_2) \leq w(l_1,r_2)+w(l_2,r_1)$ 成立,则称函数 $w$ 满足四边形不等式(简记为“交叉小于包含”)。若等号永远成立,则称函数 $w$ 满足 四边形恒等式 。

> 引理 1 :若满足关于区间包含的单调性的函数 $w(l,r)$ 满足四边形不等式,则状态 $f_{l,r}$ 也满足四边形不等式。

> 定理 1 :若状态 $f$ 满足四边形不等式,记 $m_{l,r}=\min\{k:f_{l,r}=g_{k,l,r}\}$ 表示最优决策点,则有 $m_{l,r-1} \leq m_{l,r} \leq m_{l+1,r}$

\subsection{1D1D}

$f_r = \min\limits_{l=1}^{r-1} \{f_l+w(l,r)\} \ \ (1\leq r \leq n)$

> 定理 2 :若函数 $w(l,r)$ 满足四边形不等式,记 $h_{l,r}=f_l+w(l,r)$ 表示从 $l$ 转移过来的状态 $r$ , $k_r=\min\{l|f_r=h_{l,r}\}$ 表示最优决策点,则有 $\forall r_1 \leq r_2 : k_{r1} \leq k_{r2}$

{\setmainfont{Consolas}
\begin{lstlisting}
void DP(int l, int r, int k_l, int k_r) {
  int mid = (l + r) / 2, k = k_l;
  // 求状态f[mid]的最优决策点
  for (int i = k_l; i <= min(k_r, mid - 1); ++i)
    if (w(i, mid) < w(k, mid)) k = i;
  f[mid] = w(k, mid);
  // 根据决策单调性得出左右两部分的决策区间,递归处理
  if (l < mid) DP(l, mid - 1, k_l, k);
  if (r > mid) DP(mid + 1, r, k, k_r);
}
\end{lstlisting}

\subsection{满足四边形不等式的函数类}

- 性质 1 :若函数 $w_1(l,r),w_2(l,r)$ 均满足四边形不等式(或区间包含单调性),则对于任意 $c_1,c_2\geq 0$ ,函数 $c_1w_1+c_2w_2$ 也满足四边形不等式(或区间包含单调性)。

- 性质 2 :若存在函数 $f(x),g(x)$ 使得 $w(l,r) = f(r)-g(l)$ ,则函数 $w$ 满足四边形恒等式。当函数 $f,g$ 单调增加时,函数 $w$ 还满足区间包含单调性。

- 性质 3 :设 $h(x)$ 是一个单调增加的凸函数,若函数 $w(l,r)$ 满足四边形不等式并且对区间包含关系具有单调性,则复合函数 $h(w(l,r))$ 也满足四边形不等式和区间包含单调性。

- 性质 4 :设 $h(x)$ 是一个凸函数,若函数 $w(l,r)$ 满足四边形恒等式并且对区间包含关系具有单调性,则复合函数 $h(w(l,r))$ 也满足四边形不等式。

首先需要澄清一点,凸函数(Convex Function)的定义在国内教材中有分歧,此处的凸函数指的是(可微的)下凸函数,即一阶导数单调增加的函数。

\section{ 插头DP|轮廓线DP}
\subsection{一个闭合回路}
{\setmainfont{Consolas}
\begin{lstlisting}
const int P = 299987;
const int M = 1<<21;
const int N = 15;

int n, m;
int a[N][N];
long long dp[2][M];
int head[2][P], nex[2][M], tot[2], ver[2][M];
// long long dp[2][P];
// int head[2][P], nex[2][P], tot[2], ver[2][P];

inline void clear(const int &u) {
  for (int i = 1; i <= tot[u]; ++i) {
    dp[u][i] = 0; //
    nex[u][i] = 0; //
    head[u][ver[u][i]%P] = 0;
  }
  tot[u] = 0;
}

template <typename T, typename U>
inline void insert(const int &u, const T &x, const U &v) {
  int p = x%P;
  for (int i = head[u][p]; i; i = nex[u][i]) {
    if (ver[u][i] == x) return dp[u][i] += v, void();
  }
  ++tot[u]; assert(tot[u] < M);
  ver[u][tot[u]] = x;
  nex[u][tot[u]] = head[u][p];
  head[u][p] = tot[u];
  dp[u][tot[u]] = v;
}

template <typename T>
inline int get_val(const int &u, const T &x) {
  int p = x%P;
  for (int i = head[u][p]; i; i = nex[u][i]) {
    if (ver[u][i] == x) return dp[u][i];
  }
  return 0;
}

inline long long solve() {
  int u = 0, base = (1<<m*2+2)-1;
  long long res = 0;
  clear(u);
  insert(u, 0, 1);
  for (int i = 1; i <= n; ++i) {
    for (int j = 1; j <= m; ++j) {
      clear(u ^= 1);
      for (int k = 1; k <= tot[u^1]; ++k) {
        int state = ver[u^1][k];
        long long val = dp[u^1][k];
        if (j == 1) state = (state<<2)&base;
        // b1 right b2 down
        // 0 no 1 left 2 right
        int b1 = (state>>j*2-2)%4, b2 = (state>>j*2)%4;
        if (!a[i][j]) {
          if (!b1 && !b2) insert(u, state, val);
        } else if (!b1 && !b2) {
          if (a[i+1][j] && a[i][j+1]) insert(u, state+(1<<j*2-2)+(2<<j*2), val);
        } else if (!b1 && b2) {
          if (a[i][j+1]) insert(u, state, val);
          if (a[i+1][j]) insert(u, state+(b2<<j*2-2)-(b2<<j*2), val);
        } else if (b1 && !b2) {
          if (a[i+1][j]) insert(u, state, val);
          if (a[i][j+1]) insert(u, state-(b1<<j*2-2)+(b1<<j*2), val);
        } else if (b1 == 1 && b2 == 1) { // find 2 turn to 1
          for (int k = j+1, t = 1; k <= m; ++k) {
            if ((state>>k*2)%4 == 1) ++t;
            if ((state>>k*2)%4 == 2) --t;
            if (!t) { insert(u, state-(1<<j*2-2)-(1<<j*2)-(1<<k*2), val); break; }
          }
        } else if (b1 == 2 && b2 == 2) { // find 1 turn to 2
          for (int k = j-2, t = 1; k >= 0; --k) {
            if ((state>>k*2)%4 == 1) --t;
            if ((state>>k*2)%4 == 2) ++t;
            if (!t) { insert(u, state-(2<<j*2-2)-(2<<j*2)+(1<<k*2), val); break; }
          }
        } else if (b1 == 2 && b2 == 1) {
          insert(u, state-(2<<j*2-2)-(1<<j*2), val);
        } else if (i == ex && j == ey) { // b1 == 1, b2 == 2
          res += val;
        }
      }
    }
  }
  return res;
}
\end{lstlisting}

\subsection{多个闭合回路}
{\setmainfont{Consolas}
\begin{lstlisting}
        else if (b1 == 1 && b2 == 2) {
          if (i == ex && j == ey) res += val;
          else dp[u][bit-(1<<j*2-2)-(1<<j*2+1)] += val;
        }
\end{lstlisting}

\subsection{联通块}

{\setmainfont{Consolas}
\begin{lstlisting}
int n, u, res = -INF;
int a[N][N];
unordered_map<int, int> dp[2];

inline void decode(const int &state, int *const s) {
  for (int i = 1; i <= n; ++i) s[i] = (state>>i*3-3)%8;
}

inline void insert(const int *const s, const int &val) {
  static int vis[N];
  int state = 0, cnt = 0;
  memset(vis, 0, sizeof vis);
  for (int i = 1; i <= n; ++i) {
    if (s[i] && !vis[s[i]]) vis[s[i]] = ++cnt;
    state |= (vis[s[i]]<<i*3-3);
  }
  if (dp[u].count(state)) dp[u][state] = max(dp[u][state], val);
  else dp[u].insert({state, val});
  if (cnt == 1) res = max(res, val);
}

inline void solve() {
  static int s[N];
  dp[u = 0].clear();
  dp[u][0] = 0;
  for (int i = 1; i <= n; ++i) {
    for (int j = 1; j <= n; ++j) {
      dp[u ^= 1].clear();
      for (const auto &p : dp[u^1]) {
        decode(p.first, s);
        int b1 = s[j-1], b2 = s[j];
        // not choose
        s[j] = 0;
        int cnt = 0;
        for (int k = 1; k <= n; ++k) cnt += s[k] == b2;
        if (!b2 || cnt) insert(s, p.second);
        s[j] = b2;
        // choose
        if (!b1 && !b2) {
          s[j] = 7;
        } else {
          if (b1 > b2) swap(b1, b2); // in case b2 == 0
          s[j] = b2;
          if (b1) for (int k = 1; k <= n; ++k) if (s[k] == b1) s[k] = b2;
        }
        insert(s, p.second+a[i][j]);
      }
    }
  }
  cout << res << endl;
}
\end{lstlisting}

\subsection{L型}

L 型地板:拐弯且仅拐弯一次。

发现没有,一个存在的插头只有两种状态:拐弯过和没拐弯过,因此我们这样定义插头:

0表没有插头,1表没拐过的插头,2表已经拐过的插头。b1代表当前点的右插头,b2代表当前点的下插头

\section{ DP套DP}

有 $dp_1, f[i]$ 为 $dp_1[n] = i$ 的方案数,求 $f$

设 $dp_2[dp_1]$ 为 $dp_1$ 状态下的方案数

\section{ 动态DP}

将 dp 转换为线段树可以求解的区间问题,动态维护

\subsection{动态线性DP}

\subsection{动态树形DP}

树链剖分,轻链暴力


\part{分数}
{\setmainfont{Consolas}
\begin{lstlisting}
template <class T>
struct Fraction {
  T a, b;
  void normalize() {
    if (a == 0) return (void)(b = 1);
    T g = __gcd(a, b);
    a /= g; b /= g;
    if (b < 0) a = -a, b = -b;
  }
  Fraction(const T &_a = 0, const T &_b = 1) : a(_a), b(_b) { normalize(); }
  friend bool operator < (const Fraction &f1, const Fraction &f2) { return f1.a*f2.b < f2.a*f1.b; }
  friend bool operator > (const Fraction &f1, const Fraction &f2) { return f1.a*f2.b > f2.a*f1.b; }
  friend bool operator == (const Fraction &f1, const Fraction &f2) { return f1.a*f2.b == f2.a*f1.b; }
  friend bool operator != (const Fraction &f1, const Fraction &f2) { return f1.a*f2.b != f2.a*f1.b; }
  friend bool operator <= (const Fraction &f1, const Fraction &f2) { return f1 < f2 || f1 == f2; }
  friend bool operator >= (const Fraction &f1, const Fraction &f2) { return f1 > f2 || f1 == f2; }
  friend Fraction operator + (const Fraction &f1, const Fraction &f2) { return Fraction(f1.a*f2.b+f2.a*f1.b, f1.b*f2.b); }
  friend Fraction operator - (const Fraction &f1, const Fraction &f2) { return Fraction(f1.a*f2.b-f2.a*f1.b, f1.b*f2.b); }
  friend Fraction operator * (const Fraction &f1, const Fraction &f2) { return Fraction(f1.a*f2.a, f1.b*f2.b); }
  friend Fraction operator / (const Fraction &f1, const Fraction &f2) { return Fraction(f1.a*f2.b, f1.b*f2.a); }
  Fraction& operator += (const Fraction &f) { return *this = *this+f; }
  Fraction& operator -= (const Fraction &f) { return *this = *this-f; }
  Fraction& operator *= (const Fraction &f) { return *this = *this*f; }
  Fraction& operator /= (const Fraction &f) { return *this = *this/f; }
  friend istream& operator >> (istream &is, Fraction &f) {
    is >> f.a >> f.b;
    f.normalize();
    return is;
  }
  friend ostream& operator << (ostream &os, const Fraction &f) {
    if (f.b == 1) return os << f.a;
    return os << f.a << "/" << f.b;
  }
};
using fraction = Fraction<long long>;
\end{lstlisting}

\end{document}