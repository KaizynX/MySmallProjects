\documentclass[UTF8]{ctexart}
    \usepackage{ctex}
\usepackage{listings}
\usepackage{xcolor}
\usepackage{geometry}
\usepackage{setspace}
\usepackage{fancyhdr}
\pagestyle{fancy}
\fancyfoot[C]{\thepage} % 页脚
\geometry{left=2.2cm,right=1.8cm,top=2.0cm,bottom=1.5cm} % 页边距
\renewcommand{\baselinestretch}{0.8} % 行间距
% 定义可能使用到的颜色
\definecolor{CPPLight}  {HTML} {686868}
\definecolor{CPPSteel}  {HTML} {888888}
\definecolor{CPPDark}   {HTML} {262626}
\definecolor{CPPBlue}   {HTML} {4172A3}
\definecolor{CPPGreen}  {HTML} {487818}
\definecolor{CPPBrown}  {HTML} {A07040}
\definecolor{CPPRed}    {HTML} {AD4D3A}
\definecolor{CPPViolet} {HTML} {7040A0}
\definecolor{CPPGray}  {HTML} {B8B8B8}
\lstset{
    columns=fixed,       
    numbers=left,                                        % 在左侧显示行号
    breaklines=true,                                     %代码过长则换行
    % frame=shadowbox,                                      %用方框框住代码块
    frame=single,                                      %用方框框住代码块
    % backgroundcolor=\color[RGB]{245,245,244},            % 设定背景颜色
    keywordstyle=\color[RGB]{40,40,255},                 % 设定关键字颜色
    numberstyle=\footnotesize\color{darkgray},           % 设定行号格式
    commentstyle=\it\color[RGB]{0,96,96},                % 设置代码注释的格式
    stringstyle=\rmfamily\slshape\color[RGB]{128,0,0},   % 设置字符串格式
    showstringspaces=false,                              % 不显示字符串中的空格
    language=c++,                                        % 设置语言
    morekeywords={alignas,continute,friend,register,true,alignof,decltype,goto,
    reinterpret_cast,try,asm,defult,if,return,typedef,auto,delete,inline,short,
    typeid,bool,do,int,signed,typename,break,double,long,sizeof,union,case,
    dynamic_cast,mutable,static,unsigned,catch,else,namespace,static_assert,using,
    char,enum,new,static_cast,virtual,char16_t,char32_t,explict,noexcept,struct,
    void,export,nullptr,switch,volatile,class,extern,operator,template,wchar_t,
    const,false,private,this,while,constexpr,float,protected,thread_local,
    const_cast,for,public,throw,std},
    emph={map,set,multimap,multiset,unordered_map,unordered_set,
    unordered_multiset,unordered_multimap,vector,string,list,deque,
    array,stack,forwared_list,iostream,memory,shared_ptr,unique_ptr,
    random,bitset,ostream,istream,cout,cin,endl,move,default_random_engine,
    uniform_int_distribution,iterator,algorithm,functional,bing,numeric,},
    emphstyle=\color{CPPViolet}, 
}

\begin{document}     
\tableofcontents
\newpage
\part{杂项}

\section{快读快写}
{\setmainfont{Consolas}
\begin{lstlisting}
template <typename T> inline void read(T &x) {
  int c; T tag = 1;
  while(!isdigit((c=getchar()))) if(c == '-') tag = -1;
  x = c-'0';
  while(isdigit((c=getchar()))) x = (x<<1)+(x<<3) + c-'0';
  x *= tag;
}
template <typename T> void write(T x) {
  if(x < 0) x = -x, putchar('-');
  if(x > 9) write(x/10);
  putchar(x%10+'0');
}
\end{lstlisting}
{\setmainfont{Consolas}
\begin{lstlisting}
ios::sync_with_stdio(false); cin.tie(NULL); cout.tie(NULL);
\end{lstlisting}

\section{正则表达式 }
{\setmainfont{Consolas}
\begin{lstlisting}
char str[];
scanf("%3s", str); // 读取长度为n的字符串
scanf("%[abc]", str); // 读取a,b,c,读到之外的立即停止
scanf("%[a-z0-9]", str); // 同上,读取小写字母和数字
scanf("%*[a-z]%s", str); // 过滤掉小写字母读取
scanf("%[^a-z]", str); // 读取小写字符外字符,^表示非 
\end{lstlisting}
\section{随机数 }
{\setmainfont{Consolas}
\begin{lstlisting}
#include <random>
// 范围 unsigned int
mt19937 rnd(time(NULL));
mt19937 rnd(chrono::high_resolution_clock::now().time_since_epoch().count());
cout << rnd() << endl;
\end{lstlisting}
{\setmainfont{Consolas}
\begin{lstlisting}
std::random_device rd;  //获取随机数种子
std::mt19937 gen(rd()); //Standard mersenne_twister_engine seeded with rd()
std::uniform_int_distribution<> dis(0, 9);
std::cout << dis(gen) << endl;
\end{lstlisting}


\section{计算log2 }
{\setmainfont{Consolas}
\begin{lstlisting}
// lg2[i] = lg2(i) +1
for(int i = 1; i <= n; ++i) lg2[i] = lg2[i>>1]+1;
// lg2[i] = (int)log2(i)
for(int i = 2; i <= n; ++i) lg2[i] = lg2[i>>1]+1;
\end{lstlisting}
\section{快速开根号|牛顿迭代法 }
{\setmainfont{Consolas}
\begin{lstlisting}
double sqrt(const double &a) {
  double x = a, y = .0;
  while (abs(x-y) > err) {
    y = x;
    x = .5*(x+a/x);
  }
  return x;
}
\end{lstlisting}

\section{i/k == j 的 k 的个数 }
{\setmainfont{Consolas}
\begin{lstlisting}
for (int i = 1; i <= n; ++i) {
  for (int j = 1, l, r; j <= n; ++j) {
    l = max(1, i/(j+1));
    while (l-1 >= 1 && i/(l-1) == j) --l;
    while (i/l > j) ++l;
    r = i/j;
    while (r+1 <= i && i/(r+1) == j) ++r;
    while (i/r < j) --r;
    if (r-l+1 != i/j-i/(j+1)) {
        cout << i << " " << j << endl;
    }
  }
}
\end{lstlisting}

\section{三分法 }
{\setmainfont{Consolas}
\begin{lstlisting}
while (l < r) {
  int mid = (l+r)>>1;
  if (f(mid) < f(mid+1)) r = mid;
  else l = mid+1;
}
\end{lstlisting}
{\setmainfont{Consolas}
\begin{lstlisting}
while(r-l>5){
	int ml=(l+l+r)/3;
	int mr=(l+r+r)/3;
	if(f(ml)<f(mr))r=mr;
	else l=ml;
}
for (int i = l; i <= r; ++i) res = min(res, f(i));
\end{lstlisting}
{\setmainfont{Consolas}
\begin{lstlisting}
while (r-l > 3) {
    int mid = (l+r)>>1;
    if (f(mid) < f(mid+1)) r = mid+1;
    else l = mid;
}
\end{lstlisting}


\part{计算几何}
\section{向量 坐标 直线 圆 (结构体) }
{\setmainfont{Consolas}
\begin{lstlisting}
struct Point {
  typedef double T;
  T x, y;
  int id;
  Point(){}
  Point(const T &_x, const T &_y, const int &_i = 0) : x(_x), y(_y), id(_i) {}
  friend Point operator + (const Point &p1, const Point &p2) {
    return Point(p1.x+p2.x, p1.y+p2.y, p1.id);
  }
  friend Point operator - (const Point &p1, const Point &p2) {
    return Point(p1.x-p2.x, p1.y-p2.y, p1.id);
  }
  friend Point operator - (const Point &p) {
    return Point(-p.x, -p.y, p.id);
  }
  friend T operator * (const Point &p1, const Point &p2) {
    return p1.x*p2.y-p1.y*p2.x;
  }
  template <typename TT>
  friend Point operator / (const Point &p, const TT &k) {
    return Point(p.x/k, p.y/k, p.id);
  }
  template <typename TT>
  friend Point operator * (const Point &p, const TT &k) {
    return Point(p.x*k, p.y*k, p.id);
  }
  Point operator += (const Point &p) { return *this = *this+p; }
  Point operator -= (const Point &p) { return *this = *this+p; }
  template <typename TT>
  Point operator *= (const TT &k) { return *this = *this*k; }
  template <typename TT>
  Point operator /= (const TT &k) { return *this = *this/k; }
  friend bool operator < (const Point &p1, const Point &p2) {
    return make_pair(p1.x, p1.y) < make_pair(p2.x, p2.y);
  }
  friend bool operator > (const Point &p1, const Point &p2) {
    return make_pair(p1.x, p1.y) > make_pair(p2.x, p2.y);
  }
  friend bool operator == (const Point &p1, const Point &p2) {
    return p1.x == p2.x && p1.y == p2.y;
  }
  friend bool operator != (const Point &p1, const Point &p2) {
    return p1.x != p2.x || p1.y != p2.y;
  }
  friend istream& operator >> (istream &is, Point &p) {
    return is >> p.x >> p.y;
  }
  friend ostream& operator << (ostream &os, Point &p) {
    return os << p.x << " " << p.y << " " << p.id << endl;
  }
  double length() { return sqrt(1.0*x*x+1.0*y*y); }
  friend double dis(const Point &p1, const Point &p2) { return (p2-p1).length(); }
  double dis(const Point &p) { return (p-*this).length(); }
  friend T dot(const Point &p1, const Point &p2) { return p1.x*p2.x+p1.y*p2.y; }
  T dot(const Point &p) { return x*p.x+y*p.y; }
  friend Point rotate_90_c(const Point &p) { return Point(p.y, -p.x, p.id); }
  Point rotate_90_c() { return Point(y, -x, id); }
  friend double atan(const Point &p) { return atan2(p.y, p.x); }
};

template <typename T = double>
struct Vec { // 三维向量
  T x, y, z;
  Vec(const T &_x = 0, const T &_y = 0, const T &_z = 0) : x(_x), y(_y), z(_z) {}
  double len() { return sqrt(1.0*x*x+1.0*y*y+1.0*z*z); }
  friend Vec operator +(const Vec &v1, const Vec &v2) { return Vec(v1.x+v2.x, v1.y+v2.y, v1.z+v2.z); }
  friend Vec operator -(const Vec &v1, const Vec &v2) { return Vec(v1.x-v2.x, v1.y-v2.y, v1.z-v2.z); }
  friend Vec operator *(const T &k, const Vec &v) { return Vec(k*v.x, k*v.y, k*v.z); }
  friend Vec operator *(const Vec &v, const T &k) { return k*v; }
  friend Vec operator *(const Vec &v1, const Vec &v2) {
    return Vec(
        v1.y*v2.z-v1.z*v2.y,
        v1.z*v2.x-v1.x*v2.z,
        v1.x*v2.y-v1.y*v2.x
    );
  }
  friend T dot(const Vec &v1, const Vec &v2) { return v1.x*v2.x+v1.y*v2.y+v1.z*v2.z; }
  T dot(const Vec &v) { return dot(*this, v); }
  Vec& operator +=(const Vec &v) { return *this = *this+v; }
  Vec& operator -=(const Vec &v) { return *this = *this-v; }
  Vec& operator *=(const T &k) { return *this = *this*k; }
  Vec& operator *=(const Vec &v) { return *this = *this*v; }
  friend istream& operator >>(istream &is, Vec &v) { return is >> v.x >> v.y >> v.z; }
};

inline bool polar_angle1(const Point &p1, const Point &p2) {
  double d1 = atan(p1), d2 = atan(p2); 
  return d1 == d2 ? p1 < p2 : d1 < d2;
}

inline bool polar_angle2(const Point &p1, const Point &p2) {
  auto tmp = p1*p2;
  return tmp == 0 ? p1 < p2 : tmp > 0;
}

inline long long S(const Point &p1, const Point &p2, const Point &p3) {
  return abs(p1.x*p2.y+p1.y*p3.x+p2.x*p3.y-p1.x*p3.y-p1.y*p2.x-p2.y*p3.x);
}

struct Line {
  Point p1, p2;
  Line(){}
  Line(const Point &_p1, const Point &_p2) : p1(_p1), p2(_p2) {}
  friend bool cross(const Line &l1, const Line &l2) {
    #define SJ1(x) max(l1.p1.x, l1.p2.x) < min(l2.p1.x, l2.p2.x) || \
             max(l2.p1.x, l2.p2.x) < min(l1.p1.x, l1.p2.x)
    if (SJ1(x) || SJ1(y)) return false;
    #undef SJ1
    #define SJ2(a, b, c, d) ((a-b)*(a-c))*((a-b)*(a-d)) <= 0
    return SJ2(l1.p1, l1.p2, l2.p1, l2.p2) &&
         SJ2(l2.p1, l2.p2, l1.p1, l1.p2);
    #undef SJ2
  }
  friend bool on_line(const Line &l, const Point &p) {
    return abs((l.p1-l.p2)*(l.p1-p)) < err;
  }
  friend Point cross_point(const Line &l1, const Line &l2) {
    Point v1 = l1.p2-l1.p1, v2 = l2.p2-l2.p1;
    if (abs(v1*v2) < err) return Point(0, 0); // no cross_point
    double t = (l2.p1-l1.p1)*v2/(v1*v2);
    return l1.p1+v1*t;
  }
};

struct Circular {
  Point o;
  double r;
  Circular(){}
  Circular(const Point &_o, const double &_r) : o(_o), r(_r) {}
  template <typename T>
  Circular(const T &_x, const T &_y, const double &_r) : o(Point(_x, _y)), r(_r) {}
  friend bool in_cir(const Circular &c, const Point &p) { return dis(c.o, p) <= c.r; }
  bool in_cir(const Point &p) { return dis(o, p) <= r; }
};

inline Circular get_cir(const Point &p1, const Point &p2, const Point &p3) {
  Circular res;
  res.o = cross_point(Line((p1+p2)/2, (p1+p2)/2+(p2-p1).rotate_90_c()),
            Line((p1+p3)/2, (p1+p3)/2+(p3-p1).rotate_90_c()));
  res.r = dis(res.o, p1);
  return res;
}
\end{lstlisting}


\section{二维凸包 }
{\setmainfont{Consolas}
\begin{lstlisting}
int n;
int stk[N], used[N], tp;
Point p[N];

inline void Andrew() {
  memset(used, 0, sizeof used);
  sort(p+1, p+n+1);
  tp = 0;
  stk[++tp] = 1;
  for (int i = 2; i <= n; ++i) {
    while (tp >= 2 && (p[stk[tp]]-p[stk[tp-1]])*(p[i]-p[stk[tp]]) <= 0)
      used[stk[tp--]] = 0;
    used[i] = 1;
    stk[++tp] = i;
  }
  int tmp = tp;
  for (int i = n-1; i; --i) {
    if (used[i]) continue;
    while (tp >= tmp && (p[stk[tp]]-p[stk[tp-1]])*(p[i]-p[stk[tp]]) <= 0)
      used[stk[tp--]] = 0;
    used[i] = 1;
    stk[++tp] = i;
  }
}
\end{lstlisting}


\section{平面最近点对 }
{\setmainfont{Consolas}
\begin{lstlisting}
Point a[N];
int n, ansa, ansb;
double mindist;

inline bool cmp_y(const Point &p1, const Point &p2) { return p1.y < p2.y; }

void upd_ans(const Point &p1, const Point &p2) {
  double dist = dis(p1, p2);
  if (dist < mindist) mindist = dist, ansa = p1.id, ansb = p2.id;
}

void rec(int l, int r) {
  if (r-l <= 3) {
    for (int i = l; i < r; ++i)
      for (int j = i+1; j <= r; ++j)
        upd_ans(a[i], a[j]);
    sort(a+l, a+r+1, cmp_y);
    return;
  }

  static Point t[N];
  int m = (l+r)>>1, midx = a[m].x;
  rec(l, m); rec(m+1, r);
  merge(a+l, a+m+1, a+m+1, a+r+1, t, cmp_y);
  copy(t, t+r-l+1, a+l);

  int tsz = 0;
  for (int i = l; i <= r; ++i)
    if (abs(a[i].x-midx) <= mindist) {
      for (int j = tsz; j && a[i].y-t[j].y < mindist; --j)
        upd_ans(a[i], t[j]);
      t[++tsz] = a[i];
    }
}

inline void mindist_pair() {
  sort(a+1, a+n+1);
  mindist = INF;
  rec(1, n);
}
\end{lstlisting}


\section{最小圆覆盖|随即增量法 }
{\setmainfont{Consolas}
\begin{lstlisting}
inline Circular RIA() {
  Circular cir;
  random_shuffle(a+1, a+n+1);
  for (int i = 1; i <= n; ++i) {
    if (cir.in_cir(a[i])) continue;
    cir = Circular(a[i], 0);
    for (int j = 1; j < i; ++j) {
      if (cir.in_cir(a[j])) continue;
      cir = Circular((a[i]+a[j])/2, dis(a[i], a[j])/2);
      for (int k = 1; k < j; ++k) {
        if (cir.in_cir(a[k])) continue;
        cir = get_cir(a[i], a[j], a[k]);
      }
    }
  }
  return cir;
}
\end{lstlisting}


\part{数据结构}
\section{堆 }
{\setmainfont{Consolas}
\begin{lstlisting}
struct Heap {
  static const int Maxn = 1e6+7;
  int sz, a[Maxn];
  Heap() { sz = 0; memset(a, 0, sizeof a); }
  inline bool cmp(int x, int y) { return x < y; } // 小根堆
  inline int size() { return sz; }
  inline bool empty() { return sz == 0; }
  inline int top() { return a[1]; }
  inline void push(int x) { a[++sz] = x; swift_up(sz); }
  inline void pop() { swap(a[1], a[sz--]); swift_down(1); }
  inline void swift_up(int p) {
    while(p > 1 && cmp(a[p], a[p>>1])) // a[p] < a[p<<1]
      swap(a[p], a[p>>1]), p >>= 1;
  }
  inline void swift_down(int p) {
    int l, r, s;
    while(true) {
      l = p<<1; r = p<<1|1;
      if(l > sz) break;
      if(r > sz || cmp(a[l], a[r])) s = l; // a[l] < a[r]
      else s = r;
      if(cmp(a[s], a[p])) // a[s] < a[p]
        swap(a[p], a[s]), p = s;
      else break;
    }
  }
};
\end{lstlisting}

\section{二叉查找树}
\section{平衡树}
\subsection{Splay }
{\setmainfont{Consolas}
\begin{lstlisting}
struct Splay {
  #define root e[0].ch[1]
  typedef int T;
  struct node {
    T v = 0;
    int ch[2] = { 0, 0 };
    int fa = 0, sum = 0, cnt = 0;
  } e[N];
  int n;
  void update(int x) { e[x].sum = e[e[x].ch[0]].sum+e[e[x].ch[1]].sum+e[x].cnt; }
  int identify(int x) { return x == e[e[x].fa].ch[1]; }
  void connect(int x,int f,int son) { e[x].fa = f; e[f].ch[son] = x; }
  void rotate(int x) {
    int y = e[x].fa,
      r = e[y].fa,
      rson = identify(y),
      yson = identify(x),
      b = e[x].ch[yson^1];
    connect(b, y, yson);
    connect(y, x, yson^1);
    connect(x, r, rson);
    update(y); update(x);
  }
  void splay(int at,int to) {
    to = e[to].fa;
    int up;
    while((up = e[at].fa) != to) {
      if(e[up].fa != to)
        rotate(identify(up) == identify(at) ? up : at);
      rotate(at);
    }
  }
  int add_point(T v, int fa) {
    ++n; e[n].v = v; e[n].fa = fa; e[n].sum = e[n].cnt = 1;
    return n;
  }
  int find(T v) {
    int now = root, last = 0;
    while (now && e[now].v != v)
      last = now, now = e[now].ch[v > e[now].v];
    splay((now ? now : last), root);
    return now;
  }
  void insert(T v) {
    if (!root) { root = add_point(v, root); return; }
    int now = root, last = 0;
    while (now && e[now].v != v)
      last = now, now = e[now].ch[v > e[now].v];
    if (now) ++e[now].cnt;
    else now = e[last].ch[v > e[last].v] = add_point(v, last);
    splay(now, root);
  }
  void erase(T v) {
    int del = find(v);
    if (!del) return;
    if (e[del].cnt > 1) {
      --e[del].cnt;
      --e[del].sum;
    } else if (!e[del].ch[0]) {
      root = e[del].ch[1];
      e[root].fa = 0;
    } else {
      int oldroot = root;
      splay(nex(e[del].ch[0], 1), root);
      connect(e[oldroot].ch[1], root, 1);
      update(root);
    }
  }
  int rank(T v) { return e[e[find(v)].ch[0]].sum+1; }
  T atrank(int x) {
    if (x > e[root].sum) return -INF;
    int now = root;
    while (true) {
      if (x <= e[e[now].ch[0]].sum) now = e[now].ch[0];
      else if ((x -= e[e[now].ch[0]].sum) <= e[now].cnt) break;
      else x -= e[now].cnt, now = e[now].ch[1];
    }
    splay(now, root);
    return e[now].v;
  }
  // small 0, big 1
  int nex(int x, int opt) { while (e[x].ch[opt]) x = e[x].ch[opt]; return x; }
  T lower(T v, int opt) {
    insert(v);
    T res = e[nex(e[root].ch[opt], opt^1)].v;
    erase(v);
    return res;
  }
  #undef root
};
\end{lstlisting}
区间反转 
{\setmainfont{Consolas}
\begin{lstlisting}
struct Splay {
  typedef int T;
  struct node {
    T v = 0;
    int ch[2] = { 0, 0 };
    int fa = 0, sum = 0, cnt = 0, tag = 0;
  } e[N];
  int sz, &root = e[0].ch[1];
  void update(int x) { e[x].sum = e[e[x].ch[0]].sum+e[e[x].ch[1]].sum+e[x].cnt; }
  int identify(int x) { return x == e[e[x].fa].ch[1]; }
  void connect(int x,int f,int son) { e[x].fa = f; e[f].ch[son] = x; }
  void rotate(int x) {
    int y = e[x].fa,
      r = e[y].fa,
      rson = identify(y),
      yson = identify(x),
      b = e[x].ch[yson^1];
    connect(b, y, yson);
    connect(y, x, yson^1);
    connect(x, r, rson);
    update(y); update(x);
  }
  void splay(int at,int to = 0) {
    to = e[to].fa;
    int up;
    while((up = e[at].fa) != to) {
      if(e[up].fa != to)
        rotate(identify(up) == identify(at) ? up : at);
      rotate(at);
    }
  }
  int add_point(T v, int fa) {
    ++sz; e[sz].v = v; e[sz].fa = fa; e[sz].sum = e[sz].cnt = 1;
    return sz;
  }
  int find(int x) {
    if (x > e[root].sum) return -INF;
    int now = root;
    while (true) {
      push_down(now);
      if (x <= e[e[now].ch[0]].sum) now = e[now].ch[0];
      else if ((x -= e[e[now].ch[0]].sum) <= e[now].cnt) break;
      else x -= e[now].cnt, now = e[now].ch[1];
    }
    return now;
  }
  int build(int l, int r, int fa) {
    if (l > r) return 0;
    int mid = (l+r)>>1,
      now = add_point(mid, fa);
    e[now].ch[0] = build(l, mid-1, now);
    e[now].ch[1] = build(mid+1, r, now);
    update(now);
    return now;
  }
  void push_down(int x) {
    if (x && e[x].tag) {
      e[e[x].ch[0]].tag ^= 1;
      e[e[x].ch[1]].tag ^= 1;
      swap(e[x].ch[0], e[x].ch[1]);
      e[x].tag = 0;
    }
  }
  void reverse(int l, int r) {
    int pl = find(l-1+1), pr = find(r+1+1);
    splay(pl); splay(pr, pl);
    e[e[e[root].ch[1]].ch[0]].tag ^= 1;
  }
  void print_LMR(int x) {
    if (!x) return;
    push_down(x);
    print_LMR(e[x].ch[0]);
    if (e[x].v != 0 && e[x].v != n+1)
      write(a[e[x].v]), putchar(' ');
    print_LMR(e[x].ch[1]);
  }
} tree;
\end{lstlisting}
\section{线段树}
\subsection{区间加减区间和 }
{\setmainfont{Consolas}
\begin{lstlisting}
template <typename T>
struct SegmentTree {
  int sz;
  T tr[N<<2], lazy[N<<2];
  SegmentTree(){}
  void build(const int &n, const T &k = 0) { sz = n; _build(1, n, k); }
  template <typename TT>
  void build(const TT a[], const int &n) { sz = n; _build(a, 1, n); }
  void modify(const int &x, const T &k) { _modify(x, k, 1, sz); }
  void add(const int &x, const T &k) { _add(x, x, k, 1, sz); }
  void add(int l, int r, const T &k) { if (l > r) swap(l, r); _add(l, r, k, 1, sz); }
  T query(const int &x) { return _query(x, x, 1, sz); }
  T query(int l, int r) { if (l > r) swap(l, r); return _query(l, r, 1, sz); }
private :
  void push_up(const int &i) { tr[i] = tr[i<<1]+tr[i<<1|1]; }
  void push_down(const int &i, const int &len) {
    if (!lazy[i]) return;
    tr[i<<1] += lazy[i]*(len-len/2);
    tr[i<<1|1] += lazy[i]*(len/2);
    lazy[i<<1] += lazy[i];
    lazy[i<<1|1] += lazy[i];
    lazy[i] = 0;
  }
  void _build(const int &l, const int &r, const T &k = 0, const int &i = 1) {
    lazy[i] = 0;
    if (l == r) { tr[i] = k; return; }
    int mid = (l+r)>>1;
    _build(l, mid, k, i<<1);
    _build(mid+1, r, k, i<<1|1);
    push_up(i);
  }
  template <typename TT>
  void _build(const TT a[], const int &l, const int &r, const int &i = 1) {
    lazy[i] = 0;
    if (l == r) { tr[i] = a[l]; return; }
    int mid = (l+r)>>1;
    _build(a, l, mid, i<<1);
    _build(a, mid+1, r, i<<1|1);
    push_up(i);
  }
  void _modify(const int &x, const T &k, const int &trl, const int &trr, const int &i = 1) {
    if (trl == x && trr == x) {
      tr[i] = k;
      lazy[i] = 0;
      return;
    }
    push_down(i, trr-trl+1);
    int mid = (trl+trr)>>1;
    if (x <= mid) _modify(x, k, trl, mid, i<<1);
    else _modify(x, k, mid+1, trr, i<<1|1);
    push_up(i);
  }
  void _add(const int &l, const int &r, const T &k, const int &trl, const int &trr, const int &i = 1) {
    if (trl >= l && trr <= r) {
      tr[i] += k*(trr-trl+1);
      lazy[i] += k;
      return;
    }
    push_down(i, trr-trl+1);
    int mid = (trl+trr)>>1;
    if (l <= mid) _add(l, r, k, trl, mid, i<<1);
    if (r >  mid) _add(l, r, k, mid+1, trr, i<<1|1);
    push_up(i);
  }
  T _query(const int &l, const int &r, const int &trl, const int &trr, const int &i = 1) {
    if (trl >= l && trr <= r) return tr[i];
    push_down(i, trr-trl+1);
    int mid = (trl+trr)>>1;
    T res = 0;
    if (l <= mid) res += _query(l, r, trl, mid, i<<1);
    if (r >  mid) res += _query(l, r, mid+1, trr, i<<1|1);
    return res;
  }
};
\end{lstlisting}

\subsection{区间加减区间最值 }
{\setmainfont{Consolas}
\begin{lstlisting}
template <typename T, typename U = greater<T>>
struct SegmentTree {
  U cmp = U();
  int n;
  T tr[N<<2], lazy[N<<2], init_val = cmp(0, 1) ? INF : -INF;
  SegmentTree(){}
  T mv(const T &x, const T &y) { return cmp(x, y) ? x : y;}
  void build(const int &_n, const T &k = 0) { n = _n; _build(1, n, k); }
  template <typename TT>
  void build(const TT a[], const int &_n) { n = _n; _build(a, 1, n); }
  void modify(const int &x, const T &k) { _modify(x, k, 1, n); }
  void add(const int &x, const T &k) { _add(x, x, k, 1, n); }
  void add(const int &l, const int &r, const T &k) { _add(l, r, k, 1, n); }
  T query(const int &x) { return _query(x, x, 1, n); }
  T query(const int &l, const int &r) { return _query(l, r, 1, n); }
private :
  void push_up(const int &i) { tr[i] = mv(tr[i<<1], tr[i<<1|1]); }
  void push_down(const int &i) {
    if (!lazy[i]) return;
    tr[i<<1] += lazy[i];
    tr[i<<1|1] += lazy[i];
    lazy[i<<1] += lazy[i];
    lazy[i<<1|1] += lazy[i];
    lazy[i] = 0;
  }
  void _build(const int &l, const int &r, const T &k = 0, const int &i = 1) {
    lazy[i] = 0;
    if (l == r) { tr[i] = k; return; }
    int mid = (l+r)>>1;
    _build(l, mid, k, i<<1);
    _build(mid+1, r, k, i<<1|1);
    push_up(i);
  }
  template <typename TT>
  void _build(const TT a[], const int &l, const int &r, const int &i = 1) {
    lazy[i] = 0;
    if (l == r) { tr[i] = a[l]; return; }
    int mid = (l+r)>>1;
    _build(a, l, mid, i<<1);
    _build(a, mid+1, r, i<<1|1);
    push_up(i);
  }
  void _modify(const int &x, const T &k, const int &trl, const int &trr, const int &i = 1) {
    if (trl == x && trr == x) {
      tr[i] = k;
      return;
    }
    push_down(i);
    int mid = (trl+trr)>>1;
    if (x <= mid) _modify(x, k, trl, mid, i<<1);
    else _modify(x, k, mid+1, trr, i<<1|1);
    push_up(i);
  }
  void _add(const int &l, const int &r, const T &k, const int &trl, const int &trr, const int &i = 1) {
    if (trl >= l && trr <= r) {
      tr[i] += k;
      lazy[i] += k;
      return;
    }
    push_down(i);
    int mid = (trl+trr)>>1;
    if (l <= mid) _add(l, r, k, trl, mid, i<<1);
    if (r >  mid) _add(l, r, k, mid+1, trr, i<<1|1);
    push_up(i);
  }
  T _query(const int &l, const int &r, const int &trl, const int &trr, const int &i = 1) {
    if (trl >= l && trr <= r) return tr[i];
    push_down(i);
    int mid = (trl+trr)>>1;
    T res = init_val;
    if (l <= mid) res = mv(res, _query(l, r, trl, mid, i<<1));
    if (r >  mid) res = mv(res, _query(l, r, mid+1, trr, i<<1|1));
    return res;
  }
};
\end{lstlisting}

\section{ZKW线段树}
warning:区间最值尚为验证 
{\setmainfont{Consolas}
\begin{lstlisting}
template <typename T>
struct zkwSegmentTree {
  int sz;
  T sum[N<<2], mn[N<<2], mx[N<<2], add[N<<2];
  void update(const int &x) {
    T tmp;
    tmp = min(mn[x], mn[x^1]); mn[x] -= tmp; mn[x^1] -= tmp; mn[x>>1] += tmp;
    tmp = max(mx[x], mx[x^1]); mx[x] -= tmp; mx[x^1] -= tmp; mx[x>>1] += tmp;
  }
  template <typename TT>
  void build(const TT a[], const int &n) {
    for (sz = 1; sz <= n+1; sz <<= 1);
    for (int i = sz+1; i <= sz+n; ++i)
      sum[i] = mn[i] = mx[i] = a[i-sz];
    for (int i = sz-1; i; --i) {
      sum[i] = sum[i<<1]+sum[i<<1|1];
      mn[i] = min(mn[i<<1], mn[i<<1|1]); mn[i<<1] -= mn[i]; mn[i<<1|1] -= mn[i];
      mx[i] = max(mx[i<<1], mx[i<<1|1]); mx[i<<1] -= mx[i]; mx[i<<1|1] -= mx[i];
    }
  }
  void update(int x, const T &v) {
    x += sz; mx[x] += v; mn[x] += v; sum[x] += v;
    for ( ; x > 1; x >>= 1) {
      sum[x] += v;
      update(x);
    }
  }
  void update(int s, int t, const T &v) {
    int lc = 0, rc = 0, len = 1;
    for (s += sz-1, t += sz+1; s^t^1; s >>= 1, t >>= 1, len <<= 1) {
      if (~s&1) add[s^1] += v, lc += len, mn[s^1] += v, mx[s^1] += v;
      if ( t&1) add[t^1] += v, rc += len, mn[t^1] += v, mx[t^1] += v;
      sum[s>>1] += v*lc; sum[t>>1] += v*rc;
      update(s); update(t);
    }
    for (lc += rc; s; s >>= 1) {
      sum[s>>1] += v*lc;
      update(s);
    }
  }
  T query(int x) {
    T res = 0;
    for (x += sz; x; x >>= 1) res += mn[x];
    return res;
  }
  T query_sum(int s, int t) {
    int lc = 0, rc = 0, len = 1;
    T res = 0;
    for (s += sz-1, t += sz+1; s^t^1; s >>= 1, t >>= 1, len <<= 1) {
      if (~s&1) res += sum[s^1]+len*add[s^1], lc += len;
      if ( t&1) res += sum[t^1]+len*add[t^1], rc += len;
      if (add[s>>1]) res += add[s>>1]*lc;
      if (add[t>>1]) res += add[t>>1]*rc;
    }
    for (lc += rc, s >>= 1; s; s >>= 1) if (add[s]) res += add[s]*lc;
    return res;
  }
  T query_min(int s, int t) {
    if (s == t) return query(s);
    T l = 0, r = 0, res = 0;
    for (s += sz, t += sz; s^t^1; s >>= 1, t >>= 1) {
      l += mn[s]; r += mn[t];
      if (~s^1) l = min(l, mn[s^1]);
      if ( t^1) r = min(r, mn[t^1]);
    }
    for (res = min(l, r), s >>= 1; s; s >>= 1) res += mn[s];
    return res;
  }
  T query_max(int s, int t) {
    if (s == t) return query(s);
    T l = 0, r = 0, res = 0;
    for (s += sz, t += sz; s^t^1; s >>= 1, t >>= 1) {
      l += mx[s]; r += mx[t];
      if (~s^1) l = max(l, mx[s^1]);
      if ( t^1) r = max(r, mx[t^1]);
    }
    for (res = max(l, r), s >>= 1; s; s >>= 1) res += mx[s];
    return res;
  }
};
\end{lstlisting}


\section{树状数组}
\subsection{一维}
{\setmainfont{Consolas}
\begin{lstlisting}
template <typename T>
struct BinaryIndexedTree {
  int n;
  T tr[N];
  BinaryIndexedTree() { memset(tr, 0, sizeof tr); }
  void init(const int &_n) { n = _n; clear(); }
  void clear() { for (int i = 1; i <= n; ++i) tr[i] = 0; }
  void add(const int &x, const T &v) { for (int i = x ; i <= n; i += i&-i) tr[i] += v; }
  void add(const int &x, const int &y, const T &v) { add(x, v); add(y+1, -v); }
  T query(const int &x) { T res = 0; for (int i = x ; i; i -= i&-i) res += tr[i]; return res; }
  T query(const int &x, const int &y) { return query(y)-query(x-1); }
};
\end{lstlisting}

O(n)初始化 
{\setmainfont{Consolas}
\begin{lstlisting}
template <typename TT>
void init(const int &_n, const TT a[]) {
  n = _n; clear();
  for (int i = 1; i <= n; ++i) {
    tr[i] += a[i];
    if (i+(i&-i) <= n) tr[i+(i&-i)] += tr[i];
  }
}
\end{lstlisting}
\subsection{二维}
\subsubsection{单点修改区间查询 }
{\setmainfont{Consolas}
\begin{lstlisting}
template <typename T>
struct BIT_2D {
  int n, m;
  T a[N][N], tr[N][N];
  BIT_2D() { memset(tr, 0, sizeof tr); }
  void init(const int &_n, const int &_m) {
    n = _n; m = _m;
    memset(a, 0, sizeof a);
    memset(tr, 0, sizeof tr);
  }
  void add(const int &x, const int &y, const T &k) {
    a[x][y] += k;
    for (int i = x; i <= n; i += i&-i)
      for (int j = y; j <= m; j += j&-j)
        tr[i][j] += k;
  }
  T query(const int &x, const int &y) {
    return a[x][y];
    // return query(x, y, x, y);
  }
  T query(int r1, int c1, int r2, int c2) {
    if (r1 > r2) swap(r1, r2);
    if (c1 > c2) swap(c1, c2);
    return _query(r2, c2)-_query(r1-1, c2)-_query(r2, c1-1)+_query(r1-1, c1-1);
  }
  T _query(const int &x, const int &y) {
    T res = 0;
    for (int i = x; i; i -= i&-i)
      for (int j = y; j; j -= j&-j)
        res += tr[i][j];
    return res;
  }
};
\end{lstlisting}
\section{可持久化线段树(主席树) }
{\setmainfont{Consolas}
\begin{lstlisting}
template <typename T>
struct PersistenceSegmentTree {
  static const int NN = N*(log2(N)+5);
  int rt[N], sum[NN], ls[NN], rs[NN], tot, sz;
  vector<T> des;
  void build(const T a[], const int &n) {
    vector<T>(a+1, a+n+1).swap(des);
    sort(des.begin(), des.end());
    des.erase(unique(des.begin(), des.end()), des.end());
    sz = des.size();
    tot = 0;
    rt[0] = _build(1, sz);
    for (int i = 1; i <= n; ++i) {
      int t = lower_bound(des.begin(), des.end(), a[i])-des.begin()+1;
      rt[i] = _update(rt[i-1], 1, sz, t);
    }
  }
  void update(const int &id, const T &k) {
    int t = lower_bound(des.begin(), des.end(), k)-des.begin()+1;
    rt[id] = _update(rt[id-1], 1, sz, t);
  }
  T query(const int &l, const int &r, const int &k) {
    return des[_query(rt[l-1], rt[r], 1, sz, k)-1];
  }
private:
  int _build(const int &l, const int &r) {
    int cur = ++tot;
    sum[cur] = 0;
    if (l >= r) return cur;
    int mid = (l+r)>>1;
    ls[cur] = _build(l, mid);
    rs[cur] = _build(mid+1, r);
    return cur;
  }
  int _update(const int &pre, const int &l, const int &r, const int &k) {
    int cur = ++tot;
    ls[cur] = ls[pre]; rs[cur] = rs[pre]; sum[cur] = sum[pre]+1;
    if (l >= r) return cur;
    int mid = (l+r)>>1;
    if (k <= mid) ls[cur] = _update(ls[pre], l, mid, k);
    else rs[cur] = _update(rs[pre], mid+1, r, k);
    return cur;
  }
  int _query(const int &u, const int &v, const int &l, const int &r, const int &k) {
    if (l >= r) return l;
    int num = sum[ls[v]]-sum[ls[u]], mid = (l+r)>>1;
    if (num >= k) return _query(ls[u], ls[v], l, mid, k);
    else return _query(rs[u], rs[v], mid+1, r, k-num);
  }
};
\end{lstlisting}
\section{分块}
{\setmainfont{Consolas}
\begin{lstlisting}
struct FenKuai {
  typedef long long T;
  int t; // 每组大小
  static const int NN = static_cast<int>(sqrt(N))+7;
  T a[N], sum[NN], add[NN];
  FenKuai() {
    memset(a, 0, sizeof a);
    memset(sum, 0, sizeof sum);
    memset(add, 0, sizeof add);
  }
  void init() {
    t = static_cast<int>(sqrt(n)+0.5);
    for (int i = 0; i < n; ++i) sum[i/t] += a[i];
  }
  void update(int x, T k) { a[x] += k; sum[x/t] += k; }
  void update(int x, int y, T k) {
    for ( ; x <= y && x%t; ++x) a[x] += k, sum[x/t] += k;
    for ( ; x+t-1 <= y; x += t) sum[x/t] += k*t, add[x/t] += k;
    for ( ; x <= y; ++x) a[x] += k, sum[x/t] += k;
  }
  T query(int x) { return a[x]+add[x/t]; }
  T query(int x, int y) {
    T res = 0;
    for ( ; x <= y && x%t; ++x) res += a[x]+add[x/t];
    for ( ; x+t-1 <= y; x += t) res += sum[x/t];
    for ( ; x <= y; ++x) res += a[x]+add[x/t];
    return res;
  }
} B;
\end{lstlisting}

\section{ST表}
\subsection{一维 }
{\setmainfont{Consolas}
\begin{lstlisting}
template <typename T, typename U = std::greater<T>>
struct ST {
  static const int NN = (int)log2(N)+3;
  static const T INF = 1e9;
  int lg2[N];
  U cmp = U();
  T rmq[N][NN];
  ST() {
    fill(rmq[0], rmq[0]+N*NN, cmp(-INF, +INF) ? INF : -INF);
    for (int i = 2; i < N; ++i) lg2[i] = lg2[i>>1]+1;
  }
  T& operator [] (const int &i) { return rmq[i][0]; }
  void init(const T &val = 0) { fill(rmq[0], rmq[0]+N*NN, val); }
  T mv(const T &x, const T &y) { return cmp(x, y) ? x : y; }
  // rmq[i][j] ==> [i, i+2^j-1]
  void build(T a[], const int &n) {
    for (int i = n; i; --i) {
      rmq[i][0] = a[i];
      for (int j = 1; j <= lg2[n-i+1]; ++j)
        rmq[i][j] =  mv(rmq[i][j-1], rmq[i+(1<<(j-1))][j-1]);
    }
  }
  T query(const int &l, const int &r) {
    if (l > r) return query(r, l);
    int k = lg2[r-l+1];
    return mv(rmq[l][k], rmq[r-(1<<k)+1][k]);
  }
};
  /* rmq[i][j] ==> [i-2^j+1, i]
  void build(T a[], const int &n) {
    for (int i = 1; i <= n; ++i) {
      rmq[i][0] = a[i];
      for (int j = 1; j <= lg2[i]; ++j)
        rmq[i][j] =  mv(rmq[i][j-1], rmq[i-(1<<(j-1))][j-1]);
    }
  }
  T query(const int &l, const int &r) {
    if (l > r) return query(r, l);
    int k = lg2[r-l+1];
    return mv(rmq[r][k], rmq[l+(1<<k)-1][k]);
  }
  */
\end{lstlisting}

\subsection{二维 }
{\setmainfont{Consolas}
\begin{lstlisting}
template <typename T, typename U = std::greater<T>>
struct ST {
  static const int NN = (int)log2(N)+3;
  static const T INF = 1e9;
  U cmp = U();
  T rmq[N][N][NN][NN]; // rmq[i][j][k][l] [i, j] [i+2^k-1, j+2^l-1]
  ST() { init(); }
  ST(const T &val) { init(val); }
  T& operator [] (const int &i) { return rmq[i][0]; }
  void init(){ fill(rmq[0][0][0], rmq[0][0][0]+N*N*NN*NN, cmp(-INF, +INF) ? INF : -INF); }
  void init(const T &val) { fill(rmq[0][0][0], rmq[0][0][0]+N*N*NN*NN, val); }
  T mv(const T &x, const T &y) { return cmp(x, y) ? x : y; }
  void build(T a[N][N], const int &n, const int &m) {
    for (int k = 0; k <= log_2[n]; ++k)
    for (int l = 0; l <= log_2[m]; ++l)
    for (int i = 1; i+(1<<k)-1 <= n; ++i)
    for (int j = 1; j+(1<<l)-1 <= m; ++j) {
      T &cur = rmq[i][j][k][l];
      if (!k && !l) cur = a[i][j];
      else if (!l) cur = mv(rmq[i][j][k-1][l], rmq[i+(1<<(k-1))][j][k-1][l]);
      else cur = mv(rmq[i][j][k][l-1], rmq[i][j+(1<<(l-1))][k][l-1]);
    }
  }
  T query(const int &r1, const int &c1, const int &r2, const int &c2) {
    int k = log_2[r2-r1+1], l = log_2[c2-c1+1];
    return mv(mv(rmq[r1][c1][k][l], rmq[r2-(1<<k)+1][c2-(1<<l)+1][k][l]),
          mv(rmq[r2-(1<<k)+1][c1][k][l], rmq[r1][c2-(1<<l)+1][k][l]));
  }
};
\end{lstlisting}

\subsection{反向ST }
{\setmainfont{Consolas}
\begin{lstlisting}
template <typename T, typename U = std::greater<T>>
struct rST {
  static const int NN = (int)log2(N)+3;
  static const T INF = 1e9;
  int n;
  int lg2[N];
  U cmp = U();
  T rmq[N][NN]; // rmq[i][j] ==> [i, i+2^j-1]
  rST() { for (int i = 2; i < N; ++i) lg2[i] = lg2[i>>1]+1; }
  T& operator [] (const int &i) { return rmq[i][0]; }
  T mv(const T &x, const T &y) { return cmp(x, y) ? x : y; }
  void init(const int &_n, const T &val = 0) { 
    n = _n;
    for (int i = 1; i <= n; ++i) fill(rmq[i], rmq[i]+NN, val);
  }
  void update(const int &l, const int &r, const T &k) {
    if (l > r) return void(update(r, l, k));
    int b = lg2[r-l+1];
    rmq[l][b] = mv(rmq[l][b], k);
    rmq[r-(1<<b)+1][b] = mv(rmq[r-(1<<b)+1][b], k);
  }
  void build() {
    for (int i = lg2[n]; i >= 0; --i) {
      for (int l = 1, r; l <= n; ++l) {
        r = l+(1<<i);
        if (r <= n) rmq[r][i] = mv(rmq[r][i], rmq[l][i+1]);
        rmq[l][i] = mv(rmq[l][i], rmq[l][i+1]);
      }
    }
  }
  T query(const int &l, const int &r) {
    if (l > r) return query(r, l);
    int b = lg2[r-l+1];
    return mv(rmq[l][b], rmq[r-(1<<b)+1][b]);
  }
};
\end{lstlisting}


\section{并查集 }
{\setmainfont{Consolas}
\begin{lstlisting}
struct DSU {
  int fa[N];
  void init(int sz) { for (int i = 0; i <= sz; ++i) fa[i] = i; }
  int get(int s) { return s == fa[s] ? s : fa[s] = get(fa[s]); }
  int& operator [] (int i) { return fa[get(i)]; }
  bool merge(int x, int y) {
    int fx = get(x), fy = get(y);
    if (fx == fy) return false;
    fa[fx] = fy; return true;
  }
} dsu;
\end{lstlisting}
加上按秩合并 
{\setmainfont{Consolas}
\begin{lstlisting}
struct DSU {
  int fa[N], num[N];
  void init(int sz) { for (int i = 0; i <= sz; ++i) fa[i] = i, num[i] = 1; }
  int get(int s) { return s == fa[s] ? s : fa[s] = get(fa[s]); }
  int& operator [] (int i) { return fa[get(i)]; }
  bool merge(int x, int y) {
    int fx = get(x), fy = get(y);
    if (fx == fy) return false;
    if (num[fx] >= num[fy]) num[fx] += num[fy], fa[fy] = fx;
    else num[fy] += num[fx], fa[fx] = fy;
    return true;
  }
} dsu;
\end{lstlisting}


\part{字符串}
\section{回文字符串|manacher算法}
从 0 开始,第 i 位对应 p[i*2+2] 
{\setmainfont{Consolas}
\begin{lstlisting}
inline int manacher(const char *str, char *buf, int *p) {
  int str_len = strlen(str), buf_len = 2;
  buf[0] = buf[1] = '#';
  for(int i = 0; i < str_len; ++i)
    buf[buf_len++] = str[i], buf[buf_len++] = '#';

  int mx = 0, id, ans = 0;
  for(int i = 1; i < buf_len; ++i) {
    if(i <= mx) p[i] = min(p[id*2-i], mx-i);
    else p[i] = 1;
    while(buf[i-p[i]] == buf[i+p[i]]) p[i]++;
    if(i+p[i] > mx) mx = i+p[i], id = i;
    ans = max(ans, p[i]-1);
  }
  return ans;
}
\end{lstlisting}
\subsection{判断s[l, r]是否为回文 }
{\setmainfont{Consolas}
\begin{lstlisting}
p[l+r+2]-1 >= r-l+1
\end{lstlisting}


\section{KMP}
{\setmainfont{Consolas}
\begin{lstlisting}
inline void get_next(const string &s, int nex[]) { get_next(s.c_str(), nex); }
inline void get_next(const char *s, int nex[]) {
  nex[0] = nex[1] = 0;
  for (int i = 1, j = 0, l = strlen(s); i < l; ++i) {
    while (j && s[i] != s[j]) j = nex[j];
    nex[i+1] = s[i] == s[j] ? ++j : 0;
  }
}

inline void kmp(const string &s1, const string &s2, int nex[]) { kmp(s1.c_str(), s2.c_str(), nex); }
inline void kmp(const char *s1, const char *s2, int nex[]) {
  for (int i = 0, j = 0, l1 = strlen(s1), l2 = strlen(s2); i < l1; ++i){
    while (j && s1[i] != s2[j]) j = nex[j];
    if (s1[i] == s2[j]) ++j;
    if (j == l2) {
      cout << i-l2+2 << endl;
      j = nex[j];
    }
  }
}
\end{lstlisting}
{\setmainfont{Consolas}
\begin{lstlisting}
inline void get_next(const string &s, int nex[]) {
  nex[0] = nex[1] = 0;
  for (int i = 1, j = 0; i < (int)s.size(); ++i) {
    while (j && s[i] != s[j]) j = nex[j];
    nex[i+1] = s[i] == s[j] ? ++j : 0;
  }
}

inline void kmp(const string &s1, const string &s2, int nex[]) {
  for (int i = 0, j = 0; i < (int)s1.size(); ++i) {
    while (j && s1[i] != s2[j]) j = nex[j];
    if (s1[i] == s2[j]) ++j;
    if (j == (int)s2.size()) {
      cout << i-s2.size()+2 << endl;
      j = nex[j];
    }
  }
}
\end{lstlisting}
\section{扩展KMP|Z函数 }
{\setmainfont{Consolas}
\begin{lstlisting}
inline void GetNext(char *s, int *_nex) {
  int len = strlen(s);
  int a = 0, p = 0;
  _nex[0] = len;
  for (int i = 1; i < len; ++i) {
    if (i >= p || i+_nex[i-a] >= p) {
      if (i > p) p = i;
      while (p < len && s[p] == s[p-i]) ++p;
      a = i;
      _nex[i] = p-i;
    } else {
      _nex[i] = _nex[i-a];
    }
  }
}

inline void GetExtend(char *s, char *ss, int *_ext, int *_nex) {
  int lens = strlen(s), lenss = strlen(ss);
  int a = 0, p = 0;
  for (int i = 0; i < lens; ++i) {
    if (i >= p || i+_nex[i-a] >= p) {
      if (i > p) p = i;
      while (p < lens && p-i < lenss && s[p] == ss[p-i]) ++p;
      a = i;
      _ext[i] = p-i;
    } else {
      _ext[i] = _nex[i-a];
    }
  }
}
\end{lstlisting}


\section{字符串哈希 }
{\setmainfont{Consolas}
\begin{lstlisting}
inline unsigned long long _hash(const string &s) {
  unsigned long long res = 0;
  for(int i = 0; i < s.length(); ++i)
    res = (res*Base+s[i])%Mod+Prime;
  return res;
}
\end{lstlisting}


\section{后缀数组|SA}
\subsection{$O(nlog^2n)$ }
{\setmainfont{Consolas}
\begin{lstlisting}
int sa[N], rk[N<<1], height[N];
template <typename T> // s start from 1
inline void SA(const T *s, const int &n) {
  static int oldrk[N<<1];
  memset(rk+n+1, 0, sizeof(int)*n);
  for (int i = 1; i <= n; ++i) rk[i] = s[i];
  for (int w = 1; w <= n; w <<= 1) {
    iota(sa+1, sa+n+1, 1);
    sort(sa+1, sa+n+1, [&](const int &x, const int &y) {
      return rk[x] == rk[y] ? rk[x+w] < rk[y+w] : rk[x] < rk[y];
    });
    memcpy(oldrk+1, rk+1, sizeof(int)*2*n);
    for (int p = 0, i = 1; i <= n; ++i) {
      if (oldrk[sa[i]] == oldrk[sa[i-1]] &&
        oldrk[sa[i]+w] == oldrk[sa[i-1]+w]) {
        rk[sa[i]] = p;
      } else {
        rk[sa[i]] = ++p;
      }
    }
  }
  for (int i = 1, k = 0; i <= n; ++i) {
    if (k) --k;
    while (s[i+k] == s[sa[rk[i]-1]+k]) ++k;
    height[rk[i]] = k;
  }
}
\end{lstlisting}

\subsection{$O(nlogn)$ }
{\setmainfont{Consolas}
\begin{lstlisting}
int sa[N], rk[N<<1], height[N];
template <typename T> // s start from 1
inline void SA(const T *s, const int &n) {
#define cmp(x, y, w) oldrk[x] == oldrk[y] && oldrk[x + w] == oldrk[y + w]
  static int oldrk[N<<1], id[N], px[N], cnt[N], m;
  memset(cnt, 0, sizeof(int) * (m = 128));
  for (int i = 1; i <= n; ++i) ++cnt[rk[i] = s[i]];
  for (int i = 1; i <= m; ++i) cnt[i] += cnt[i - 1];
  for (int i = n; i; --i) sa[cnt[rk[i]]--] = i;
  for (int w = 1, p, i; w <= n; w <<= 1, m = p) {
    for (p = 0, i = n; i > n - w; --i) id[++p] = i;
    for (int i = 1; i <= n; ++i)
      if (sa[i] > w)
        id[++p] = sa[i] - w;
    memset(cnt + 1, 0, sizeof(int) * m);
    for (int i = 1; i <= n; ++i) ++cnt[px[i] = rk[id[i]]];
    for (int i = 1; i <= m; ++i) cnt[i] += cnt[i - 1];
    for (int i = n; i; --i) sa[cnt[px[i]]--] = id[i];
    memcpy(oldrk + 1, rk + 1, sizeof(int) * 2 * n);
    for (p = 0, i = 1; i <= n; ++i) rk[sa[i]] = cmp(sa[i], sa[i - 1], w) ? p : ++p;
  }
  for (int i = 1, k = 0; i <= n; ++i) {
    if (k) --k;
    while (s[i+k] == s[sa[rk[i]-1]+k]) ++k;
    height[rk[i]] = k;
  }
#undef cmp
}
\end{lstlisting}

\subsection{$O(n)$ }
{\setmainfont{Consolas}
\begin{lstlisting}
namespace SuffixArray {

int sa[N], rk[N], ht[N];
bool t[N << 1];

inline bool islms(const int i, const bool *t) { return i > 0 && t[i] && !t[i - 1]; }

template <class T>
inline void sort(T s, int *sa, const int len, const int sz, const int sigma, bool *t, int *b, int *cb, int *p) {
  memset(b, 0, sizeof(int) * sigma);
  memset(sa, -1, sizeof(int) * len);
  for (register int i = 0; i < len; i++) b[static_cast<int>(s[i])]++;
  cb[0] = b[0];
  for (register int i = 1; i < sigma; i++) cb[i] = cb[i - 1] + b[i];
  for (register int i = sz - 1; i >= 0; i--) sa[--cb[static_cast<int>(s[p[i]])]] = p[i];
  for (register int i = 1; i < sigma; i++) cb[i] = cb[i - 1] + b[i - 1];
  for (register int i = 0; i < len; i++)
    if (sa[i] > 0 && !t[sa[i] - 1])
      sa[cb[static_cast<int>(s[sa[i] - 1])]++] = sa[i] - 1;
  cb[0] = b[0];
  for (register int i = 1; i < sigma; i++) cb[i] = cb[i - 1] + b[i];
  for (register int i = len - 1; i >= 0; i--)
    if (sa[i] > 0 && t[sa[i] - 1])
      sa[--cb[static_cast<int>(s[sa[i] - 1])]] = sa[i] - 1;
}

template <class T>
inline void sais(T s, int *sa, const int len, bool *t, int *b, int *b1, const int sigma) {
  register int i, j, x, p = -1, cnt = 0, sz = 0, *cb = b + sigma;
  for (t[len - 1] = 1, i = len - 2; i >= 0; i--) t[i] = s[i] < s[i + 1] || (s[i] == s[i + 1] && t[i + 1]);
  for (i = 1; i < len; i++)
    if (t[i] && !t[i - 1])
      b1[sz++] = i;
  sort(s, sa, len, sz, sigma, t, b, cb, b1);
  for (i = sz = 0; i < len; i++)
    if (islms(sa[i], t))
      sa[sz++] = sa[i];
  for (i = sz; i < len; i++) sa[i] = -1;
  for (i = 0; i < sz; i++) {
    for (x = sa[i], j = 0; j < len; j++) {
      if (p == -1 || s[x + j] != s[p + j] || t[x + j] != t[p + j]) {
        cnt++, p = x;
        break;
      } else if (j > 0 && (islms(x + j, t) || islms(p + j, t))) {
        break;
      }
    }
    sa[sz + (x >>= 1)] = cnt - 1;
  }
  for (i = j = len - 1; i >= sz; i--)
    if (sa[i] >= 0)
      sa[j--] = sa[i];
  register int *s1 = sa + len - sz, *b2 = b1 + sz;
  if (cnt < sz)
    sais(s1, sa, sz, t + len, b, b1 + sz, cnt);
  else
    for (i = 0; i < sz; i++) sa[s1[i]] = i;
  for (i = 0; i < sz; i++) b2[i] = b1[sa[i]];
  sort(s, sa, len, sz, sigma, t, b, cb, b2);
}

template <class T>
inline void getHeight(T s, int n) {
  for (register int i = 1; i <= n; i++) rk[sa[i]] = i;
  register int j = 0, k = 0;
  for (register int i = 0; i < n; ht[rk[i++]] = k)
    for (k ? k-- : 0, j = sa[rk[i] - 1]; s[i + k] == s[j + k]; k++)
      ;
}

template <class T>  // s start from 0
inline void init(T s, const int len, const int sigma = 128) {
  sais(s, sa, len + 1, t, rk, ht, sigma);
  getHeight(s, len);
  for (int i = 1; i <= len; ++i) ++sa[i];
  for (int i = len; i; --i) rk[i] = rk[i-1];
}

}  // namespace SuffixArray
\end{lstlisting}

\section{字典树 }
{\setmainfont{Consolas}
\begin{lstlisting}
struct TireTree {
  static const int NN = 5e5+7;
  static const int SZ = 26;
  char beg;
  int nex[NN][SZ], num[NN], cnt;
  bool exist[NN];
  TireTree(char _beg = 'a') : beg(_beg) { clear(); }
  void clear() {
    memset(nex, 0, sizeof nex);
    memset(num, 0, sizeof num);
    memset(exist, 0, sizeof exist);
    cnt = 0;
  }
  void insert(const char *s) {
    int len = strlen(s), p = 0;
    for (int i = 0, c; i < len; ++i) {
      c = s[i]-beg;
      if (!nex[p][c]) nex[p][c] = ++cnt;
      p = nex[p][c];
      ++num[p];
    }
    exist[p] = true;
  }
  bool find(const char *s) {
    int len = strlen(s), p = 0;
    for (int i = 0, c; i < len; ++i) {
      c = s[i]-beg;
      if (!nex[p][c]) return false;
      p = nex[p][c];
    }
    return exist[p];
  }
  int count(const char *s) {
    int len = strlen(s), p = 0;
    for (int i = 0, c; i < len; ++i) {
      c = s[i]-beg;
      if (!nex[p][c]) return 0;
      p = nex[p][c];
    }
    return num[p];
  }
  void insert(const string &s) { insert(s.c_str()); }
  bool find(const string &s) { return find(s.c_str()); }
  int count(const string &s) { return count(s.c_str()); }
};
\end{lstlisting}

\section{AC自动机}
如需构造可重建AC自动机,每次构造建一个nex数组的拷贝 
{\setmainfont{Consolas}
\begin{lstlisting}
struct Aho_Corasick_Automaton {
  static const int NN = 5e6+7;
  static const int SZ = 26;
  char beg;
  int nex[NN][SZ], num[NN], fail[NN], cnt;
  Aho_Corasick_Automaton(const char &_beg = 'a') : beg(_beg) {}
  void clear() {
    memset(nex, 0, sizeof(nex[0])*(cnt+1));
    memset(num, 0, sizeof(int)*(cnt+1));
    memset(fail, 0, sizeof(int)*(cnt+1));
    cnt = 0;
  }
  void insert(const char *s) {
    int len = strlen(s), p = 0;
    for (int i = 0, c; i < len; ++i) {
      c = s[i]-beg;
      if (!nex[p][c]) nex[p][c] = ++cnt;
      p = nex[p][c];
    }
    ++num[p];
  }
  void build() {
    static queue<int> q;
    for (int i = 0; i < SZ; ++i) if (nex[0][i]) q.push(nex[0][i]);
    while (q.size()) {
      int u = q.front();
      q.pop();
      for (int i = 0; i < SZ; ++i) {
        if (nex[u][i]) {
          fail[nex[u][i]] = nex[fail[u]][i];
          q.push(nex[u][i]);
        } else {
          nex[u][i] = nex[fail[u]][i];
        }
      }
    }
  }
  int query(const char *s) {
    int len = strlen(s), p = 0, res = 0;
    for (int i = 0; i < len; ++i) {
      p = nex[p][s[i]-beg];
      for (int t = p; t && ~num[t]; t = fail[t]) {
        res += num[t];
        num[t] = -1;
      }
    }
    return res;
  }
};
\end{lstlisting}

{\setmainfont{Consolas}
\begin{lstlisting}
struct Aho_Corasick_Automaton {
  static const int NN = 2e5+7;
  static const int SZ = 26;
  char beg;
  int cnt;
  int nex[NN][SZ], fail[NN], vis[NN];
  Aho_Corasick_Automaton(const char &_beg = 'a') : beg(_beg) {}
  void clear() {
    memset(nex, 0, sizeof(nex[0])*(cnt+1));
    memset(fail, 0, sizeof(int)*(cnt+1));
    memset(vis, 0, sizeof(int)*(cnt+1));
    cnt = 0;
  }
  int insert(const char *s) {
    int len = strlen(s), p = 0;
    for (int i = 0, c; i < len; ++i) {
      c = s[i]-beg;
      if (!nex[p][c]) nex[p][c] = ++cnt;
      p = nex[p][c];
    }
    return p;
  }
  void build() {
    static queue<int> q;
    for (int i = 0; i < SZ; ++i) if (nex[0][i]) q.push(nex[0][i]);
    while (q.size()) {
      int u = q.front();
      q.pop();
      for (int i = 0; i < SZ; ++i) {
        if (nex[u][i]) {
          fail[nex[u][i]] = nex[fail[u]][i];
          q.push(nex[u][i]);
        } else {
          nex[u][i] = nex[fail[u]][i];
        }
      }
    }
  }
  void query(char *s) {
    static int deg[NN];
    static queue<int> q;

    int len = strlen(s);
    for (int i = 0, p = 0; i < len; ++i) {
      p = nex[p][s[i]-beg];
      ++vis[p];
      // for (int t = p; t; t = fail[t]) ++vis[t];
    }
    for (int i = 1; i <= cnt; ++i) ++deg[fail[i]];
    for (int i = 1; i <= cnt; ++i) if (!deg[i]) q.push(i);
    while (q.size()) {
      int u = q.front();
      q.pop();
      vis[fail[u]] += vis[u];
      if (--deg[fail[u]] == 0) q.push(fail[u]);
    }
  }
} ac;
\end{lstlisting}


\part{图论|树论}
\section{DFS树}
\section{树的重心 }
{\setmainfont{Consolas}
\begin{lstlisting}
void treedp(int cur, int fa) {
  s[cur] = c[cur];
  for(int i = fir[cur]; i; i = nex[i]) {
    if(e[i] == fa) continue;
    treedp(e[i], cur);
    s[cur] += s[e[i]];
    maxs[cur] = max(maxs[cur], s[e[i]]);
  }
  maxs[cur] = max(maxs[cur], sum-s[cur]);
}
\end{lstlisting}
\section{最大团}
最大独立集数=补图的最大团 
{\setmainfont{Consolas}
\begin{lstlisting}
struct MaxClique {
  vector<int> res, tmp, cnt;
  bool dfs(int p) {
    for (int i = p+1, flag; i <= n; ++i) {
      if (cnt[i]+tmp.size() <= res.size()) return false;
      if (!g[p][i]) continue;
      flag = 1;
      for (int j : tmp)
        if (!g[i][j]) flag = 0;
      if (!flag) continue;
      tmp.push_back(i);
      if (dfs(i)) return true;
      tmp.pop_back();
    }
    if (tmp.size() > res.size()) {
      res = tmp;
      return true;
    }
    return false;
  }
  void solve() {
    vector<int>(n+1, 0).swap(cnt);
    vector<int>().swap(res);
    for (int i = n; i; --i) {
      vector<int>(1, i).swap(tmp);
      dfs(i);
      cnt[i] = res.size();
    }
  }
} MC;
\end{lstlisting}


\section{稳定婚姻匹配 }
{\setmainfont{Consolas}
\begin{lstlisting}
template <typename T = int> struct Stable_Marriage {
  int t[N], b[N], g[N], rkb[N][N], rkg[N][N];
  T wb[N][N], wg[N][N];
  queue<int> q;
  void init(const int &n) {
    queue<int>().swap(q);
    memset(t, 0, sizeof(int)*(n+3));
    memset(b, 0, sizeof(int)*(n+3));
    memset(g, 0, sizeof(int)*(n+3));
    for (int i = 1; i <= n; ++i) {
      q.push(i);
      for (int j = 1; j <= n; ++j)
        rkb[i][j] = rkg[i][j] = j;
      sort(rkb[i]+1, rkb[i]+n+1,
         [&](const int &x, const int &y) { return wb[i][y] < wb[i][x]; });
      //sort(rkg[i]+1, rkg[i]+n+1,
      //   [&](const int &x, const int &y) { return wg[i][y] < wg[i][x]; });
    }
  }
  bool match(const int &x, const int &y) {
    if (g[y]) {
      if (wg[y][x] < wg[y][g[y]]) return false;
      b[g[y]] = 0;
      q.push(g[y]);
    }
    b[x] = y; g[y] = x;
    return true;
  }
  void gale_shapely(const int &n) {
    init(n);
    while (q.size()) {
      int x = q.front(); q.pop();
      int y = rkb[x][++t[x]];
      if (!match(x, y)) q.push(x);
    }
  }
};
\end{lstlisting}


\section{最小生成树}

\section{二分图}
\subsection{二分图匹配}
**匈牙利算法** 
{\setmainfont{Consolas}
\begin{lstlisting}
bool check(int u) {
  for (int v : e[u]) {
    if (vis[v]) continue;
    vis[v] = 1;
    if (!co[v] || check(co[v])) {
      co[v] = u;
      return true;
    }
  }
  return false;
}

inline int solve() {
  int res = 0;
  memset(co, 0, sizeof co);
  for (int i = 1; i <= n; ++i) {
    memset(vis, 0, sizeof(int)*(n+3));
    res += check(i);
  }
  return res;
}
\end{lstlisting}
\subsection{二分图最小顶点覆盖}
定义:假如选了一个点就相当于覆盖了以它为端点的所有边。最小顶点覆盖就是选择最少的点来覆盖所有的边。

定理:最小顶点覆盖等于二分图的最大匹配。
\subsection{最大独立集}

定义:选出一些顶点使得这些顶点两两不相邻,则这些点构成的集合称为独立集。找出一个包含顶点数最多的独立集称为最大独立集。

定理:最大独立集 = 所有顶点数 - 最小顶点覆盖 = 所有顶点数 -   最大匹配


\section{LCA }
{\setmainfont{Consolas}
\begin{lstlisting}
struct LCA {
  static const int NN = (int)log2(N)+3;
  int f[N][NN], d[N], lg2[N];
  LCA() { for (int i = 2; i < N; ++i) lg2[i] = lg2[i>>1]+1; }
  template <typename TT>
  void build(const TT e[], const int &u = 1, const int &fa = 0) {
    d[u] = d[fa]+1;
    f[u][0] = fa;
    for (int i = 1; (1<<i) <= d[u]; ++i)
      f[u][i] = f[f[u][i-1]][i-1];
    for (auto v : e[u]) if (v != fa)
      build(e, v, u);
  }
  int get(int x, int y) {
    if (d[x] < d[y]) swap(x, y);
    while (d[x] > d[y])
      x = f[x][lg2[d[x]-d[y]]];
    if (x == y) return x;
    for (int i = lg2[d[x]]; i >= 0; --i)
      if(f[x][i] != f[y][i])
        x = f[x][i], y = f[y][i];
    return f[x][0];
  }
};
\end{lstlisting}
带权LCA 
{\setmainfont{Consolas}
\begin{lstlisting}
template <typename T>
struct LCA {
  static const int NN = (int)log2(N)+3;
  int f[N][NN], d[N], lg2[N];
  T w[N][NN], init_val = 0;
  LCA() {
    for (int i = 2; i < N; ++i) lg2[i] = lg2[i>>1]+1;
    init();
  }
  // set sum or min or max, and don't forget to set init_val
  T update(const T &x, const T &y) { return x+y; }
  void init(const int &n = N-1) {
    fill(w[0], w[0]+(n+1)*NN, init_val);
  }
  template <typename TT>
  void build(const TT e[], const int &u = 1, const int &fa = 0) {
    d[u] = d[fa]+1;
    f[u][0] = fa;
    for (int i = 1; (1<<i) <= d[u]; ++i) {
      f[u][i] = f[f[u][i-1]][i-1];
      w[u][i] = update(w[u][i-1], w[f[u][i-1]][i-1]);
    }
    for (auto v : e[u]) if (v.first != fa) {
      w[v.first][0] = v.second;
      build(e, v.first, u);
    }
  }
  T get(int x, int y) {
    T res = init_val;
    if (d[x] < d[y]) swap(x, y);
    while (d[x] > d[y]) {
      res = update(res, w[x][lg2[d[x]-d[y]]]);
      x = f[x][lg2[d[x]-d[y]]];
    }
    if (x == y) return res;
    for (int i = lg2[d[x]]; i >= 0; --i)
      if(f[x][i] != f[y][i]) {
        res = update(res, w[x][i]);
        res = update(res, w[y][i]);
        x = f[x][i], y = f[y][i];
      }
    return update(res, update(w[x][0], w[y][0]));
  }
};
\end{lstlisting}

\section{树上差分 }
{\setmainfont{Consolas}
\begin{lstlisting}
template <typename T>
struct Tree {
  T val[N];
  void update_point(const int &x, const int &y, const T &k) {
    int _lca = lca(x, y);
    val[x] += k; val[y] += k;
    val[_lca] -= k; val[f[_lca][0]] -= k;
  }
  void update_edge(const int &x, const int &y, const T &k) {
    int _lca = lca(x, y);
    val[x] += k; val[y] += k; val[_lca] -= 2*k;
  }
  void dfs(const int &u = 1, const int &fa = 0) {
    for (int v : e[u]) if (v != fa) {
      dfs(v, u);
      val[u] += val[v];
    }
  }
};
\end{lstlisting}
\section{树链剖分 }
{\setmainfont{Consolas}
\begin{lstlisting}
template <typename T>
struct ShuPou {
  int dfn;
  int fa[N], d[N], num[N], son[N], id[N], tp[N];
  T init_val[N];
  SegmentTree<T> ST;
  template <typename Edge, typename TT>
  void build(const Edge e[], const TT a[], const int &n, const int &rt = 1) {
    fa[rt] = dfn = 0;
    dfs1(e, rt);
    dfs2(e, rt);
    for (int i = 1; i <= n; ++i)
      init_val[id[i]] = a[i];
    ST.build(init_val, n);
  }
  template <typename Edge>
  void dfs1(const Edge e[], const int &u = 1) {
    d[u] = d[fa[u]]+1;
    num[u] = 1;
    son[u] = 0;
    for (const int &v : e[u]) if (v != fa[u]) {
      fa[v] = u;
      dfs1(e, v);
      num[u] += num[v];
      if (num[v] > num[son[u]]) son[u] = v;
    }
  }
  template <typename Edge>
  void dfs2(const Edge e[], const int &u = 1) {
    tp[u] = son[fa[u]] == u ? tp[fa[u]] : u;
    id[u] = ++dfn;
    if (son[u]) dfs2(e, son[u]);
    for (const int &v : e[u]) if (v != son[u] && v != fa[u])
      dfs2(e, v);
  }
  void add_sons(const int &x, const T &k) { ST.add(id[x], id[x]+num[x]-1, k); }
  void add(int x, int y, const T &k, const int &is_edge = 0) {
    while (tp[x] != tp[y]) {
      if (d[tp[x]] < d[tp[y]]) swap(x, y);
      ST.add(id[tp[x]], id[x], k);
      x = fa[tp[x]];
    }
    if (d[x] > d[y]) swap(x, y);
    ST.add(id[x], id[y], k);
    if (is_edge) ST.add(id[x], -k);
  }
  T query_sons(const int &x) { return ST.query(id[x], id[x]+num[x]-1); }
  T query(const int &x) { return ST.query(id[x]); }
  T query(int x, int y) {
    T res = 0;
    while (tp[x] != tp[y]) {
      if (d[tp[x]] < d[tp[y]]) swap(x, y);
      res += ST.query(id[tp[x]], id[x]);
      x = fa[tp[x]];
    }
    if (d[x] > d[y]) swap(x, y);
    return res+ST.query(id[x], id[y]);
  }
};
\end{lstlisting}


\section{网络流}
\subsection{最大流}
\subsubsection{Dinic}
普通情况下 $O(n^2m)$
二分图中 $O(\sqrt{n}m)$ 
{\setmainfont{Consolas}
\begin{lstlisting}
template <typename T>
struct Dinic {
  struct EDGE {
    int v, nex;
    T w;
    EDGE(const int &_v, const int &_nex, const T &_w) : v(_v), nex(_nex), w(_w) {}
  };
  vector<EDGE> e;
  int n, s, t;
  int fir[N], dep[N], cur[N];
  Dinic() { e.reserve(N<<2); }
  T work(const int &_s, const int &_t) {
    s = _s; t = _t;
    T maxflow = 0, flow;
    while (bfs())
      while ((flow = dfs(s, INF)))
        maxflow += flow;
    return maxflow;
  }
  void init(const int &_n) {
    n = _n;
    e.clear();
    memset(fir, -1, sizeof(int)*(n+3));
  }
  void add_edge(const int &u, const int &v, const T &w) {
    e.emplace_back(v, fir[u], w); fir[u] = e.size()-1;
    e.emplace_back(u, fir[v], 0); fir[v] = e.size()-1;
  }
  bool bfs() {
    queue<int> q;
    memset(dep, 0, sizeof(int)*(n+3));
    q.push(s);
    dep[s] = 1;
    for (int i = 0; i <= n; ++i) cur[i] = fir[i];
    while (q.size()) {
      int u = q.front();
      q.pop();
      for (int i = fir[u], v; i != -1; i = e[i].nex) {
        v = e[i].v;
        if (dep[v] || !e[i].w) continue;
        dep[v] = dep[u]+1;
        if (v == t) return true;
        q.push(v);
      }
    }
    return false;
  }
  T dfs(const int &u, const T &flow) {
    if (!flow || u == t) return flow;
    T rest = flow, now;
    for (int &i = cur[u], v; i != -1; i = e[i].nex) {
      v = e[i].v;
      if (dep[v] != dep[u]+1 || !e[i].w) continue;
      now = dfs(v, min(rest, e[i].w));
      if (!now) {
        dep[v] = 0;
      } else {
        e[i].w -= now;
        e[i^1].w += now;
        rest -= now;
        if (rest == flow) break;
      }
    }
    return flow-rest;
  }
};
\end{lstlisting}
\subsubsection{ISAP}
渐进时间复杂度和dinic相同,但是非二分图的情况下isap更具优势 
{\setmainfont{Consolas}
\begin{lstlisting}
template <typename T>
struct ISAP {
  struct EDGE
  {
    int v, nex;
    T w;
    EDGE(const int &_v, const int &_nex, const T &_w) : v(_v), nex(_nex), w(_w) {}
  };
  vector<EDGE> e;
  int n, s, t;
  T maxflow;
  int fir[N], gap[N], dep[N];
  T work(const int &_s, const int &_t) {
    s = _s; t = _t;
    maxflow = 0;
    bfs();
    while (dep[s] < n) dfs(s, INF);
    return maxflow;
  }
  void init(const int &_n) {
    n = _n;
    e.clear();
    e.reserve(N<<2);
    memset(fir, -1, sizeof(int)*(n+3));
  }
  void add_edge(const int &u, const int &v, const T &w) {
    e.emplace_back(v, fir[u], w); fir[u] = e.size()-1;
    e.emplace_back(u, fir[v], 0); fir[v] = e.size()-1;
  }
  void bfs() {
    queue<int> q;
    memset(dep, -1, sizeof(int)*(n+3));
    memset(gap, 0, sizeof(int)*(n+3));
    dep[t] = 0;
    gap[0] = 1;
    q.push(t);
    while (q.size()) {
      int u = q.front();
      q.pop();
      for (int i = fir[u], v; i != -1; i = e[i].nex) {
        v = e[i].v;
        if (dep[v] != -1) continue;
        q.push(v);
        dep[v] = dep[u]+1;
        ++gap[dep[v]];
      }
    }
  }
  T dfs(const int &u, const T &flow) {
    if (u == t) {
      maxflow += flow;
      return flow;
    }
    T used = 0;
    for (int i = fir[u], v; i != -1; i = e[i].nex) {
      v = e[i].v;
      if (!e[i].w || dep[v]+1 != dep[u]) continue;
      T minf = dfs(v, min(e[i].w, flow-used));
      if (minf) {
        e[i].w -= minf;
        e[i^1].w += minf;
        used += minf;
      }
      if (used == flow) return used;
    }
    if (--gap[dep[u]] == 0) dep[s] = n+1;
    ++gap[++dep[u]];
    return used;
  }
};
\end{lstlisting}
\subsubsection{HLPP}
\subsection{最小割}
最小割等价最大流
\subsection{费用流}
\subsubsection{ZKW\_SPFA}
{\setmainfont{Consolas}
\begin{lstlisting}
template <typename T>
struct ZKW_SPFA {
  struct Edge {
    int v, nex;
    T w, c; // edge wight and cost
    Edge(const int &_v, const int &_nex, const T &_w, const T &_c) \
    : v(_v), nex(_nex), w(_w), c(_c) {}
  };
  vector<Edge> e;
  int n, s, t;
  int fir[N], vis[N];
  T maxflow, mincost;
  T dis[N];
  ZKW_SPFA() { e.reserve(N<<4); }
  void init(const int &_n) {
    n = _n;
    maxflow = mincost = 0;
    e.clear();
    memset(fir, -1, sizeof(int)*(n+3));
  }
  void add_edge(const int &u, const int &v, const T &w = 1, const T &c = 0) {
    e.emplace_back(v, fir[u], w, c); fir[u] = e.size()-1;
    e.emplace_back(u, fir[v], 0, -c); fir[v] = e.size()-1;
  }
  pair<T, T> work(const int &_s, const int &_t) {
    s = _s; t = _t;
    while (spfa()) {
      vis[t] = 1;
      while (vis[t]) {
        memset(vis, 0, sizeof(int)*(n+3));
        maxflow += dfs(s, INF);
      }
    }
    return {maxflow, mincost};
  }
  private:
  bool spfa() {
    memset(dis, 0x3f, sizeof(T)*(n+3));
    memset(vis, 0, sizeof(int)*(n+3));
    deque<int> q;
    q.push_back(t);
    dis[t] = 0;
    vis[t] = 1;
    while (q.size()) {
      int u = q.front(); q.pop_front();
      for (int i = fir[u], v; ~i; i = e[i].nex) {
        v = e[i].v;
        if (!e[i^1].w || dis[v] <= dis[u]+e[i^1].c) continue;
        dis[v] = dis[u]+e[i^1].c;
        if (vis[v]) continue;
        vis[v] = 1;
        if (q.size() && dis[v] < dis[q.front()]) q.push_front(v);
        else q.push_back(v);
      }
      vis[u] = 0;
    }
    return dis[s] < INF;
  }
  T dfs(const int &u, const T &flow) {
    vis[u] = 1;
    if (u == t || flow <= 0) return flow;
    T res, used = 0;
    for (int i = fir[u], v; ~i; i = e[i].nex) {
      v = e[i].v;
      if (vis[v] || !e[i].w || dis[u] != dis[v]+e[i].c) continue;
      res = dfs(v, min(e[i].w, flow-used));
      if (!res) continue;
      mincost += res*e[i].c;
      e[i].w -= res;
      e[i^1].w += res;
      used += res;
      if (used == flow) break;
    }
    return used;
  }
};
\end{lstlisting}
\subsection{上下界网络流}

\section{最短路}
\subsection{Floyd}
\subsection{Dijiskra}
\subsection{SPFA }
{\setmainfont{Consolas}
\begin{lstlisting}
inline void SPFA() {
  fill(dis+1, dis+n+1, INT_MAX);
  dis[S] = 0;
  head = tail = 0;
  q[++tail] = S;
  while(head < tail) {
    int cur = q[++head];
    for(int i = fir[cur], to, tmp; i; i = nex[i]) {
      to = ver[i];
      tmp = dis[cur]+w[i];
      if(tmp >= dis[to]) continue;
      dis[to] = tmp;
      q[++tail] = to;
    }
  }
}
\end{lstlisting}


\section{负环 }
{\setmainfont{Consolas}
\begin{lstlisting}
// 返回true有负环,返回false没负环
inline bool SPFA() {
  q[++tail] = 1;
  vis[1] = 1;
  cnt[1] = 1;
  dis[1] = 0;
  while(head < tail) {
    int cur = q[(++head)%Maxn];
    vis[cur] = 0;
    for(int i = fir[cur], to; i; i = nex[i]) {
      to = ver[i];
      if(dis[cur]+w[i] < dis[to]) {
        dis[to] = dis[cur]+w[i];
        if(!vis[to]) {
          q[(++tail)%Maxn] = to;
          vis[to] = 1;
          if(++cnt[to] > n) return true;
        }
      }
    }
  }
  return false;
}
\end{lstlisting}


\section{割点 }
{\setmainfont{Consolas}
\begin{lstlisting}
void tarjan(int cur, int fa) {
  dfn[cur] = low[cur] = ++_dfn;
  int child = 0;
  for(auto i : e[cur]) {
    if(!dfn[i]) {
      child++;
      tarjan(i, fa);
      low[cur] = min(low[cur], low[i]);
      if(cur != fa && low[i] >= dfn[cur]) flag[cur] = 1;
    }
    low[cur] = min(low[cur], dfn[i]);
  }
  if(cur == fa && child >= 2) flag[cur] = 1;
}
\end{lstlisting}


\section{SCC强连通分量|Tarjan}
\subsection{递归版本 }
{\setmainfont{Consolas}
\begin{lstlisting}
int _dfn, _col, _top;
int dfn[N], low[N], vis[N], col[N], sta[N];

void tarjan(const int &u) {
  dfn[u] = low[u] = ++_dfn;
  vis[u] = 1;
  sta[++_top] = u;
  for (int v : e[u]) {
    if (!dfn[v]) {
      tarjan(v);
      low[u] = min(low[u], low[v]);
    } else if (vis[v]) {
      low[u] = min(low[u], low[v]);
    }
  }
  if (dfn[u] == low[u]) {
    ++_col;
    do {
      col[sta[_top]] = _col;
      vis[sta[_top]] = 0;
    } while (sta[_top--] != u);
  }
}
\end{lstlisting}

\section{缩点 }
{\setmainfont{Consolas}
\begin{lstlisting}
void tarjan(int u) {
  dfn[u] = low[u] = ++_dfn;
  vis[u] = 1;
  sta[++top] = u;
  for (int v : e[u]) {
    if (!dfn[v]) {
      tarjan(v);
      low[u] = min(low[u], low[v]);
    } else if (vis[v]) {
      low[u] = min(low[u], low[v]);
    }
  }
  if (dfn[u] == low[u]) {
    w_col[++_col] = 0;
    do {
      col[sta[top]] = _col;
      vis[sta[top]] = 0;
      w_col[_col] += w[sta[top]];
    } while (sta[top--] != u);
  }
}

inline void suodian() {
  for (int i = 1; i <= n; ++i) {
    if (!dfn[i]) tarjan(i);
  }
  for (int i = 1; i <= n; ++i) {
    for (int j : e[i]) {
      if (col[i] == col[j]) continue;
      e_col[col[i]].push_back(col[j]);
    }
  }
}
\end{lstlisting}
\section{2-SAT}
\subsection{SCC Tarjan}
$O(n+m)$
从0开始 
{\setmainfont{Consolas}
\begin{lstlisting}
struct TWO_SAT {
  int top, _dfn, _col;
  int dfn[N<<1], low[N<<1], vis[N<<1], sta[N<<1], col[N<<1], res[N];
  vector<int> e[N<<1];
  void init(const int &n) {
    top = 0;
    memset(dfn, 0, sizeof(int)*n*2);
    memset(low, 0, sizeof(int)*n*2);
    memset(vis, 0, sizeof(int)*n*2);
    for (int i = 0; i < n<<1; ++i) vector<int>().swap(e[i]);
  }
  // if u then v
  void add_edge(const int &u, const int &v) {
    e[u].emplace_back(v);
  }
  void add_edge(const int &u, const int &uv, const int &v, const int &vv) {
    e[u<<1^uv].emplace_back(v<<1^vv);
  }
  // pt i ==> i<<1 && i<<1|1 ==> 0 && 1
  inline bool work(const int &n) {
    for (int i = 0; i <= n<<1; ++i)
      if (!dfn[i]) tarjan(i);
    for (int i = 0; i < n; ++i) {
      if (col[i<<1] == col[i<<1|1]) return false;
      res[i] = col[i<<1] > col[i<<1|1];
    }
    return true;
  }
  void tarjan(const int &u) {
    dfn[u] = low[u] = ++_dfn;
    vis[u] = 1;
    sta[++top] = u;
    for (int &v : e[u]) {
      if (!dfn[v]) {
        tarjan(v);
        low[u] = min(low[u], low[v]);
      } else if (vis[v]) {
        low[u] = min(low[u], low[v]);
      }
    }
    if (dfn[u] == low[u]) {
      ++_col;
      do {
        col[sta[top]] = _col;
        vis[sta[top]] = 0;
      } while (sta[top--] != u);
    }
  }
};
\end{lstlisting}
\subsection{DFS}
$O(nm)$
所求结果字典序最小 
{\setmainfont{Consolas}
\begin{lstlisting}
struct TWO_SAT {
  int n, cnt;
  int res[N], mem[N<<1], mark[N<<1];
  vector<int> e[N<<1];
  void init(const int &_n) {
    n = _n;
    memset(mark, 0, sizeof(int)*n*2);
    for (int i = 0; i < n<<1; ++i) vector<int>().swap(e[i]);
  }
  // if u then v
  void add_edge(const int &u, const int &v) {
    e[u].emplace_back(v);
  }
  // pt i ==> i<<1 && i<<1|1 ==> 0 && 1
  void add_edge(const int &u, const int &uv, const int &v, const int &vv) {
    e[u<<1|uv].emplace_back(v<<1|vv);
  }
  // tag 0 any 1 smallest
  bool work() {
    for (int i = 0; i < n; ++i) {
      if (mark[i<<1] || mark[i<<1|1]) continue;
      cnt = 0;
      if (!dfs(i<<1)) {
        while (cnt) mark[mem[cnt--]] = 0;
        if (!dfs(i<<1|1)) return false;
      }
    }
    for (int i = 0; i < n<<1; ++i) if (mark[i]) res[i>>1] = i&1;
    return true;
  }
  bool dfs(const int &u) {
    if (mark[u^1]) return false;
    if (mark[u]) return true;
    mark[mem[++cnt] = u] = 1;
    for (int v : e[u]) if (!dfs(v)) return false;
    return true;
  }
};
\end{lstlisting}


\part{数论}
\section{快排 }
{\setmainfont{Consolas}
\begin{lstlisting}
void quick_sort(int l, int r) {
  if(l >= r) return;
  swap(a[l], a[l+rand()%(r-l)]);
  int i = l, j = r, mid = a[l];
  while(i < j) {
    while(i < j && a[j] >= mid) --j;
    swap(a[i], a[j]);
    while(i < j && a[i] < mid) ++i;
    swap(a[i], a[j]);
  }
  quick_sort(l, i-1);
  quick_sort(i+1, r);
}
\end{lstlisting}


\section{求第K大数}
{\setmainfont{Consolas}
\begin{lstlisting}
int kth_element(int l, int r, int k) {
  if(l == r) return a[l];
  swap(a[l], a[l+rand()%(r-l)]);
  int mid = a[l], i = l, j = r;
  while(i < j) {
    while(i < j && a[j] >= mid) --j;
    swap(a[i], a[j]);
    while(i < j && a[i] < mid) ++i;
    swap(a[i], a[j]);
  }
  a[i] = mid;
  if(i == k) return mid;
  else if(i > k) return kth_element(l, i-1, k);
  else return kth_element(i+1, r, k);
}
\end{lstlisting}
**STL** (排序,无返回值) 
{\setmainfont{Consolas}
\begin{lstlisting}
nth_element(a+1, a+k+1, a+n+1);
\end{lstlisting}


\section{求逆序对(归并排序) }
{\setmainfont{Consolas}
\begin{lstlisting}
void merge_sort(int l, int r) {
  if(l == r) return;
  int mid = (l+r)>>1;
  merge_sort(l, mid);
  merge_sort(mid+1, r);
  int i = l, j = mid+1, k = l;
  while(k <= r) {
    if(j <= r && (i > mid || a[j] < a[i])) {
      ans += mid-i+1;
      b[k++] = a[j++];
    }
    else b[k++] = a[i++];
  }
  memcpy(a+l, b+l, sizeof(int)*(r-l+1));
}
\end{lstlisting}


\section{线性基}
{\setmainfont{Consolas}
\begin{lstlisting}
template <typename T>
struct LinearBase {
  int sz = sizeof(T)*8, zero;
  T tot;
  vector<T> b, rb, p;
  LinearBase(){ init(); }
  void init() {
    tot = zero = 0;;
    vector<T>(sz, 0).swap(b);
    vector<T>().swap(rb);
    vector<T>().swap(p);
  }
  template <typename TT>
  void build(TT a[], const int &n) {
    init();
    for (int i = 1; i <= n; ++i) insert(a[i]);
  }
  void merge(const LinearBase xj) {
    for (int i : xj.b) if (i) insert(i);
  }
  void insert(T x) {
    for (int i = sz-1; i >= 0; --i) if ((x>>i)&1) {
      if (!b[i]) { b[i] = x; return; }
      x ^= b[i];
    }
    zero = 1;
  }
  bool find(T x) {
    for (int i = sz-1; i >= 0; --i) if ((x>>i)&1) {
      if (!b[i]) { return false; }
      x ^= b[i];
    }
    return true;
  }
  T max_xor() {
    T res = 0;
    for (int i = sz-1; i >= 0; --i)
      if (~(res>>i)&1) res ^= b[i];
      // res = max(res, res^b[i]);
    return res;
  }
  T min_xor() {
    if (zero) return 0;
    for (int i = 0; i < sz; ++i)
      if (b[i]) return b[i];
  }
  void rebuild() {
    rb = b;
    vector<T>().swap(p);
    for (int i = sz-1; i >= 0; --i)
      for (int j = i-1; j >= 0; --j)
        if ((rb[i]>>j)&1) rb[i] ^= rb[j];
    for (int i = 0; i < sz; ++i)
      if (rb[i]) p.emplace_back(rb[i]);
    tot = ((T)1<<p.size())+zero;
  }
  T kth_min(T k) {
    if (k >= tot || k < 1) return -1;
    if (zero && k == 1) return 0;
    if (zero) --k;
    T res = 0;
    for (int i = (int)p.size()-1; i >= 0; --i)
      if ((k>>i)&1) res ^= p[i];
    return res;
  }
  T kth_max(const T &k) {
    return kth_min(tot-k);
  }
};
\end{lstlisting}

{\setmainfont{Consolas}
\begin{lstlisting}
template <class T>
struct PreSumLB {
  int tot, sz = sizeof(T)*8;
  vector<T> b[N];
  vector<int> p[N];
  PreSumLB() { init(); }
  void init() {
    tot = 0;
    vector<T>(sz, 0).swap(b[0]);
    vector<int>(sz, 0).swap(p[0]);
  }
  void append(T val) {
    int pos = ++tot;
    vector<T> &bb = b[tot];
    vector<int> &pp = p[tot];
    pp = p[tot-1];
    bb = b[tot-1];
    for (int i = sz-1; i >= 0; --i) if ((val>>i)&1) {
      if (bb[i]) {
        if (pos > pp[i]) swap(pos, pp[i]), swap(val, bb[i]);
        val ^= bb[i];
      } else {
        bb[i] = val;
        pp[i] = pos;
        return;
      }
    }
  }
  T query(const int &l, const int &r) {
    T res = 0;
    vector<T> &bb = b[r];
    vector<int> &pp = p[r];
    for (int i = sz-1; i >= 0; --i)
      if (pp[i] >= l) res = max(res, res^bb[i]);
    return res;
  }
};
\end{lstlisting}


\section{矩阵}
\subsection{矩阵快速幂}
\subsection{矩阵求逆 }
{\setmainfont{Consolas}
\begin{lstlisting}
template <typename T>
struct Martix {
  int n, m;
  T a[N][N];
  Martix(){}
  Martix(const int &_n) : n(_n), m(_n) { init(); }
  Martix(const int &_n, const int &_m) : n(_n), m(_m) { init(); }
  T* operator [] (const int &i) { return a[i]; }
  void init(const int &tag = 0) {
    for (int i = 1; i <= n; ++i) memset(a[i], 0, sizeof(T)*(n+1));
    for (int i = 1; i <= n; ++i) a[i][i] = tag;
  }
  friend Martix operator * (const Martix &m1, const Martix &m2) {
    Martix res(m1.n, m2.m);
    for (int i = 1; i <= res.n; ++i)
      for (int j = 1; j <= res.m; ++j)
        for (int k = 1; k <= m1.m; ++k)
          res.a[i][j] = (res.a[i][j]+m1.a[i][k]*m2.a[k][j])%MOD;
    return res;
  }
  Martix& operator *= (const Martix &mx) { return *this = *this*mx; }
  template <typename TT>
  Martix pow(const TT &p) const {
    Martix res(n, m), a = *this;
    res.init(1);
    for (TT i = p; i; i >>= 1, a *= a) if (i&1) res *= a;
    return res;
  }
  Martix inv() const {
    Martix res = *this;
    vector<int> is(n+1), js(n+1);
    for (int k = 1; k <= n; ++k) {
      for (int i = k; i <= n; ++i)
        for (int j = k; j <= n; ++j) if (res.a[i][j]) {
          is[k] = i; js[k] = j; break;
        }
      for (int i = 1; i <= n; ++i) swap(res.a[k][i], res.a[is[k]][i]);
      for (int i = 1; i <= n; ++i) swap(res.a[i][k], res.a[i][js[k]]);
      if (!res.a[k][k]) return Martix(0);
      res.a[k][k] = mul_inverse(res.a[k][k]); // get inv of number
      for (int j = 1; j <= n; ++j) if (j != k)
        res.a[k][j] = res.a[k][j]*res.a[k][k]%MOD;
      for (int i = 1; i <= n; ++i) if (i != k)
        for (int j = 1; j <= n; ++j) if (j != k)
          res.a[i][j] = (res.a[i][j]+MOD-res.a[i][k]*res.a[k][j]%MOD)%MOD;
      for (int i = 1; i <= n; ++i) if (i != k)
        res.a[i][k] = (MOD-res.a[i][k]*res.a[k][k]%MOD)%MOD;
    }
    for (int k = n; k; --k) {
      for (int i = 1; i <= n; ++i) swap(res.a[js[k]][i], res.a[k][i]);
      for (int i = 1; i <= n; ++i) swap(res.a[i][is[k]], res.a[i][k]);
    }
    return res;
  }
  T det() {
    long long res = 1;
    Martix cpy = *this;
    for (int i = 1; i <= n; ++i) {
      for (int j = i+1; j <= n; ++j) while (cpy.a[j][i]) {
        long long t = cpy.a[i][i]/cpy.a[j][i];
        for (int k = i; k <= n; ++k)
          cpy.a[i][k] = (cpy.a[i][k]+MOD-t*cpy.a[j][k]%MOD)%MOD;
        swap(cpy.a[i], cpy.a[j]);
        res = -res;
      }
      res = res*cpy.a[i][i]%MOD;
    }
    return (res+MOD)%MOD;
  }
  friend ostream& operator << (ostream &os, Martix<T> &mx) {
    for (int i = 1; i <= mx.n; ++i)
      for (int j = 1; j <= mx.m; ++j)
        os << mx[i][j] << " \n"[j==mx.m];
    return os;
  }
};
\end{lstlisting}
\section{高斯消元 }
{\setmainfont{Consolas}
\begin{lstlisting}
struct GaussElimination {
  double a[N][N];
  void init() { memset(a, 0, sizeof a); }
  void init(const int &n) {
    for (int i = 1; i <= n; ++i)
      for (int j = 1; j <= n+1; ++j)
        a[i][j] = 0;
  }
  // ans is a[i][n+1]
  bool solve(const int &n) {
    for (int i = 1, j, k; i <= n; ++i) {
      for (j = i+1, k = i; j <= n; ++j)
        if (abs(a[j][i]) > abs(a[k][i])) k = j;
      if (abs(a[k][i]) < eps) return false;
      swap(a[k], a[i]);
      for (j = 1; j <= n; ++j) if (i != j) {
        double d = a[j][i]/a[i][i];
        for (k = i+1; k <= n+1; ++k)
          a[j][k] -= d*a[i][k];
      }
    }
    for (int i = 1; i <= n; ++i) a[i][n+1] /= a[i][i];
    return true;
  }
};
\end{lstlisting}
\subsection{异或方程组}

a[i][j] 第i个是否对j有影响

a[i][n+1] 第i个最后被翻转与否 
{\setmainfont{Consolas}
\begin{lstlisting}
// -1 : no solution, 0 : multi , 1 : one
template <typename T>
int XorGauss(T a[N], const int &n) {
  for (int i = 1, j, k; i <= n; ++i) {
    for (k = i; !a[k][i] && k <= n; ++k) {}
    if (k <= n) swap(a[k], a[i]);
    for (j = 1; j <= n; ++j) if (i != j && a[j][i])
      for (k = i; k <= n+1; ++k) a[j][k] ^= a[i][k];
      // a[j] ^= a[i]; // bitset<N> a[N]
  }
  for (int i = 1; i <= n; ++i) if (!a[i][i]) return -a[i][n+1];
  return 1;
}
// dfs(n, 0)
void dfs(const int &u, const int &num) {
  if (num >= res) return;
  if (u <= 0) { res = num; return; }
  if (a[u][u]) {
    int t = a[u][n+1];
    for (int i = u+1; i <= n; ++i) {
      if (a[u][i]) t ^= used[i];
    }
    dfs(u-1, num+t);
  } else { // 自由元
    dfs(u-1, num);
    used[u] = 1;
    dfs(u-1, num+1);
    used[u] = 0;
  }
}
\end{lstlisting}


\section{拉格朗日插值 }
{\setmainfont{Consolas}
\begin{lstlisting}
template <typename T, typename H, typename P>
long long Largrange(const T &k, const int &n, const H x[], const P y[]) {
  long long res = 0, s1 = 1, s2 = 1;
  for (int i = 1; i <= n; ++i, s1 = s2 = 1) {
    for (int j = 1; j <= n; ++j) if (i != j) {
      s1 = s1*(x[i]-x[j]+MOD)%MOD;
      s2 = s2*(k-x[j]+MOD)%MOD;
    }
    res = (res+y[i]*s2%MOD*mul_inverse(s1)%MOD)%MOD;
  }
  return res;
}
\end{lstlisting}
{\setmainfont{Consolas}
\begin{lstlisting}
template <typename T, typename P> // x[i] = i -> y[i] = f(i)
long long Largrange(const T &k, const int &n, const P y[]) {
  if (k <= n) return y[k];
  static long long pre[N], suf[N];
  long long res = 0;
  pre[0] = suf[n+1] = 1;
  for (int i = 1; i <= n; ++i) pre[i] = pre[i-1]*(k-i)%MOD;
  for (int i = n; i >= 1; --i) suf[i] = suf[i+1]*(k-i)%MOD;
  for (int i = 1; i <= n; ++i) {
    res = (res+y[i]*(pre[i-1]*suf[i+1]%MOD)%MOD
      *mul_inverse(((n-i)&1 ? -1 : 1)*fac[i-1]*fac[n-i]%MOD)%MOD)%MOD;
  }
  return (res+MOD)%MOD;
}
\end{lstlisting}


\section{快速幂 }
{\setmainfont{Consolas}
\begin{lstlisting}
template <typename T, typename H>
inline T qpow(const T &a, const H &p, const int &mo = MOD) {
  long long res = 1, x = a;
  for (H i = p; i; i >>= 1, x = x*x%mo)
    if (i&1) res = res*x%mo;
  return static_cast<T>(res);
}
\end{lstlisting}


\section{快速乘 }
{\setmainfont{Consolas}
\begin{lstlisting}
inline long long qmul(long long x, long long y, long long mo) {
  long long res = 0;
  while (y) {
    if (y&1) res = (res+x)%mo;
    x = (x<<1)%mo;
    y >>= 1;
  }
  return res;
}
\end{lstlisting}
{\setmainfont{Consolas}
\begin{lstlisting}
inline long long qmul(long long x, long long y, long long mo) {
  return (long long)((__int128)x*y%mo);
}
\end{lstlisting}
{\setmainfont{Consolas}
\begin{lstlisting}
inline long long qmul(long long x, long long y, long long mo) {
  // x*y - floor(x*y/mo)*mo
  typedef unsigned long long ull;
  typedef long double ld;
  return ((ull)x*y-(ull)((ld)x/mo*y)*mo+mo)%mo;
}
\end{lstlisting}


\section{复数 }
{\setmainfont{Consolas}
\begin{lstlisting}
struct comp {
  typedef double T; // maybe long double ?
  T real, imag;
  comp (const double &_real = 0, const double &_imag = 0) : real(_real), imag(_imag) {}
  friend comp operator + (const comp &c1, const comp &c2) { return comp(c1.real+c2.real, c1.imag+c2.imag); }
  friend comp operator - (const comp &c1, const comp &c2) { return comp(c1.real-c2.real, c1.imag-c2.imag); }
  friend comp operator * (const comp &c1, const comp &c2) { return comp(c1.real*c2.real-c1.imag*c2.imag, c1.real*c2.imag+c1.imag*c2.real); }
  comp& operator += (const comp &c) { return *this = *this+c; }
  comp& operator -= (const comp &c) { return *this = *this-c; }
  comp& operator *= (const comp &c) { return *this = *this*c; }
  friend istream& operator >> (istream &is, comp &c) { return is >> c.real >> c.imag; }
  friend ostream& operator << (ostream &os, comp &c) { return os << c.real << setiosflags(ios::showpos) << c.imag << "i";}
  comp conjugate() { return comp(real, -imag); }
  friend comp conjugate(const comp &c) { return comp(c.real, -c.imag); }
};
\end{lstlisting}


\section{快速傅里叶变换|FFT }
{\setmainfont{Consolas}
\begin{lstlisting}
// array [0, n)
namespace FFT {
  static const int SIZE = (1<<18)+3;
  int len, bit;
  int rev[SIZE];
  // #define comp complex<long double>
  void fft(comp a[], int flag = 1) {
    for (int i = 0; i < len; ++i)
      if (i < rev[i]) swap(a[i], a[rev[i]]);
    for (int base = 1; base < len; base <<= 1) {
      comp w, wn = {cos(PI/base), flag*sin(PI/base)};
      for (int i = 0; i < len; i += base*2) {
        w = { 1.0, 0.0 };
        for (int j = 0; j < base; ++j) {
          comp x = a[i+j], y = w*a[i+j+base];
          a[i+j] = x+y;
          a[i+j+base] = x-y;
          w *= wn;
        }
      }
    }
  }
  void work(comp f[], const int &n, comp g[], const int &m) {
    len = 1; bit = 0;
    while (len < n+m) len <<= 1, ++bit;
    // multi-testcase
    for (int i = n; i < len; ++i) f[i] = 0;
    for (int i = m; i < len; ++i) g[i] = 0;
    for (int i = 0; i < len; ++i)
      rev[i] = (rev[i>>1]>>1)|((i&1)<<(bit-1));
    fft(f, 1); fft(g, 1);
    for (int i = 0; i < len; ++i) f[i] *= g[i];
    fft(f, -1);
    for (int i = 0; i < n+m; ++i) f[i].real /= len;
  }
  /*
  template <class T>
  void work(T a[], const int &n) {
    static comp f[SIZE];
    len = 1; bit = 0;
    while (len < n+n) len <<= 1, ++bit;
    // multi-testcase
    for (int i = 0; i < n; ++i) f[i] = a[i];
    for (int i = n; i < len; ++i) f[i] = 0;
    for (int i = 0; i < len; ++i) rev[i] = (rev[i>>1]>>1)|((i&1)<<(bit-1));
    fft(f, 1);
    for (int i = 0; i <= len; ++i) f[i] *= f[i];
    fft(f, -1);
    for (int i = 0; i < n+n; ++i) a[i] = static_cast<T>(f[i].real/len+.5);
  }
  */
}
\end{lstlisting}


\section{快速数论变换|NTT }
{\setmainfont{Consolas}
\begin{lstlisting}
// array [0, n)
namespace NTT {
  static const int SIZE = (1<<18)+3;
  const int G = 3;
  int len, bit;
  int rev[SIZE];
  long long f[SIZE], g[SIZE];
  template <class T>
  void ntt(T a[], int flag = 1) {
    for (int i = 0; i < len; ++i)
      if (i < rev[i]) swap(a[i], a[rev[i]]);
    for (int base = 1; base < len; base <<= 1) {
      long long wn = qpow(G, (MOD-1)/(base*2)), w;
      if (flag == -1) wn = qpow(wn, MOD-2);
      for (int i = 0; i < len; i += base*2) {
        w = 1;
        for (int j = 0; j < base; ++j) {
          long long x = a[i+j], y = w*a[i+j+base]%MOD;
          a[i+j] = (x+y)%MOD;
          a[i+j+base] = (x-y+MOD)%MOD;
          w = w*wn%MOD;
        }
      }
    }
  }
  template <class T>
  void work(T a[], const int &n, T b[], const int &m) {
    len = 1; bit = 0;
    while (len < n+m) len <<= 1, ++bit;
    for (int i = 0; i < n; ++i) f[i] = a[i];
    for (int i = n; i < len; ++i) f[i] = 0;
    for (int i = 0; i < m; ++i) g[i] = b[i];
    for (int i = m; i < len; ++i) g[i] = 0;
    for (int i = 0; i < len; ++i)
      rev[i] = (rev[i>>1]>>1)|((i&1)<<(bit-1));
    ntt(f, 1); ntt(g, 1);
    for (int i = 0; i < len; ++i) f[i] = f[i]*g[i]%MOD;
    ntt(f, -1);
    long long inv = qpow(len, MOD-2);
    for (int i = 0; i < n+m-1; ++i) f[i] = f[i]*inv%MOD;
  }
}
\end{lstlisting}
\section{任意模数NTT|MTT }
{\setmainfont{Consolas}
\begin{lstlisting}
namespace MTT {
  static const int SIZE = (1<<18)+7;
  int Mod = MOD;
  comp w[SIZE];
  int bitrev[SIZE];
  long long f[SIZE];
  void fft(comp *a, const int &n) {
    for (int i = 0; i < n; ++i) if (i < bitrev[i]) swap(a[i], a[bitrev[i]]);
    for (int i = 2, lyc = n >> 1; i <= n; i <<= 1, lyc >>= 1)
      for (int j = 0; j < n; j += i) {
        comp *l = a + j, *r = a + j + (i >> 1), *p = w;
        for (int k = 0; k < i>>1; ++k) {
          comp tmp = *r * *p;
          *r = *l - tmp, *l = *l + tmp;
          ++l, ++r, p += lyc;
        }
      }
  }
  template <class T>
  inline void work(T *x, const int &n, T *y, const int &m) {
    static int bit, L;
    static comp a[SIZE], b[SIZE];
    static comp dfta[SIZE], dftb[SIZE];

    for (L = 1, bit = 0; L < n+m-1; ++bit, L <<= 1);
    for (int i = 0; i < L; ++i) bitrev[i] = bitrev[i >> 1] >> 1 | ((i & 1) << (bit - 1));
    for (int i = 0; i < L; ++i) w[i] = comp(cos(2 * PI * i / L), sin(2 * PI * i / L));

    for (int i = 0; i < n; ++i) (x[i] += Mod) %= Mod, a[i] = comp(x[i] & 32767, x[i] >> 15);
    for (int i = n; i < L; ++i) a[i] = 0;
    for (int i = 0; i < m; ++i) (y[i] += Mod) %= Mod, b[i] = comp(y[i] & 32767, y[i] >> 15);
    for (int i = m; i < L; ++i) b[i] = 0;
    fft(a, L), fft(b, L);
    for (int i = 0; i < L; ++i) {
      int j = (L - i) & (L - 1);
      static comp da, db, dc, dd;
      da = (a[i] + conjugate(a[j])) * comp(.5, 0);
      db = (a[i] - conjugate(a[j])) * comp(0, -.5);
      dc = (b[i] + conjugate(b[j])) * comp(.5, 0);
      dd = (b[i] - conjugate(b[j])) * comp(0, -.5);
      dfta[j] = da*dc + da*dd*comp(0, 1);
      dftb[j] = db*dc + db*dd*comp(0, 1);
    }
    for (int i = 0; i < L; ++i) a[i] = dfta[i];
    for (int i = 0; i < L; ++i) b[i] = dftb[i];
    fft(a, L), fft(b, L);
    for (int i = 0; i < L; ++i) {
      int da = (long long)(a[i].real / L + 0.5) % Mod;
      int db = (long long)(a[i].imag / L + 0.5) % Mod;
      int dc = (long long)(b[i].real / L + 0.5) % Mod;
      int dd = (long long)(b[i].imag / L + 0.5) % Mod;
      f[i] = (da + ((long long)(db + dc) << 15) + ((long long)dd << 30)) % Mod;
    }
    for (int i = 0; i < n+m-1; ++i) (f[i] += Mod) %= Mod;
  }
}

\end{lstlisting}
\section{分治FFT }
{\setmainfont{Consolas}
\begin{lstlisting}
// give g[1, n) ask f[0, n)
// f[i] = sigma f[i-j]*g[j] (1 <= j <= i)
template <class T> // [l, r]
void cdq_fft(T f[], T g[], const int &l, const int &r) {
  if (r-l <= 1) return;
  int mid = (l+r)>>1;
  cdq_fft(f, g, l, mid);
  NTT::work(f+l, mid-l, g, r-l);
  for (int i = mid; i < r; ++i)
    (f[i] += NTT::f[i-l]) %= MOD;
  cdq_fft(f, g, mid, r);
}
// f[0] = 1; cdq_fft(f, g, 0, n);
\end{lstlisting}


\section{第二类斯特林数 }
{\setmainfont{Consolas}
\begin{lstlisting}
inline void stirling(const int &n) {
  S[0][0] = 1;
  // 注意取模
  for (int i = 1; i <= n; ++i)
    for (int j = 1; j <= i; ++j)
      S[i][j] = S[i-1][j-1]+S[i-1][j]*j;
}
\end{lstlisting}
{\setmainfont{Consolas}
\begin{lstlisting}
void stirling(const int &n) {
  inv[0] = inv[1] = 1;
  for(int i = 2; i <= n; ++i)
    inv[i] = MOD-MOD/i*inv[MOD%i]%MOD;
  for (int i = 1; i <= n; ++i)
    inv[i] = inv[i-1]*inv[i]%MOD;
  while (len <= (n<<1)) len <<= 1, ++bit;
  for (int i = 0; i < len; ++i)
    rev[i] = (rev[i>>1]>>1)|((i&1)<<(bit-1));
  for (int i = 0, one = 1; i <= n; ++i, one = MOD-one) {
    f[i] = one*inv[i]%MOD;
    g[i] = qpow(i, n)*inv[i]%MOD;
  }
  NTT(f, 1); NTT(g, 1);
  for (int i = 0; i < len; ++i) f[i] = f[i]*g[i]%MOD;
  NTT(f, -1);
  long long invv = qpow(len, MOD-2);
  for (int i = 0; i <= n; ++i)
    printf("%lld%c", f[i]*invv%MOD, " \n"[i==n]);
}
\end{lstlisting}


\section{约瑟夫环}
\subsection{O(n) }
{\setmainfont{Consolas}
\begin{lstlisting}
int solve(int n, int v) { return n == 1 ? 0 : (solve(n-1, v)+v)%n; }
// res = solve(num, step)+1
\end{lstlisting}


\section{最大公因数 gcd }
{\setmainfont{Consolas}
\begin{lstlisting}
__gcd(a, b); // <algorithm>
int gcd(int a, int b) { return b ? gcd(b, a%b) : a; }
inline int gcd(int a, int b) { while (b) a %= b, swap(a, b); return a; }
\end{lstlisting}
\section{最小公倍数 lcm}
$LCM(\frac{a}{b},\frac{c}{d})=\frac{LCM(a, c)}{GCD(b,d)}$

$LCM(\frac{a_1}{b_1},\frac{a_2}{b_2},...)=\frac{LCM(a1, a2,...)}{GCD(b1, b2,...)}$ 
{\setmainfont{Consolas}
\begin{lstlisting}
inline int lcm(int a, int b) { return a/gcd(a, b)*b; }
\end{lstlisting}
\section{扩展欧几里得(同余方程)}
{\setmainfont{Consolas}
\begin{lstlisting}
template <typename T>
T exgcd(const T a, const T b, T &x, T &y) {
  if (!b) { x = 1; y = 0; return a; }
  T d = exgcd(b, a%b, y, x);
  y -= a/b*x;
  return d;
}
\end{lstlisting}


\section{乘法逆元}
\subsection{拓展欧几里得 }
{\setmainfont{Consolas}
\begin{lstlisting}
template <typename T>
inline T mul_inverse(const T &a, const T &mo = MOD) {
  T x, y;
  exgcd(a, mo, x, y);
  return (x%mo+mo)%mo;
}
\end{lstlisting}
\subsection{费马小定理 }
{\setmainfont{Consolas}
\begin{lstlisting}
template <typename T>
inline T mul_inverse(const T &a, const int &mo = MOD) {
  return qpow(a, mo-2);
}
\end{lstlisting}
\subsection{线性递推 }
{\setmainfont{Consolas}
\begin{lstlisting}
template <typename T>
inline void mul_inverse(T *inv, int mod = MOD) {
  inv[0] = inv[1] = 1;
  for(int i = 2; i <= n; ++i)
    inv[i] = 1ll*(mod-mod/i)*inv[mod%i]%mod;
}
\end{lstlisting}


\section{中国剩余定理}
\subsection{中国剩余定理CRT(m互质) }
{\setmainfont{Consolas}
\begin{lstlisting}
inline long long CRT(int a[], int m[]) {
  long long res = 0, M = 1;
  for (int i = 1; i <= n; ++i)
    M *= m[i];
  for (int i = 1; i <= n; ++i)
    res = (res + a[i]*(M/m[i])*mul_inverse(M/m[i], m[i]))%M;
  return (res+M)%M;
}
\end{lstlisting}
\subsection{扩展中国剩余定理EXCRT(m不互质) }
{\setmainfont{Consolas}
\begin{lstlisting}
inline long long EXCRT(long long a[], long long m[]) {
  // M*x + m[i]*y = a[i]-res (mod m[i])
  // res = res+x*M;
  long long M = m[1], res = a[1], x, y, c, d;
  for (int i = 2; i <= n; ++i) {
    d = exgcd(M, m[i], x, y);
    c = (a[i]-res%m[i]+m[i])%m[i];
    if (c%d != 0) return -1;
    x = (c/d)*x%(m[i]/d);
    res += x*M;
    M *= m[i]/d;
    res = (res%M+M)%M;
  }
  return res;
}

\end{lstlisting}


\section{排列组合}
\subsection{奇偶性}
C(n,k) 当 n\&k == k 为奇数反之偶数


\section{欧拉函数 }
{\setmainfont{Consolas}
\begin{lstlisting}
inline long long phi(long long x) {
  long long res = x;
  for (long long i = 2; i*i <= x; ++i) {
    if (x%i) continue;
    res = res/i*(i-1);
    while (x%i == 0) x /= i;
  }
  if (x > 1) res = res/x*(x-1);
  return res;
}
\end{lstlisting}
\subsection{筛法 }
{\setmainfont{Consolas}
\begin{lstlisting}
struct Euler {
  int phi[N], check[N];
  vector<int> prime;
  void init(int sz) {
    for (int i = 1; i <= sz; ++i) check[i] = 1;
    phi[1] = 1; check[1] = 0;
    for (int i = 2; i <= sz; ++i) {
      if (check[i]) {
        prime.emplace_back(i);
        phi[i] = i-1;
      }
      for (int j : prime) {
        if (i*j > sz) break;
        check[i*j] = 0;
        if (i%j) {
          phi[i*j] = (j-1)*phi[i];
        } else {
          phi[i*j] = j*phi[i];
          break;
        }
      }
    }
  }
} E;
\end{lstlisting}


\section{线性筛 }
{\setmainfont{Consolas}
\begin{lstlisting}
struct Euler {
  int tot = 0;
  int prime[N];
  bool check[N];
  bool& operator [] (const int i) { return check[i]; }
  void init(int sz) {
    tot = 0;
    for (int i = 1; i <= sz; ++i) check[i] = true;
    check[1] = false;
    for (register int i = 2, j; i <= sz; ++i) {
      if (check[i]) prime[++tot] = i;
      for (j = 1; j <= tot && i*prime[j] <= sz; ++j) {
        check[i*prime[j]] = false;
        if (i%prime[j] == 0) break;
      }
    }
  }
} E;
\end{lstlisting}


\section{判断素数(质数)}
 某较优方法 
{\setmainfont{Consolas}
\begin{lstlisting}
inline bool is_prime(long long x) {
  if(x == 1) return false;
  if(x == 2 || x == 3) return true;
  if(x%6 != 1 && x%6 != 5) return false;
  for(long long i = 5; i*i <= x; i += 6)
    if(x%i == 0 || x%(i+2) == 0) return false;
  return true;
}
\end{lstlisting}
\subsection{Miller-Rabin 素性测试 }
{\setmainfont{Consolas}
\begin{lstlisting}
inline bool MillerRabin(int x) {
  static const int test_time = 10;
  if (x < 3) return x == 2;
  int a = x-1, b = 0;
  while (!(a&1)) a >>= 1, ++b;
  for (int i = 1, j, v; i <= test_time; ++i) {
    v = (qpow(rnd()%(x-2)+2, a, x));
    if (v == 1 || v == x-1) continue;
    for (j = 0; j < b && v != x-1; ++j)
      v = static_cast<int>(1ll*v*v%x);
    if (j >= b) return false;
  }
  return true;
}
\end{lstlisting}

\section{BSGS}
{\setmainfont{Consolas}
\begin{lstlisting}
// map<long long, int> mmp; // a^n = x
inline long long BSGS(long long a, long long x, long long m) {
  long long t = (long long)ceil(sqrt(m)); // b = a^i
  for(int i = 0; i < t; ++i)
    mmp[mul(x, qpow(a, i))] = i;
  a = qpow(a, t);
  long long now, ans; // now = (a^t)^i
  for(int i = 0; i <= t; ++i)
  {
    now = qpow(a, i);
    if(mmp.count(now))
    {
      ans = t*i-mmp[now];
      if(ans > 0) return ans;
    }
  }
  return -1;
}
\end{lstlisting}


\part{动态规划 DP}
\section{线性DP}
\subsection{最长上升子序列LIS }
{\setmainfont{Consolas}
\begin{lstlisting}
for(int i = 1; i <= n; ++i) {
  f[i] = 1;
  for(int j = 1; j < i; ++j)
    if(a[i] > a[j]) f[i] = max(f[i],f[j]+1);
}
\end{lstlisting}
\subsection{最长公共子序列LCS }
{\setmainfont{Consolas}
\begin{lstlisting}
f[i][j] = max{f[i-1][j],
              f[i][j-1],
              f[i-1][j-1]+1 (if A[i] == B[j])}
\end{lstlisting}
\subsection{数字三角形}
\section{状压DP}
\subsection{枚举子集 }
{\setmainfont{Consolas}
\begin{lstlisting}
for (int i = s; i; i = (i-1)&s) {}
\end{lstlisting}
\subsection{枚举n个元素大小为k的二进制子集 }
{\setmainfont{Consolas}
\begin{lstlisting}
int s=(1<<k)-1;
while(s<(1<<n)){
  work(s);
  int x=s&-s,y=s+x;
  s=((s&~y)/x>>1)|y; //这里有一个位反~
}
\end{lstlisting}

\section{背包问题}
\subsection{多重背包}
**二进制拆分** 
{\setmainfont{Consolas}
\begin{lstlisting}
for(int i = 1, cnt, vi, wi, m; i <= n; ++i) {
  scanf("%d%d%d", &vi, &wi, &m);
  cnt = 1;
  while(m-cnt > 0) {
    m -= cnt;
    v.push_back(vi*cnt);
    w.push_back(wi*cnt);
    cnt <<= 1;
  }
  v.push_back(vi*m);
  w.push_back(wi*m);
}
for(int i = 0; i < w.size(); ++i)
  for(int j = W; j >= w[i]; --j)
    b[j] = max(b[j], b[j-w[i]]+v[i]);
\end{lstlisting}
**单调队列** 
{\setmainfont{Consolas}
\begin{lstlisting}
for(int i = 1; i <= n; ++i) {
  scanf("%d%d%d", &v, &w, &m);
  for(int u = 0; u < w; ++u) {
    int maxp = (W-u)/w;
    head = 1; tail = 0;
    for(int k = maxp-1; k >= max(0, maxp-m); --k) {
      while(head <= tail && calc(u, q[tail]) <= calc(u, k)) tail--;
      q[++tail] = k;
    }
    for(int p = maxp; p >= 0; --p) {
      while(head <= tail && q[head] >= p) head++;
      if(head <= tail) f[u+p*w] = max(f[u+p*w], p*v+calc(u, q[head]));
      if(p-m-1 < 0) continue;
      while(head <= tail && calc(u, q[tail]) <= calc(u, p-m-1)) tail--;
      q[++tail] = p-m-1;
    }
  }
}
int ans = 0;
for(int i = 1; i <= W; ++i)
  ans = max(ans, f[i]);
\end{lstlisting}


\part{STL}
\section{unordered\_map 重载}
{\setmainfont{Consolas}
\begin{lstlisting}
struct Node {
  int a, b;
  // 重载 ==
  friend bool operator == (const Node &x, const Node &y) {
    return x.a == y.a && x.b == y.b;
  }
};
// 方法一
namespace std {
  template <>
  struct hash<Node> {
    size_t operator () (const Node &x) const {
      return hash<int>()(x.a)^hash<int>()(x.b);
    }
  };
}
unordered_map<Node, int> mp;
// 方法二
struct KeyHasher {
  size_t operator () (const Node &x) const {
    return hash<int>()(x.a)^hash<int>()(x.b);
  }
};
unordered_map<Node, int, KeyHasher> mmp;
\end{lstlisting}
\section{定义函数 }
{\setmainfont{Consolas}
\begin{lstlisting}
function<void(int&, int)> f = [&](int &x, int y) -> void {
  x += y;
};
\end{lstlisting}

\end{document}